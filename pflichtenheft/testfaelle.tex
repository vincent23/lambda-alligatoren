
\section{Globale Testfälle}

\subsection{Funktionssequenzen}

\begin{requirements}

	\req{T110} Ein Benutzerprofil erstellen
	
	
	\begin{itemize}
  			\item Der Anwender befindet sich im Startfenster und klickt den 'Benutzer'-Button (Hauptmenü / A5). Er gelangt so in ein neues Fenster in welchem ihm eine Liste aller zurzeit im System gespeicherten Benutzer präsentiert wird.
  			
  			\item Er scrollt an das Ende dieser Liste und wählt den Listeneintrag mit dem Plus-Symbol (Profilauswahl / L2). Dadurch öffnet sich ein Dialogfenster zur Eingabe eines Benutzernamens.
  			
  			\item Der Benutzer klickt das zur Eingabe des Namens vorgesehene Feld (Profilerstellung (Teil 1) / M1) und gibt über die angezeigt Bildschirmtastatur seinen gewünschten Benutzernamen ein. Anschließend gelangt er durch einen Klick auf den 'Weiter'-Button (Profilerstellung (Teil 1) / M2) in das nächste Dialogfenster in welchem dem Anwender verschiedenen Profilbilder (Profilerstellung (Teil 2) / N1) präsentiert werden.
  			
  			\item Durch einen Klick wählte er eines dieser Bilder und bestätigt mit dem Drücken des 'Weiter'-Buttons (Profilerstellung (Teil 2) / N2) seine Eingaben. Im System ist nun ein neues Benutzerprofil mit dem von ihm gewählten Namen und Profilbild gespeichert.
  	\end{itemize}
  	
  	
	
	\req{T120} Ein Benutzerprofil editieren
	
	
	\begin{itemize}
  			\item Der Anwender wählt im Startfenster den 'Einstellungen'-Button (Hauptmenü / A2) und kommt so in das Einstellungsmenü.
  			
  			\item Hier klickt er den 'Profil Editieren'-Button (Einstellungsmenü / J6) und wechselt dadurch in die Listenansicht der im System gespeicherten Benutzerprofile. Er wählt nun einen der Listeneinträge aus wodurch ein analog zu T110 ablaufenden Dialog zur Eingabe eines Benutzernamens und der Wahl eines Profilbilds gestartet wird.
  			
  			\item Der Benutzer gibt einen neuen Namen ein, wählt ein neues Profilbild und speichert abschließend durch einen Klick des 'Weiter'-Buttons (Profilerstellung (Teil 2) / N2) die von ihm gemachten Änderungen im System.
  	\end{itemize}
  	
	
	
	\req{T130} Ein Benutzerprofil löschen
	
	
	\begin{itemize}
  			\item Im Einstellungsmenü klickt der Anwender den 'Profil Editieren'-Button (Einstellungsmenü / J6) und wechselt so in die Listenansicht der im System gespeicherten Benutzerprofile. 
  			
  			\item Er wählt nun einen der Listeneinträge und klickt auf das 'X'-Symbol rechts neben dem Namen. Dadurch öffnet sich ein Dialog und der Benutzer bejaht die darin gestellte Frage, ob er das gewählte Profil wirklich löschen will, mit einem Klick auf den 'Weiter'-Button. Das gewählte Profil und alle damit verbundenen Daten sind so aus dem System gelöscht.
  			
	\end{itemize}
	

	\req{T140} Ein Level starten
	
	
	\begin{itemize}
  			\item Im Startmenü wählt der Spieler den 'Spielen'-Button (Startmenü / A1) und öffnet dadurch die Levelübersicht.
  			
  			\item Hier klickt er einen der Levelblöcke (Levelübersicht / B2) und gelangt so in die Leveldetailansicht, in dem ihm alle im Block zusammengefassten Level präsentiert werden.
  			
  			\item Durch einen Klick auf ein bereits freigeschaltetes Level (Leveldetailübersicht / C2) startet er dieses und wechselt so in den zum gewählten Level gehörenden Platziermodus.
  	\end{itemize}
  	
  	
  	
  	\req{T150} Den Platziermodus bedienen
  	
	
	\begin{itemize}
			\item Nach dem Start eines Levels findet sich der Spieler im Platziermodus wieder. Per Drag and Drop zieht der er einzelne Spielelemente (Level (Platziermodus) / D7) auf die Arbeitsfläche (Level (Platziermodus) / D5) und verändert so die Ausgangskonstellation.
  			
  			\item Er klickt nun ein besonders gekennzeichneten Alligator an und öffnet so ein Kontextmenü in dem er die Möglichkeit hat die Farbe des Alligators neu zu wählen.
  			
  			\item Durch einen Klick auf eine der angezeigten Farben schließt sich das Kontextmenü wieder und Alligator hat nun die gewählte Farbe.
  			
  			\item Der Anwender benutzt nun den 'Zoom-In'-Button (Level (Platziermodus) / D1) und vergrößert so die auf der Arbeitsfläche dargestellten Elemente was er dann wieder durch den Klick des 'Zoom-Out'-Buttons  (Level (Platziermodus) / D2) wieder rückgängig macht.
  			
  			\item Er wählt jetzt das sich am oberen Bildschirmrand befindlichen Menüelement aus und zieht es nach unten auf den Bildschirm. Im dadurch entstandenen Bildschirmelement ist so die zu erreichende Endkonstellation des Levels angezeigt, welche durch Loslassen des Bildschirms anschließend wieder verschwindet. 
  			
  			\item Der Benutzer klick nun den 'Spielmenü'-Button (Level (Platziermodus) / D2) und wechselt so in das Spielmenü.
  			
  			\item Hier wählt er den Eintrag 'Zurücksetzen' (Spielmenü / G2) was den Neustart des Levels zur Folge hat. Aller bisher im Level gemachter Fortschritt wird dadurch verworfen.
  			
  			\item Er klickt anschließend den 'Tipp'-Button (Level (Platziermodus) / D3) und bekommt so einen Tipp zum derzeitigen Level angezeigt. Durch einen das Drücken des Klick des 'Next'-Buttons wird dieser wieder ausblendet.
  			
  			\item Der Spieler beginnt von neuem die auf der Arbeitsfläche präsentierte Ausgangskonstellation mithilfe der ihm zu Verfügung stehenden Spielelemente zu verändern. Abschließend wechselt er durch einen Klick auf den 'Start'-Button (Level (Platziermodus) / D6) in den Simulationsmodus.
  	\end{itemize}
  	
  	
  	
  	\req{T160} Den Simulationsmodus bedienen
  	
	
	\begin{itemize}
	
  			\item Der Benutzer befindet sich im Simulationsmodus und drückt den 'Start'-Button (Level (Simulationsmodus) / B5). Er startet so die automatische Auswertung der von ihm im Platziermodus erstellten Konstellation, welche auf der Arbeitsfläche (Level (Simulationsmodus) / E5) angezeigt wird.
  			
  			\item Durch den erneuten Klick des Buttons stoppt er die Auswertung wieder.
  			
  			\item Anschließend benutzt er den 'Vorwärts'-Button (Level (Simulationsmodus) / E7) um einen einzigen Schritt der Auswertung auszuführen, welchen er dann durch einen Klick des 'Rückwärts'-Buttons (Level (Simulationsmodus) / E3) wieder rückgängig macht.
  			
  			\item Jetzt verändert er die Geschwindigkeit der automatischen Simulation über eine Scrollbar in der Menüleiste und lässt diese dann wieder durch einen Klick des 'Start'-Buttons in der von ihm eingestellten Geschwindigkeit ablaufen. 
  			
  			\item Als der Endzustand durch die Auswertung erreicht wird, stoppt diese und der Anwender bekommt über ein Dialogfenster angezeigt, dass er das Level erfolgreich abgeschlossen hat. Mit einen Klick des 'Level'-Buttons (Levelerfolg / F2) wechselt er abschließend in in die Leveldetailansicht.
  			
  			
  	\end{itemize}
  			

  	
  	\req{T170} Informationen über Benutzer abrufen
  	
  	
  	\begin{itemize}
  	
		\item Der Anwender befindet sich im Startmenü und wählt den 'Statistiken'-Button (Startmenü / A3) und wechselt so in das 'Statistiken'-Fenster
		
		\item Hier wählt er über eine Dropdown-Liste (Statistiken / K1) ein Benutzerprofil und über die Tabauswahl (Statistiken / K3) die Informationskategorie 'Spielzeit'. Ihm werden nun im dafür vorgesehenen Feld (Statistiken / K4) alle im System gespeicherten Informationen über die Spielzeit des ausgewählten Benutzer präsentiert.
		
		\item Der Nutzer wählt jetzt über die Drop-Down Liste ein anderes Benutzerprofil und bekommt so entsprechen Informationen angezeigt, welche sich dann durch die anschließende Wahl der Informationskategorie 'Spielfortschritt' wieder verändern.
		
  	
  	\end{itemize}
  	
  	
  	
  	\req{T180} Einstellungen ändern und speichern
  	
  	
  	\begin{itemize}
  	
		\item Im Startmenü klickt der Anwender den 'Einstellungen'-Button und wechselt so in das entsprechende Fenster.
		
		\item Im jetzt eingeblendeten Bildschirm verändert der Nutzer mithilfe von Scrollbars (Einstellungen / J4, J5) die Musik- und Effektlautstärke und setzt bei den Punkten Rot-Grün und Zoonmfunktion (Einstellungen / J2, J3) per Klick ein Häkchen an die entsprechende Stelle. Durch einen Klick auf den 'Zurück'-Button (Einstellungen / J1) speichert er abschließend die von ihm gemachten Einstellungen im System.
  	
  	\end{itemize}
  	
  	
\end{requirements}
  		
\subsection{Datenkonsistenzen}

\begin{requirements}
  		
  	\req{T210} Der Levelfortschritt wird für jedes Benutzerprofil individuell gespeichert und verwaltet. Gleiches gilt für die Informationen über den Lernfortschritt des jeweiligen Benutzers und für dessen Achievements.
  		
	\req{T220} Die vom Benutzer gemachten Einstellungen werden unter dessen Benutzerprofil gespeichert. Meldet sich der Benutzer wieder nach dem Schließen der Application an, werden die an sein Benutzerprofil gebundenen Einstellungen verwendet.
	
	\req{T230} Es gibt keine Möglichkeit, zwei Profile mit dem selben Namen anzulegen. Wird ein Profil gelöscht, steht der dazu gehörende Name bei der Erstellung eines neuen Profils wieder frei zur Verfügung.

\end{requirements}	
  			
  		
