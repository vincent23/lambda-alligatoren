\section{Globale Testfälle}

\subsection{Funktionssequenzen}

Die im Folgenden gemachten Verweise der Form "`(Hauptmenü / A5)"' beziehen sich auf die im Kapitel "`Benutzerschnittstelle"'  (10.5) beschriebenen Bedienelemente.

\begin{requirements}

	\req{T110} Ein Benutzerprofil erstellen
	
	
	\begin{itemize}
  			\item Der Anwender befindet sich im Startfenster und drückt den "`Benutzer"'-Button (Hauptmenü / A5). Er gelangt so in ein neues Fenster, in welchem ihm eine Liste aller zurzeit im System gespeicherten Benutzer präsentiert wird.
  			
  			\item Er scrollt an das Ende dieser Liste und wählt den Listeneintrag mit dem Plus-Symbol (Profilauswahl / L2). Dadurch öffnet sich ein Dialogfenster zur Eingabe eines Benutzernamens.
  			
  			\item Der Benutzer drückt auf das zur Eingabe des Namens vorgesehene Feld (Profilerstellung (Teil 1) / M1) und gibt über die angezeigte Bildschirmtastatur seinen gewünschten Benutzernamen ein. Anschließend gelangt er durch Drücken auf den "`Weiter"'-Button (Profilerstellung (Teil 1) / M2) in das nächste Dialogfenster, in welchem dem Anwender verschiedenen Profilbilder (Profilerstellung (Teil 2) / N1) präsentiert werden.
  			
  			\item Durch Drücken wählt er eines dieser Bilder und bestätigt mit dem Drücken des "`Weiter"'-Buttons (Profilerstellung (Teil 2) / N2) seine Eingaben. Im System ist nun ein neues Benutzerprofil mit dem von ihm gewählten Namen und Profilbild gespeichert.
  	\end{itemize}
  	
  	
  	
  	\req{T120} Sich mit einem Benutzerprofil anmelden
  	
  	\begin{itemize}
  			\item Der Spieler befindet sich im Startmenü, drückt auf den "`Benutzer"'-Button (Hauptmenü / A5) und wechselt dadurch in eine Listenansicht, in der ihm alle gespeicherten Profile angezeigt werden.
  			
  			\item Er scrollt durch die Liste und wählt einen der Einträge (Profilauswahl / L1) aus und ist so im System mit dem gewählten Benutzerprofil angemeldet.
  			
	\end{itemize}
  	
  	
	
	\req{T130} Ein Benutzerprofil editieren
	
	
	\begin{itemize}
  			\item Der Anwender wählt im Startfenster den "`Einstellungen"'-Button (Hauptmenü / A2) und kommt in das Einstellungsmenü.
  			
  			\item Hier drückt er den "`Profil Editieren"'-Button (Einstellungsmenü / J6) und wechselt dadurch in ein Fenster, in dem sein aktueller Benutzernamen und aktuelles Benutzerprofil sowie ein "`Löschen"'-Button angezeigt werden.
  			
  			\item Der Benutzer drückt auf den Benutzernamen, wodurch sich ein in T110 beschriebenes Dialogfenster zur Eingabe eines Benutzernamens öffnet.
  			
  			\item Er gibt jetzt in das dafür vorgesehene Feld seinen neuen Benutzernamen ein, bestätigt diesen mit dem "`Weiter"'-Button und gelangt so in das vorherige Fenster.
  			
  			\item Abschließend drückt er auf das Profilbild und kommt so entsprechend in ein Dialogfenster zur Wahl eines neuen Profilbilds. Durch Drücken auf das gewünschte Bild wird dieses als neues zum Profil gehörende Bild im System gespeichert.
  			
  			
  	\end{itemize}

  	
	
	
	\req{T140} Ein Benutzerprofil löschen
	
	
	\begin{itemize}
  			\item Der Benutzer hat sich mit seinem Profil angemeldet und befindet sich im "`Profil Editieren"'-Fenster.
			Er drückt auf den "`Löschen"'-Button, wodurch sich ein Dialog öffnet, in dem der Benutzer gefragt wird, ob er das derzeitige gewählte Profil wirklich löschen möchte.
  			
  			\item Er bestätigt dies mit Drücken auf den "`Weiter"'-Button und der Bildschirm wechselt wieder in das Startmenü. Das ausgewählte Profil und alle damit verbundenen Daten sind nun aus dem System gelöscht.
  			
	\end{itemize}
	

	\req{T150} Ein Level starten
	
	
	\begin{itemize}
  			\item Im Startmenü wählt der Spieler den "`Spielen"'-Button (Startmenü / A1) und öffnet so die Levelübersicht.
  			
  			\item Hier drückt er eines der Levelpakete (Levelübersicht / B2) und gelangt in die Leveldetailansicht, in der ihm alle im Paket zusammengefassten Level präsentiert werden.
  			
  			\item Durch Drücken auf ein bereits freigeschaltetes Level (Leveldetailübersicht / C2) startet er dieses und wechselt in den zum gewählten Level gehörenden Platziermodus.
  	\end{itemize}
  	
  	
  	
  	\req{T160} Den Platziermodus bedienen
  	
	
	\begin{itemize}
			\item Nach dem Start eines Levels findet sich der Spieler im Platziermodus wieder. Per Drag\&Drop zieht er einzelne Spielelemente (Level (Platziermodus) / D7) auf die Arbeitsfläche (Level (Platziermodus) / D5) und verändert so die Ausgangskonstellation.
  			
  			\item Er wählt nun einen besonders gekennzeichneten Alligator an und öffnet dadurch ein Kontextmenü, in dem er die Möglichkeit hat die Farbe des Alligators neu zu wählen.
  			
  			\item Durch Drücken auf eine der angezeigten Farben schließt sich das Kontextmenü wieder, und der Alligator hat nun die gewählte Farbe.
  			
  			\item Der Anwender benutzt nun den "`Zoom-In"'-Button (Level (Platziermodus) / D1) und vergrößert so die auf der Arbeitsfläche dargestellten Elemente, was er anschließend durch Drückend des "`Zoom-Out"'-Buttons  (Level (Platziermodus) / D2) wieder rückgängig macht.
  			
  			\item Er wählt jetzt das sich am oberen Bildschirmrand befindliche Menüelement aus und zieht es nach unten auf den Bildschirm. Im dadurch entstandenen Bildschirmelement ist jetzt die zu erreichende Endkonstellation des Levels angezeigt, welche durch Loslassen des Bildschirms anschließend wieder verschwindet. 
  			
  			\item Der Benutzer drückt nun den "`Spielmenü"'-Button (Level (Platziermodus) / D2) und wechselt in das Spielmenü.
  			
  			\item Hier wählt er den Eintrag "`Zurücksetzen"' (Spielmenü / G2), was den Neustart des Levels zur Folge hat. Aller bisher im Level gemachter Fortschritt wird dadurch verworfen.
  			
  			\item Er drückt anschließend den "`Tipp"'-Button (Level (Platziermodus) / D3) und bekommt so einen Tipp zum derzeitigen Level angezeigt. Durch Drücken des "`Next"'-Buttons wird dieser wieder ausblendet.
  			
  			\item Der Spieler beginnt von neuem, die auf der Arbeitsfläche präsentierte Ausgangskonstellation mithilfe der ihm zu Verfügung stehenden Spielelemente zu verändern. Abschließend wechselt er durch Drücken auf den "`Start"'-Button (Level (Platziermodus) / D6) in den Simulationsmodus.
  	\end{itemize}
  	
  	
  	
  	\req{T170} Den Simulationsmodus bedienen
  	
	
	\begin{itemize}
	
  			\item Der Benutzer befindet sich im Simulationsmodus und drückt den "`Start"'-Button (Level (Simulationsmodus) / B5). Er startet so die automatische Auswertung der von ihm im Platziermodus erstellten Konstellation, welche auf der Arbeitsfläche (Level (Simulationsmodus) / E5) angezeigt wird.
  			
  			\item Durch erneutes Drücken des Buttons stoppt er die Auswertung wieder.
  			
  			\item Anschließend benutzt er den "`Vorwärts"'-Button (Level (Simulationsmodus) / E7) um einen einzigen Schritt der Auswertung auszuführen, welchen er dann durch Drücken des "`Rückwärts"'-Buttons (Level (Simulationsmodus) / E3) wieder rückgängig macht.
  			
			\item Jetzt lässt er die Simulation wieder durch Drücken des "`Schneller"'-Buttons (Level (Simulationsmodus) / E9) in höherer Geschwindigkeit ablaufen.
  			
  			\item Wenn der Endzustand durch die Auswertung erreicht wird, stoppt diese, und der Anwender bekommt über ein Dialogfenster angezeigt, dass er das Level erfolgreich abgeschlossen hat. Mit Drücken des "`Level"'-Buttons (Levelerfolg / F2) wechselt er abschließend in in die Leveldetailansicht.
  			
  			
  	\end{itemize}
  			

  	
  	\req{T180} Informationen über Benutzer abrufen
  	
  	
  	\begin{itemize}
  	
		\item Der Anwender befindet sich im Startmenü, wählt den "`Statistiken"'-Button (Startmenü / A3) und wechselt in das "`Statistiken"'-Fenster
		
		\item Hier wählt er über eine Dropdown-Liste (Statistiken / K1) ein Benutzerprofil und über die Tabauswahl (Statistiken / K3) die Informationskategorie "`Spielzeit"'. Ihm werden nun im dafür vorgesehenen Feld (Statistiken / K4) alle im System gespeicherten Informationen über die Spielzeit des ausgewählten Benutzer präsentiert.
		
		\item Der Nutzer wählt jetzt über die Dropdown Liste ein anderes Benutzerprofil und bekommt so entsprechende Informationen angezeigt, welche sich dann durch die anschließende Wahl der Informationskategorie "`Spielfortschritt"' wieder verändern.
		
  	
  	\end{itemize}
  	
  	
  	
  	\req{T190} Einstellungen ändern und speichern
  	
  	
  	\begin{itemize}
  	
		\item Im Startmenü drückt der Anwender den "`Einstellungen"'-Button und wechselt so in das entsprechende Fenster.
		
		\item Im jetzt eingeblendeten Bildschirm verändert der Nutzer mithilfe von Scrollbars (Einstellungen / J4, J5) die Musik- und Effektlautstärke und setzt bei den Punkten Farbenblindheitsmodus und Zoomfunktion (Einstellungen / J2, J3) ein Häkchen an die entsprechende Stelle. Durch Drücken des "`Zurück"'-Buttons (Einstellungen / J1) speichert er abschließend die von ihm gemachten Einstellungen im System.
  	
  	\end{itemize}
  	
  	
\end{requirements}
  		
\subsection{Datenkonsistenzen}

\begin{requirements}
  		
  	\req{T210} Der Levelfortschritt wird für jedes Benutzerprofil individuell gespeichert und verwaltet. Gleiches gilt für die Informationen über den Lernfortschritt des jeweiligen Benutzers und für dessen Achievements.
  		
	\req{T220} Die vom Benutzer gemachten Einstellungen werden unter dessen Benutzerprofil gespeichert.
	Meldet sich der Benutzer wieder nach dem Schließen der Applikation an, werden die an sein Benutzerprofil gebundenen Einstellungen verwendet.
	
	\req{T230} Es gibt keine Möglichkeit, zwei Profile mit dem selben Namen anzulegen. Wird ein Profil gelöscht, steht der dazu gehörende Name bei der Erstellung eines neuen Profils wieder frei zur Verfügung.

\end{requirements}	
