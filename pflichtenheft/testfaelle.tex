\section{Globale Testfälle}

\subsection{Funktionssequenzen}

\begin{requirements}

  \req{T110} Ein Benutzerprofil erstellen, speichern, löschen, auswählen, zum Anmelden benutzen.
  	
  \req{T120} Das Tutorial starten, beenden, wieder öffnen, erfolgreich abschließen.
  	
  \req{T130} Ein Level auswählen, starten, spielen, schließen, wieder laden, erfolgreich beenden.
  	
  \req{T140} Den Platziermodus starten, die Endkonstellation angezeigen lassen, die Ausgangskonstellation mithilfe von Drag and Drop manipulieren, Tipps einblenden lassen, Elemente durch anklicken färben, zum Ursprungszustand zurückkehren, den Platziermodus beenden.
  
  \req{T150} Den Simulationsmodus starten, die automatische Simulation starten, stoppen, schrittweise die Simulation ablaufen lassen, Schritte zurück gehen, die Simulationsgeschwindigkeit einstellen, die Zoomfunktion nutzen, den Simulationsmodus beenden.
  	
  \req{T160} Den Sandboxmoudus öffnen, eigene Anfangs- und Endkonstellationen mithilfe von Drag and Drop erstellen, Elemente durch Anklicken färben, das erstellte Level speichern, ein bereits erstelltes Level in den Sandboxmodus importieren, den Sandboxmouds beenden.
  
  \req{T170} Den Elternbereich öffnen, einzelne Benutzer auswählen, ausgewählte Informationen über deren Lernfortschritt abrufen, den Elternbereich wieder verlassen.
  		
\end{requirements}
  		
\subsection{Datenkonsistenzen}

\begin{requirements}
  		
  \req{T210} Der Levelfortschritt wird für jedes Benutzerprofil individuell gespeichert und verwaltet. Gleiches gilt für die Informationen über den Lernfortschritt des jeweiligen Benutzers und für dessen Achievements.
  		
  \req{T220} Die vom Benutzer gemachten Einstellungen werden unter dessen Benutzerprofil gespeichert. Meldet sich der Benutzer wieder nach dem Schließen der Application an, werden die an sein Benutzerprofil gebundenen Einstellungen verwendet.

\end{requirements}	
