\section{Einleitung}

Programmieren ist eine immer wichtiger werdende Fähigkeit, die aber trotzdem nicht von sehr vielen Menschen beherrscht wird.
Oft wird die Wichtigkeit dessen durch den fortschreitenden Einzug von Computern jeder Form in den Alltag bereits mit Lesen, Schreiben und Rechnen verglichen oder sogar gleichgesetzt.
Daher gibt es verschiedene Ansätze, Kinder schon sehr früh an das Programmieren heranzuführen.
Eine häufig vertretene Forderung ist beispielsweise die Einführung eines zusätzliches Schulfaches, um schon Grundschüler mit den dahinterstehenden gedanklichen Konzepten vertraut zu machen.

Weiterhin ist die funktionale Programmierung ein an Beliebtheit gewinnendes Paradigma.
Dies ist einerseits durch den hohen dadurch ermöglichten Abstraktionsgrad, andererseits aber auch mit den Vorteilen bei der Parallelisierbarkeit begründet.
Die Grundlage der funktionalen Programmierung bildet das (untypisierte) Lambda-Kalkül (\(\lambda\)-Kalkül).
Es besteht trotz seiner hohen Mächtigkeit nur aus wenigen Regeln und ist dadurch in seinen Grundzügen relativ einfach zu verstehen und anzuwenden.

Eine Spielidee \footnote{\url{http://worrydream.com/AlligatorEggs/}} von Bret Victor kombiniert nun diese beiden Ansätze.
Dabei werden die Lambda-Terme durch verschiedenfarbige Alligatoren und deren Eier dargestellt und die Regeln darauf übertragen, um die Hürde der Formalisierung abzubauen.
Dadurch wirkt die Notation weniger abschreckend und kann Kindern leichter vermittelt werden.
Auf der Grundlage dieser Idee soll nun eine Lern-App für Android entwickelt werden, die Victor's Spielidee umsetzt.
Kinder im Grundschulalter sollen dann in diesem abgewandelten \(\lambda\)-Kalkül gestellte Aufgaben selbständig lösen, und ihr Spielfortschritt durch Eltern oder Lehrer einsehbar sein.
