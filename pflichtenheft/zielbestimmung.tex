\section{Zielbestimmung}
Das Produkt soll es Benutzern ermöglichen, spielerisch einen Einstieg in die Konzepte des Lambda-Kalküls, und damit die Grundlagen funktionaler Programmierung, zu finden.
Dabei werden, nach einer Spielidee \footnote{\url{http://worrydream.com/AlligatorEggs/}} von Bret Victor, die Lambda-Terme durch verschiedenfarbige Alligatoren und deren Eier dargestellt.
Grundschüler sollen dann in diesem abgewandelten \(\lambda\)-Kalkül gestellte Aufgaben selbständig lösen, und ihr Spielfortschritt durch Eltern oder Lehrer einsehbar sein.


\subsection{Musskriterien}

\begin{itemize}
	\item Erstellung und Auswertung von Lambda-Termen
	\item Kontrolle des Lernfortschritts durch Eltern oder Lehrer
	\item Interaktive Einführung und Erklärung der Regeln
	\item Bedienung über ein Tablet mit Toucheingabe
	\item Langzeitmotivation des wird wird aufrechterhalten
\end{itemize}


\subsection{Wunschkriterien}

\begin{itemize}
	\item Speicherung mehrerer Spielstände für verschiedene Benutzer
	\item Verschiedene Gruppen von Leveln, um unterschiedliche Aspekte des Lambda-Kalküls zu erlernen
	\item Unterstützung mehrerer Sprachen
	\item Hilfestellung für Farbenblinde
	\item Editor zur Erstellung eigener Level
	\item Konfigurierbare Spieleravatare
	\item Für Smartphones angepasste Version
\end{itemize}


\subsection{Abgrenzungskriterien}

\begin{itemize}
	\item Keine explizite Verknüpfung mit Programmierung oder Lambda-Kalkül
	\item Funktionales Programmiermodell im Gegensatz zu häufig eher imperativen Ansätzen
\end{itemize}
