\section{Zielbestimmung}
Das Produkt soll es Benutzern ermöglichen, einen spielerischen Einstieg in die Konzepte des Lambda-Kalküls, und damit die Grundlagen funktionaler Programmierung, zu finden.

\subsection{Musskriterien}

\begin{itemize}
	\item Erstellung und Auswertung von Alligatorenkonstellationen (\(\lambda\)-Terme)
	\item Kontrolle des Lernfortschritts durch Eltern oder Lehrer
	\item Interaktive Einführung und Erklärung der Regeln
	\item Bedienung über ein Tablet mit Toucheingabe
	\item Langzeitmotivation des Spielers wird aufrechterhalten
\end{itemize}


\subsection{Wunschkriterien}

\begin{itemize}
	\item Speicherung mehrerer Spielstände für verschiedene Benutzer
	\item Verschiedene Gruppen von Leveln, um unterschiedliche Aspekte des Lambda-Kalküls zu erlernen
	\item Unterstützung mehrerer Sprachen
	\item Hilfestellung für Farbenblinde
	\item Editor zur Erstellung eigener Level
	\item Konfigurierbare Spieleravatare
	\item Für Smartphones angepasste Version
	\item Vermittlung einer Geschichte durch Animationen
\end{itemize}


\subsection{Abgrenzungskriterien}

\begin{itemize}
	\item Keine explizite Verknüpfung mit Programmierung oder Lambda-Kalkül
	\item Funktionales Programmiermodell im Gegensatz zu häufig eher imperativen Ansätzen
	\item Spieler über 12 Jahren sind nicht Teil der Zielgruppe
	\item Das Spiel wird über keine Online-Funktion haben
	\item Es wird keinen Coop Modus geben
\end{itemize}
