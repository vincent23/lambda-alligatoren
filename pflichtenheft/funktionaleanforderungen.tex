\section{Funktionale Anforderungen}

Die Anforderungen, weleche Wunsch- statt Musskriterien sind, sind mit einem "`+"' in der Identifikationsnummer markiert.

\subsection{Profile}

\begin{requirements}
	\req[Spielerprofile]{F110+}
	Man hat die Möglichkeit, insgesamt bis zu 5 Spielerprofile zu erstellen und zwischen ihnen zu wechseln.
	\req[Spielerprofile hinzufügen]{F111+} Spielerprofile werden einzeln hinzugefügt. Dabei werden zunächst der Name und dann der Avatar ausgewählt.
	\req[Spielerprofile ändern] {F112+} Man kann Namen und Avatare der einzelnen Spielerprofile ändern.
	\req[Spielerprofile löschen] {F113+} Spielerprofile lassen sich einzeln und mit sämtlichen zugehörigen Daten löschen.
	\req[Individueller Spielfortschritt für jedes Profil]{F120+}
	Die Spielerprofile sind vollkommen separat, was ihren Spielfortschritt angeht.
	\req[Avatarerstellung]{F130+}
	Es besteht die Möglichkeit, eigene Avatare zu erstellen und diese zur Identifikation der verschiedenen Spielerprofile zu nutzen.
\end{requirements}

\subsection{Spieldesign}

\begin{requirements}
	\req[Tutoriallevel]{F210+}
	Tutoriallevel helfen dem lernenden Kind, spielerisch neue Herausforderungen zu meistern. Bei jeder Erweiterung des Spielkonzeptes gibt es ein Tutoriallevel, dass dem Spieler diese Erweiterung vermittelt.
	\req[Platziermodus]{F220} In diesem Modus werden die Ausdrücke zusammengebaut.
		\begin{requirements}
			\req[Elemente hinzufügen]{F221}
			Eier und Alligatoren werden per Drag and Drop aus der Objektleiste auf dem Spielfeld abgelegt.
			\req[Elemente verschieben]{F222}
			Ebenso werden Eier und Alligatoren durch Drag and Drop auf dem Spielfeld verschoben.
			\req[Elemente löschen]{F223}
			Eier und Alligatoren können durch Ziehen in die Objektleiste vom Spielfeld entfernt werden.
			\req[Elemente umfärben]{F224}
			Die Farbe der Elemente wird durch ein Kontextmenü eingestellt, das sich durch einfache Berührung des Elements öffnet.
		\end{requirements}
	\req[Simulationsmodus]{F230}
	Es gibt verschiedene Simulationsarten, mit denen Ausdrücke durchlaufen werden.
		\begin{requirements}
			\req[Start der automatischen Auswertung]{F231}
			Der Ausdruck kann automatisch schrittweise immer weiter ausgewertet werden.
			\req[Anhalten der automatischen Auswertung]{F232}
			Die automatische Auswertung des Ausdruckes kann pausiert werden.
			\req[Evaluationsschritt ausführen]{F233}
			Es kann ein einzelner Auswertungsschritt ausgeführt werden.
			\req[Auswertung zurücksetzen]{F234}
			Die Auswertung kann neu gestartet werden.
			\req[Schritt zurück]{F235+}
			Mindestens 30 Schritte können rückwärts abgerufen werden.
			\req[Geschwindigkeit]{F236+} Die automatische Auswertung kann in ihrer Geschwindigkeit zwischen ``Schnell'' und ``Langsam'' geregelt werden.
		\end{requirements}
	\req[Modusübergang] {F240+}Sollte ein nichtauswertbarer Ausdruck im Platziermodus gebaut worden sein, z.B. nicht eingefärbte Elemente, ist es nicht möglich in den Simulationsmodus zu wechseln.
	\req [Farben der Elemente] Der Spieler hat die Möglichkeit Alligatoren in bis zu 10 Farben einzufärben. Sollten durch den Ausdruck mehr benötigt werden unterstützt das Spiel mehr Farben, die dem Spieler jedoch nicht direkt zugänglich sind.
	\req[Achievementsystem]{F250+}
	Als Feedback und Motivationselement für den Spieler wird ein Achievementsystem integriert.
	Mindestens folgende Achievements sollen möglich sein:
		\begin{requirements}
			\req{F251+} Feste Anzahl Level in einem Levelpaket gelöst.
			\req{F252+} Feste Anzahl gefressener Alligatoren erreicht.
			\req{F253+} Feste Anzahl geschlüpfter oder in das Spielfeld eingefügter Alligatoren erreicht.
			\req{F254+} Feste Gesamtspielzeit erreicht.
			\req{F255+} In einer Spielsitzung eine feste Anzahl an Leveln gelöst.
			\req{F256+} Feste Anzahl an Zeilen oder Spalten verwendet.
			\req{F257+} Feste Anzahl gefressener Alligatoren in einem Level erreicht.
			\req{F258+} Lösen eines Levels ohne Hinweise.
			\req{F259+} Erstellen eines nichtterminierenden Ausdrucks.
		\end{requirements}
	\req[Import und Export von Sandbox-Leveln]{F260+}
	Im Leveleditor ist es möglich, (u.a. fremde) Spielstände vom Gerätespeicher zu laden und eigene Spielstände dorthin zu exportieren.
	\req[Hinweise] {F270+} Der Spieler hat die Möglichkeit, einen Hinweis einzublenden, sollte er das Level nicht selbstständig lösen können.
	\req[Spielgrenzen]{F280+} Das Überschreiten gewisser Grenzen in der App wird verhindert und der Nutzer wird darüber informiert. Beispiele für diese Grenzen sind:
	\begin{itemize}
		\item Mehr als 300 Elemente können nicht gleichzeitig auf dem Spielfeld sein.
		\item Mehr als 30 verschiedene Farben werden beim Umfärben benötigt.
	\end{itemize} 
\end{requirements}
	

\subsection{Kontrolle des Lernfortschritts}

\begin{requirements}
	\req[Elternbereich mit Statistiken]{F310}
	Eltern oder Lehrkörper haben die Möglichkeit, die Fortschritte der verschiedenen Spielerprofile über verschiedene detaillierte Statistiken zu verfolgen.
	\begin{requirements}
		\req[Spielzeit]{F311} Man kann die insgesamt im Spiel verbrachte Zeit für jeden Spieler, ebenso wie die Zeit, die ein beliebiger Spieler in der letzten Woche im Spiel verbracht hat, einsehen.
		\req[Spielfortschritt]{F412} Man kann sehen wie viele Level und Levelpakte gemeistert wurden.
		Außerdem wird die durchschnittliche Zeit pro Level angegeben.
		\req[Nutzung der Hinweise]{F313+} Man kann sehen, wie oft die einzelnen Spieler auf das eingebaute Hinweissystem zurückgreifen mussten, um die Lösung zu erreichen.
		Außerdem wird die durchschnittliche Hinweisnutzung pro Level angegeben.
		\req[Level zurücksetzten] {F314+} Man kann sehen, wie oft die einzelnen Spieler gesamte Level zurückgesetzt haben.
		Außerdem wird die durchschnittliche Anzahl der Zurücksetzungen pro Level angegeben.
		\req[Achievements]{F315+} Man kann einsehen, welche Achievements die einzelnen Spieler errungen haben.
	\end{requirements}
\end{requirements}

\subsection {Spielbedienung}
\begin {requirements}
	\req[Touch]{F410} Die Steuerung des Spiels erfolgt über den Touchscreen des genutzten Gerätes.
	\req[Singe Touch]{F420} Es ist möglich, den gesamten Funktionsumfang der App zu nutzen, ohne mehr als einen Finger dazu zu benötigen.
	\req[Zoom]{F430} Der Spieler kann die Vergrößerungs-/Verkleinerungsstufe des Bildausschnittes im Spiel ändern.
	Dies kann sowohl über eigens dafür vorgesehene Buttons, als auch über eine Pinch-Geste erfolgen.
	\req[Drag and Drop]{F440}
	Der Spieler kann den Bildausschnitt des Spielfeldes durch Verschieben mit einer Touch-Geste ändern und so der Situation anpassen.
	\req[Mehrsprachigkeit]{F450+}
	Kompabilität mit dem Deutschen und mindesten einer weiteren Fremdsprache.
	\req[Kompabilität für Farbenblinde]{F460+}
	Das fertige Endprodukt kann ohne Abstriche auch von Menschen mit Rot-Grün Schwäche genutzt werden. Dazu werden die Alligatoren nicht durch Farbe, sondern durch Muster unterschieden.
	\req[Lautstärke] {F470+} Der Spieler kann die Lautstärke von Musik und Soundeffekten separat einstellen.
\end {requirements}

\subsection{Ladeverhalten}
\begin {requirements}
	\req[Splashscreen]{F400} Beim Starten der App gibt es einen Eröffnungsbildschirm, der vor dem Hauptmenü geöffnet wird.
	\req[Ladebildschirm]{F410} Bei Ladezeiten länger als drei Sekunden wird dem Spieler ein Ladebildschirm gezeigt.
	\req [Ladebalken]{F420+} Auf dem Ladebildschirm zeigt ein Ladebalken dem Spieler den Ladefortschritt an.
	\req[Alligator-Trivia]{F430+} Beim Laden des Level können dem Spieler lustige und interessante Fakten über Krokodile und Alligatoren angezeigt werden.
	\req[Animationen]{F435+} Beim Laden des Levels können dem Spieler Animationen gezeigt werden, z.B. um eine fortlaufende Geschichte zu erzählen.
	\begin{itemize}
		\item[+] die Animation kann mittels Knopfdruck übersprungen werden.
	\end{itemize}
	\req[Spieldaten]{F440+} Das Spiel behält sich Implizit alle relevanten Interaktionen des Spielers.
	\begin{itemize}
		\item Relevante Daten werden beim Verlassen von Zuständen automatisch gesichert.
		\item Falls dies nicht möglich ist, wird dem Nutzer eine Warnung dargestellt.
		\item Beim Wiederbetreten eines zuvor verlassenen Zustands werden die gespeicherten relevanten Daten wiederhergestellt.
	\end{itemize}
\end {requirements}
