\section{Funktionale Anforderungen}

\subsection{Usability}

\begin{requirements}
	\req[Spielerprofile]{F10}
	Man hat die Möglichkeit mehrere Spielerprofile zu erstellen und zwischen ihnen zu wechseln.
	\req[Individueller Spielfortschritt für jedes Profil]{F20}
	Die Spielerprofile sind vollkommen separat was ihren Spielfortschritt angeht. 
\end{requirements}

\subsection{Spieldesign}

\begin{requirements}
	\req[Spieleinführung]{F100}
	Die Lernkurve soll anfangst sehr vorsichtig ansteigen. Tutorial-Level helfen dem lernenden Kind spielerisch an neue Herausforderungen heran.
	
	\req[Platziermodus]{F120}
	Im Plaziermodus kann der Spieler Alligatoren und Eier aus einer Seitenleiste per Drag and Drop auf dem Spielfeld positionieren. Die Farben dieser Elemente lassen sich durch das tabben des individuellen Elementes einstellen.
	\req[Simulationsmodus]{F120}
	Es gibt verschiedene Simulationsarten, mit denen Ausdrücke durchlaufen werden. Im automatischen Modus kann der Spieler die Geschwindigkeit einstellen mit der die Ausdrücke abgearbeitet werden. Im manuellen Modus kann der Spieler schrittweise durch den Ausdruck gehen. Bis zu 30 Schritte der Simulation kann der Spieler rückwärts laufen.
	\req[Zoom]{F130}
	Der Spielende wird die Möglichkeit haben den  gezeigten Bildausschnitt des Spielfeldes seiner persönlichen Präferenz bzw. der Situation anzupassen.	

\end{requirements}


\subsection{Kontrolle des Lernfortschritts}

\begin{requirements}
	\req[Elternbereich mit Statistiken]{F10}
	Eltern oder Lehrkörper haben die Möglichkeit die Fortschritte der verschiedenen Spielerprofile über verschiedene detaillierte Statistiken, beispielsweise ``verbrachte Zeit'' oder ``gelöste Level'' und mehr,  zu überwachen.
	\req[Kindgerechtes Achivement System]{F10}
	Als Feedback und Motivationelement für den Spielenden wird ein Achivement System integriert.
\end{requirements}

\subsection{Wunschkriterien}

\begin{requirements}
	\req[Kompabilität für Farbenblinde]{F10}
	Das fertige Endprodukt kann ohne Abstriche auch von Menschen mit Rot-Grün Schwäche genutzt werden. Dazu werden die Alligatoren nicht durch Farbe sondern durch Muster unterschieden.
	\req[Mehrsprachigkeit]{F10}
	Kompabilität mit dem Deutschen und mindesten einer weiteren Fremdsprache
	\req[Avatar Erstellung]{F10}
	Es besteht die Möglichkeit eigene Avatare zu erstellen und diese als Indentifikationsobjekt der verschiedenen Spielerprofile zu nutzen.
	\req[Import und Export von Sandbox-Leveln]{F10}
	Im Leveleditor ist es möglich fremde Kreationen zu laden und eigene Kreationen zu exportieren.
\end{requirements}
