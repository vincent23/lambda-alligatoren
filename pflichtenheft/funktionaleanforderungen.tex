\section{Funktionale Anforderungen}

Die Anforderungen die Wunschkriterien statt Musskriterien sind sind mit einem + in der Indentifikationsnummer markiert.

\subsection{Profile}

\begin{requirements}
	\req[Spielerprofile]{F110}
	Man hat die Möglichkeit mehrere Spielerprofile zu erstellen und zwischen ihnen zu wechseln.
	\req[Individueller Spielfortschritt für jedes Profil]{F120}
	Die Spielerprofile sind vollkommen separat was ihren Spielfortschritt angeht. 
	\req[Avatar Erstellung]{F130+}
	Es besteht die Möglichkeit eigene Avatare zu erstellen und diese als Indentifikationsobjekt der verschiedenen Spielerprofile zu nutzen.
\end{requirements}

\subsection{Spieldesign}

\begin{requirements}
	\req[Tutoriallevel]{F210}
	Tutorial-Level helfen dem lernenden Kind spielerisch neue Herausforderungen zu meistern.	
	\req[Platziermodus]{F220} In diesem Modus werden die Ausdrücke zusamengebaut
		\begin{requirements}
			\req[Drag and Drop]{F221}
			Eier und Alligatoren werden per Drag and Drop auf dem Spielfeld positioniert.
			\req[Farben festlegen]{F222}
			Die Farben der Elemente wird durch ein Kontextmenü eingestellt, das sich durch einfache Berührung des Elemente öffnet.
		\end{requirements}
	\req[Simulationsmodus]{F230}
	Es gibt verschiedene Simulationsarten, mit denen Ausdrücke durchlaufen werden.
		\begin{requirements}
			\req[Automatischer Modus]{F231}
			Einstellbare Geschwindigkeit der Abarbeitung der Ausdrücke.
			\req[Manueller Modus]{F232}
			Schrittweise Abarbeitung des Ausdrucks.
			\req[Schritt zurück]{F233}	
			Mindestens 30 Schritte können rückwärts abgerufen werden.
		\end{requirements}
	\req[Zoom]{F240+}
	Der Spielende wird die Möglichkeit haben den  gezeigten Bildausschnitt des Spielfeldes seiner persönlichen Präferenz bzw. der Situation anzupassen.	
	\req[Kindgerechtes Achievementsystem]{F250+}
	Als Feedback und Motivationelement für den Spielenden wird ein Achivementsystem integriert.
	Mindestens folgende Achievements sollen möglich sein:
		\begin{requirements}
			\req{F251+} Feste Anzahl Level in einem Levelpaket gelöst.
			\req{F252+} Feste Anzahl gestorbener Alligatoren erreicht.
			\req{F253+} Feste Anzahl erzeugter Alligatoren erreicht.
			\req{F254+} Feste Gesamtspielzeit erreicht.
			\req{F255+} Seit Spielstart eine feste Anzahl an Leveln gelöst.
			\req{F256+} Feste Anzahl an Zeilen oder Spalten verwendet.
			\req{F257+} Feste Anzahl gefressener Alligatoren in einem Level erreicht.
			\req{F258+} Lösen eines Levels ohne Hinweise.
			\req{F259+} Erstellen eines nichtterminierenden Ausdrucks.
		\end{requirements}
	\req[Kompabilität für Farbenblinde]{F260+}
	Das fertige Endprodukt kann ohne Abstriche auch von Menschen mit Rot-Grün Schwäche genutzt werden. Dazu werden die Alligatoren nicht durch Farbe sondern durch Muster unterschieden.
	\req[Mehrsprachigkeit]{F270+}
	Kompabilität mit dem Deutschen und mindesten einer weiteren Fremdsprache
	\req[Import und Export von Sandbox-Leveln]{F280+}
	Im Leveleditor ist es möglich fremde Kreationen zu laden und eigene Kreationen zu exportieren.
	\req[Hinweis] {F290+} Der Spielende hat die Möglichkeit, sollte er alleine nicht mehr weiter wissen, sich einen Hinweis einblenden.


\end{requirements}


\subsection{Kontrolle des Lernfortschritts}

\begin{requirements}
	\req[Elternbereich mit Statistiken]{F310}
	Eltern oder Lehrkörper haben die Möglichkeit die Fortschritte der verschiedenen Spielerprofile über verschiedene detaillierte Statistiken zu verfolgen..
	\begin{requirements}
		\req[Spielzeit]{311} Man kann die insgesamt im Spiel verbrachte Zeit für jeden Account ebenso sehen wie die Zeit die die Accounts in der letzten Woche im Spiel verbracht haben. 
		\req[Spielfortschritt]{312} Man kann sehen wie viele Level, Levelpakte gemeistert wurden. Außerdem wir die Durchschnittliche Zeit pro Level angegeben.
		\req[Nutzung der Hinweise]{313+} Man kann sehen wie oft die einzelnen Accounts auf das eingebaute Hinweissystem zurückgreifen mussten um die Lösung zu erreichen. Außerdem wird die durchschnittliche Hinweisnutzung pro Level angegeben.
		\req[Level zurücksetzten] {314+} Man kann sehen wie oft die einzelnen Accounts gesamte Level zurückgesetzt haben. Außerdem wird die durchschnittliche Resetrate pro Level angegeben.
		\req[Achievements]{315+} Man kann einsehen welche Achievements die einzelnen Accounts errungen hat.
	\end{requirements}
\end{requirements}


