\section{Funktionale Anforderungen}

\subsection{Profile}

\begin{requirements}
	\req[Spielerprofile]{F110}
	Man hat die Möglichkeit mehrere Spielerprofile zu erstellen und zwischen ihnen zu wechseln.
	\req[Individueller Spielfortschritt für jedes Profil]{F120}
	Die Spielerprofile sind vollkommen separat was ihren Spielfortschritt angeht. 
\end{requirements}

\subsection{Spieldesign}

\begin{requirements}
	\req[Tutoriallevel]{F210}
	Tutorial-Level helfen dem lernenden Kind spielerisch neue Herausforderungen zu meistern.	
	\req[Platziermodus]{F220}
		\begin{requirements}
			\req[Drag and Drop]{F221}
			Eier und Alligatoren werden per Drag and Drop auf dem Spielfeld positioniert.
			\req[Farben festlegen]{F222}
			Die Farben der Elemente wird durch ein Kontextmenü eingestellt, das sich durch einfache Berührung des Elemente öffnet.
		\end{requirements}
	\req[Simulationsmodus]{F230}
	Es gibt verschiedene Simulationsarten, mit denen Ausdrücke durchlaufen werden.
		\begin{requirements}
			\req[Automatischer Modus]{F231}
			Einstellbare Geschwindigkeit der Abarbeitung der Ausdrücke.
			\req[Manueller Modus]{F232}
			Schrittweise Abarbeitung des Ausdrucks.
			\req[Schritt zurück]{F233}	
			Mindestens 30 Schritte können rückwärts abgerufen werden.
		\end{requirements}
	\req[Zoom]{F240}
	Der Spielende wird die Möglichkeit haben den  gezeigten Bildausschnitt des Spielfeldes seiner persönlichen Präferenz bzw. der Situation anzupassen.	

\end{requirements}


\subsection{Kontrolle des Lernfortschritts}

\begin{requirements}
	\req[Elternbereich mit Statistiken]{F310}
	Eltern oder Lehrkörper haben die Möglichkeit die Fortschritte der verschiedenen Spielerprofile über verschiedene detaillierte Statistiken, beispielsweise ``verbrachte Zeit'' oder ``gelöste Level'' und mehr,  zu überwachen.
	\req[Kindgerechtes Achievement System]{F320}
	Als Feedback und Motivationelement für den Spielenden wird ein Achivement System integriert.
\end{requirements}

\subsection{Wunschkriterien}

\begin{requirements}
	\req[Kompabilität für Farbenblinde]{F410}
	Das fertige Endprodukt kann ohne Abstriche auch von Menschen mit Rot-Grün Schwäche genutzt werden. Dazu werden die Alligatoren nicht durch Farbe sondern durch Muster unterschieden.
	\req[Mehrsprachigkeit]{F420}
	Kompabilität mit dem Deutschen und mindesten einer weiteren Fremdsprache
	\req[Avatar Erstellung]{F430}
	Es besteht die Möglichkeit eigene Avatare zu erstellen und diese als Indentifikationsobjekt der verschiedenen Spielerprofile zu nutzen.
	\req[Import und Export von Sandbox-Leveln]{F440}
	Im Leveleditor ist es möglich fremde Kreationen zu laden und eigene Kreationen zu exportieren.
\end{requirements}
