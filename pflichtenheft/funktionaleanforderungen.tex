\section{Funktionale Anforderungen}

\subsection{Usability}

\begin{description}
	\item[/F10/] einfache Navigation im Spiel \label{F10} \newline
	Selbst Grundschüler sollen in der Lage sein das Menü zu navigieren, zuspielende Level auszuwählen und die Einstellungen zu erreichen.
	\item[/F20/] gute Übersichtlichkeit der Statistiken \label{F10} \newline
	Auch  Eltern oder Lehrer, bzw. Lehrerinnen, die nicht besonders technisch versiert sind, sollen die Lernerfolgsstatistiken lesen und verstehen können.
\end{description}

\subsection{Spieldesign}

\begin{description}
	\item[/F100/] Spieleinführung \label{F100} \newline
	Die Lernkurve soll anfangst sehr vorsichtig ansteigen. Tutorial-Level helfen dem lernenden Kind spielerisch an neue Herausforderungen heran.
	\item[/F110/] Simulation der Ausdrücke manipulierbar\label{F110} \newline
	Der Spielende soll die Möglichkeit haben, sich das Abwickeln der Ausdrücke sowohl schrittweise vorwärts oder fortlaufend mit variierbarer Geschwindigkeit anzuschauen. Im schrittweisen Modus soll er auch die Möglichkeit haben Schritte zurück zu gehen. Zwischen den Modi muss während der Simulation gewechselt werden können.
	\item[/F120/] Steuerungselement "Drag and Drop"\label{F120} \newline
	Die einzelne Spielelemente werden auf dem Display durch Drag and Drop bewegt.
	\item[/F130/] Zoom\label{F130} \newline
	Der Spielende wird die Möglichkeit haben den Bildausschnitt der Aufgabe oder der persönlichen Präferenz anzupassen.	

\end{description}


\subsection{Kontrolle des Lernfortschritts}

\begin{description}
	\item[/F10/] Elternbereich mit Statistiken \label{F10} \newline
	Eltern oder Lehrkörper haben die Möglichkeit die Fortschritte des Kindes über verschiedene Statistiken, beispielsweise "verbrachte Zeit" oder "gelöste Level" und mehr,  zu überwachen.
	\item[/F10/] Achivement System\label{F10} \newline
	Als Feedback und Motivationelement für den Spielenden wird ein Achivement System integriert 
\end{description}

\subsection{Wunschkriterien}

\begin{description}
	\item[/F10/]Kompabilität für Farbenblinde\label{F10} \newline
	Das fertige Endprodukt kann ohne Abstriche auch von Menschen mit Rot-Grün Schwäche genutzt werden.
	\item[/F10/] Mehrsprachigkeit\label{F10} \newline
	Kompabilität mit dem Deutschen und mindesten einer weiteren Fremdsprache
\end{description}