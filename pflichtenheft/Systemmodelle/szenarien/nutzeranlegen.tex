\subsubsection{Nutzer anlegen}
\paragraph{Voraussetzungen}\mbox{}\newline
\begin{itemize}
\item Der Nutzer befindet sich entweder bei der Erstausführung oder 
in der in ``Nutzer verwalten'' beschriebenen Liste mit Nutzern und hat auf 
den Eintrag ``Nutzer anlegen'' gedrückt.
\end{itemize}

\paragraph{Konkrete Schritte}\mbox{}\newline
\begin{itemize}
\item Es öffnet sich ein Bildschirm, der den Nutzer nach seinem Namen fragt.
\item Durch Drücken der Texteingabezeile öffnet sich die Android Tastatur.
\item Der Nutzer gibt seinen Namen ein. 
\item Solange das Feld leer ist, ist die Schaltfläche, die zum nächsten Schritt führt, deaktiviert.
\item  Auch bei bereits vorhandenen Nutzerprofilnamen wird die Schaltfläche blockiert und eine Warnung angezeigt,
die den Nutzer auf die Belegung des Namens hinweist. 
\item Ist die Eingabe in Ordnung, kann zum Bildschirm zum Auswählen des Profilavatars gewechselt werden.
\item Hier befindet sich eine Schaltfläche, die ein automatisch gewähltes Avatarbild anzeigt.
\item Will der Nutzer seinen Avatar selbst einstellen, drückt er dazu auf die Schaltfläche mit dem Bild.
\item Anschließend findet er sich in einem dafür vorgesehenen Dialog wieder.
\item Dieser enthält ein Raster mit neun mitgelieferten Avataren.
\item Jeder der Avatare (auch bereits benutzte), kann durch einmaliges Drücken ausgewählt werden.
\item Nach der Auswahl schließt sich der Dialog.
\item Die Avatarwahl kann beliebig oft wiederholt werden.
\item Der Benutzer befindet sich wieder zurück im ``Avatar wählen''-Schritt der Nutzer-erstellen Sequenz.
\item Dort führen die Schaltfläche ``zurück'' in den Dialog zur Namenseingabe.
\item Die Schaltfläche ``weiter'' leitet auf einen Bildschirm, in dem alle getätigten Entscheidungen
nochmals aufgeführt werden. 
\item Diese kann er mittels einer Schaltfläche ``fertigstellen'' akzeptieren.
\item Alternativ lassen sie sich durch einen weiteren ``zurück''-Knopf 
nochmals bearbeiten.
\item Nach dem Beenden des Erstellungsprozesses ist das gerade erstellte 
Benutzerprofil automatisch als aktueller Benutzer eingestellt.
\item Der Nutzer befindet sich nach Abschluss auf dem Hauptbildschirm des Spiels.
\end{itemize}
