\subsubsection{Nutzer anlegen}
\paragraph{Voraussetzungen}
\begin{itemize}
\item Der Nutzer befindet sich entweder bei der Erstausführung, oder 
in der in "`Nutzer verwalten"' beschriebenen Liste mit Nutzern und hat auf 
den Eintrag "`Nutzer anlegen"' gedrückt.
\end{itemize}

\paragraph{Konkrete Schritte}
\begin{itemize}
\item Es öffnet sich ein Bildschirm, welcher den Nutzer nach seinem Namen fragt.
\item Durch Drücken der Texteingabezeile öffnet sich die Eingabemethode.
\item Der Nutzer gibt seinen Namen ein. 
\item Solange das Feld leer ist, ist der Button, der zum nächsten Schritt führt, deaktiviert.
\item  Auch bei bereits vorhandenen Nutzerprofilnamen wird der Button blockiert und eine Warnung angezeigt,
die den Nutzer auf die Belegung des Namens hinweist. 
\item Wird die Eingabe akzeptiert, kann zum Auswahldialog für Profilavatare gewechselt werden.
\item Hier befindet sich ein Button, der ein automatisch gewähltes Avatarbild anzeigt.
\item Will der Nutzer seinen Avatar selbst einstellen, drückt er dazu auf den Button mit dem Bild.
\item Anschließend findet er sich in einem dafür vorgesehenen Dialog wieder.
\item Dieser enthält ein Raster mit neun mitgelieferten Avataren.
\item Jeder der Avatare (auch bereits benutzte), kann durch einmaliges Drücken ausgewählt werden.
\item Nach der Auswahl schließt sich der Dialog.
\item Die Avatarwahl kann beliebig oft wiederholt werden.
\item Der Benutzer befindet sich wieder zurück im "`Avatar wählen"'-Schritt der "Nutzer erstellen" Sequenz.
\item Dort führen die Buttons "`Zurück"' in den Dialog zur Namenseingabe.
\item Der Button "`Weiter"' führt auf einen Bildschirm, in dem alle getätigten Entscheidungen
nochmals aufgeführt werden. 
\item Diese kann der Bediener mittels des Buttons "`Fertigstellen"' akzeptieren.
\item Alternativ lassen sich die gewählten Einstellungen durch einen weiteren "`Zurück"'-Knopf 
nochmals bearbeiten.
\item Nach dem Beenden des Erstellungsprozesses ist das erstellte 
Benutzerprofil automatisch als Benutzer aktiv.
\item Der Nutzer kehrt ins Hauptmenü des Spiels zurück.
\end{itemize}
