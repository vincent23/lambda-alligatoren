\subsubsection{Nutzer anlegen}
Voraussetzung: Der Nutzer befindet sich entweder bei der Erstausführung oder 
in der in "Nutzer verwalten" beschriebenen Liste mit Nutzern und hat auf 
den Eintrag "Nutzer anlegen" gedrückt.
\newline
\newline
Es öffnet sich zuerst ein Bildschirm, der den Nutzer nach seinem Namen fragt.
Für die Eingabe steht eine Texteingabezeile zur Verfügung, auf die der
Nutzer drückt, um die Android Tastatur zu öffnen, mit der die Eingabe des
Namens erfolgen kann. Solange das Feld leer ist, ist die Schaltfläche, 
die zum nächsten Schritt führt, deaktiviert. Auch bei bereits vorhandenen
Nutzerprofilnamen wird die Schaltfläche blockiert und eine Warnung angezeigt,
die den Nutzer auf die Belegung des Namens hinweist. Ist die Eingabe
in Ordnung, wird durch Drücken der erwähnten Schaltfläche der Bildschirm
zum Auswählen des Profilavatars geöffnet. Hier befindet sich eine
Schaltfläche, die ein automatisch gewähltes Avatarbild anzeigt,
als auch Schaltflächen für "zurück" und "weiter". Will der Nutzer seinen 
Avatar selbst einstellen drückt er dazu auf die Schaltfläche mit dem Bild
und findet sich anschließend in einem dafür vorgesehenen Dialog wieder.
Hier gibt es zunächst ein Raster mit neun mitgelieferten Avataren,
von denen jeder (auch bereits benutzte) durch einmaliges Drücken ausgewählt
werden kann, woraufhin der Dialog sich wieder schließt. Dort leitet ihn
die genannte Schaltfläche "zurück" in den Dialog zur Namenseingabe; "weiter"
leitet ihn auf einen Bildschirm, in dem alle getätigten Entscheidungen
nochmals aufgeführt werden. Diese kann er mittels einer Schaltfläche 
"fertigstellen" akzeptieren, oder durch einen weitern "zurück"-Knopf 
nochmals bearbeiten.
\newline
\newline
Nach dem Beenden des Erstellungsprozesses ist das gerade erstellte 
Benutzerprofil automatisch als aktueller Benutzer eingestellt und der Nutzer
befindet sich auf dem Hauptbildschirm des Spiels.
