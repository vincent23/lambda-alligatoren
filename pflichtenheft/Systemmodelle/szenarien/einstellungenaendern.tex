\subsubsection{Einstellungen ändern}
\begin{itemize}
\item Der Spieler befindet sich auf dem Hauptbildschirm des Spiels oder hat das In-Game Menü 
geöffnet.
\item Er möchte die Effektlautstärke des Spiels ändern.
\item Dafür betätigt er den Button "`Einstellungen"'.
\item Es öffnet sich die entsprechende Ansicht.
\item Hier befinden sich
	\begin{itemize}
	\item die Einstellungsmöglichkeiten für das aktuelle Benutzerprofil, 
	\item der Schalter für die Geschwindigkeit des Simulationsmodus
	\item ein Auswahlhäkchen für die Zoombuttons
	\item ein Auswahlhäkchen für den Farbenblindenmodus
	\item je ein Schieberegler für die Lautstärken von Hintergrundmusik und Effektlautstärke
	\end{itemize}
\item Der Nutzer bewegt den Schieberegler für die Effektlautstärke nach hinten.
\item Zur Kontrolle ertönt beim Loslassen ein Effekt in der
gerade eingestellten Lautstärke.
\item Diesen Vorgang wiederholt der Nutzer so lange, bis er mit dem Ergebnis zufrieden ist.
\item Die erwähnten Einstellungen sind relativ zur globalen Medienlautstärke.
\item Einstellungen werden automatisch übernommen und gespeichert.
\item Der Zurück-Button führt den Spieler dorthin zurück, woher er die Einstellungen aufgerufen hat.
\end{itemize}
