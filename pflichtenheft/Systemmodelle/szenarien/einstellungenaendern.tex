\subsubsection{Einstellungen ändern}
Der Nutzer möchte die Effektlautstärke des Spiels ändern. Dazu befindet
er sich auf dem Hauptbildschirm des Spiels oder hat das In-Game Menü 
geöffnet. Dort betätigt er die Schaltfläche "Einstellungen".
Es öffnet sich die entsprechende Ansicht, die Einstellungsmöglichkeiten
für das aktuelle Benutzerprofil, die Sprache, die Geschwindigkeit des Simulationsmodus, je
ein Auswahlhäkchen für die Zoomschaltflächen und den Farbenblindenmodus und je einen Schieberegler
für die Lautstärken von Hintergrundmusik und Effektlautstärke bietet.
Auf letzterem bewegt der Nutzer nun den Regler nach hinten. Beim Loslassen
ertönt der charakteristische Ton zur Animation "Krokodil frisst" in der
gerade eingestellten Lautstärke. Diesen Vorgang wiederholt der Nutzer
so lange, bis er mit dem Ergebnis zufriedengestellt ist. Wie in Android 
üblich, wirken sich die Einstellungen nicht auf die globale Medienlautstärke
aus, unter die auch die Spielakustik fällt. Daher sind die beiden erwähnten
Schieberegler immer relativ zu dieser. Da die Einstellungen
bei einer Änderung jeweils automatisch persistiert werden, braucht der
Nutzer keine weiteren Schritte zum Speichern durchzuführen und kann
direkt über die dafür vogesehene Schaltfläche zurück in den Bildschirm
wechseln, von dem aus er die Einstellungsansicht betreten hat.

