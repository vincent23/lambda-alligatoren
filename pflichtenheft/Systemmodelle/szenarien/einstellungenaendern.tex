\subsubsection{Einstellungen ändern}
\begin{itemize}
\item Der Spieler befindet sich auf dem Hauptbildschirm des Spiels oder hat das In-Game Menü 
geöffnet.
\item Er möchte die Effektlautstärke des Spiels ändern.
\item Dafür betätigt er die Schaltfläche "`Einstellungen"'.
\item Es öffnet sich die entsprechende Ansicht.
\item Hier befinden sich
	\begin{itemize}
	\item die Einstellungsmöglichkeiten für das aktuelle Benutzerprofil, 
	\item der Schalter für die Geschwindigkeit des Simulationsmodus
	\item ein Auswahlhäkchen für die Zoomschaltflächen
	\item ein Auswahlhäkchen für den Farbenblindenmodus
	\item je ein Schieberegler für die Lautstärken von Hintergrundmusik und Effektlautstärke
	\end{itemize}
\item Den Schieberegler für die Effektlautstärke bewegt der Nutzer nun nach hinten.
\item Beim Loslassen ertönt der charakteristische Ton zur Animation "`Alligator frisst"' in der
gerade eingestellten Lautstärke.
\item Diesen Vorgang wiederholt der Nutzer so lange, bis er mit dem Ergebnis zufriedengestellt ist.
\item Wie in Android üblich, wirken sich die Einstellungen nicht auf die globale Medienlautstärke
aus, unter die auch die Spielakustik fällt. Daher sind die beiden erwähnten Schieberegler immer relativ zu dieser. 
\item Da die Einstellungen bei einer Änderung jeweils automatisch persistiert werden, braucht der
Nutzer keine weiteren Schritte zum Speichern durchzuführen.
\item Er kann direkt über die dafür vorgesehene Schaltfläche zurück in den Bildschirm
wechseln, von dem aus er die Einstellungsansicht betreten hat.
\end{itemize}
