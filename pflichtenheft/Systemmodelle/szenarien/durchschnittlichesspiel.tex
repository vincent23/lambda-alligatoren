\subsection{Durchschnittliches Spiel}
Voraussetzungen: Der Nutzer besitzt bereits ein eingerichtetes Spielerprofil.
\newline
\newline
\paragraph{{[}Level wählen{]}}
Der Nutzer befindet sich zu Beginn auf dem Hauptbildschirm des Spiels.
Dort drückt er die Schaltfläche "Spielen", wodurch er sich anschließend 
in der Levelpaketauswahl zu befindet. Diese stellt
die Gruppierungen der einzelnen Level nach Schwierigkeit und 
Hintergrunderzählung nebeneinander dar. Durch seitliches Wischen kann hier
das gewünschte Paket in den Vordergrund gebracht und durch einfaches
Drücken auf jenes ausgewählt werden, sofern der Spielfortschritt des 
eingestellten Nutzerprofils dies vorsieht. Nach der Auswahl sieht der Nutzer
ein Raster mit nummerierten Schaltflächen, die mit den einzelnen Leveln
korrespondieren. Innerhalb eines Pakets sind immer alle Level sofort 
spielbar. Da in frühen Levels eines Pakets allerdings Techniken vermittelt
werden, die in späteren Leveln desselben vorausgesetzt werden, bietet sich 
ein chronologisches Vorgehen an. 
\newline
\newline
\paragraph{{[}Level Spielen{]}}
\#\#\# Drückt der Nutzer nun auf eine zu einem
Level gehörende Schaltfläche, so wird ihm je nach Level zuerst eine kurze
Animation gezeigt, die die Hintergrunderzählung vermittelt. Diese kann 
allerdings mit einer entsprechenden Schaltfläche übersprungen werden. \#\#\#
\newline
\newline
Anschließend startet der Platziermodus, vor den sich zunächst das Overlay
schiebt, das die für den Abschluss des Levels zu erreichenden Ziele
erklärt. Bei Leveln, in denen es um bestimmte Formationen geht, die auf
dem Spielfeld erscheinen müssen, ist dies ein Bild der Formation. Für
Level, die eine bestimmte Anzahl an durchlaufenen Transformationen
verlangen, ist dies ein mit Piktogrammen angreicherter Text.
\newline
\newline
Der Dialog muss mittels einer Schaltfläche geschlossen werden, um den
Platziermodus bedienen zu können. 
\newline
\newline
\paragraph{{[}Platziermodus{]}}
Der nun angezeigte Bildschirm enthält im Hintergrund zuerst die Darstellung
des derzeitigen Zustands des Spielfelds. Davor befindet sich eine Leiste 
mit jeweils einem Bild von Krokodil und Ei auf dem Bildschirm, von der aus 
Kopien von diesen durch Ziehen und Loslassen auf dem Spielfeld platziert
werden können. Das Ablegen an der richtigen Stelle wird durch eine zweite, 
halbtransparente Darstellung des Objekts simuliert und sichergestellt. 
Die erst einmal nur weißen Objekte können zunächst nur in bereits 
existierenden Spalten platziert werden. Durch Drücken auf ein Objekt erscheint 
ein Dialog, in dem eine diesem eine Farbe zugeordnet werden kann.
Dort befindet sich eine Palette mit bereits existierenden Farben, von denen 
grundsätzlich 12 voreingestellt sind. Durch Betätigen einer Schaltfläche
"neu" können allerdings in einem weiteren Dialog, ähnlich wie aus Adobe 
Photoshop bekannt, weitere definiert werden. Bei Krokodilen befindet sich
im ersten Farbwahlbildschirm außerdem eine Schaltfläche, die mit einem
Piktogramm das Teilen der Spalten unterhalb dessen andeutet. Ein Drücken
dieser führt dieses Verhalten aus.
\newline
Solange sich noch weiße Eier auf dem Spielfeld befinden, deren Verhalten
nicht definiert ist, kann nicht in den Simulationsmodus gewechselt werden.
Dieser ist ansonsten durch drücken einer Schaltfläche vom Platziermodus
aus erreichbar.
\newline
Außerdem enthält der Bildschirm des Platziermodus noch je eine Schaltfläche
für das Spielmenü, für das Anzeigen der zu erreichenden Ziele des Levels, 
für das Rückgängigmachen und Wiederholen einer Aktion, das Zurücksetzen
des Spielfelds und, falls eingestellt, Schaltflächen für hinein- und 
herauszoomen im Spielfeld.
\newline
\newline
\paragraph{{[}Simulationsmodus{]}}
Im Simulationsmodus sieht der Nutzer zunächst nur das Spielfeld, so wie
er es im Platziermodus belegt hat. Die Drag-and-Drop Leiste ist verschwunden
wie auch die Schaltflächen zum Rückgängigmachen und Wiederholen von 
Platzieroperationen und für das Zurücksetzen des Felds. An deren Stellen 
sind stattdessen Schaltflächen für das Rückgängigmachen und Wiederholen 
einzelner Simulationsschritte als auch ein kombinierter
"Start" und "Pause" Knopf für die automatische Simulation. Eine Schaltfläche,
die zurück in den Platziermodus führt, ersetzt zudem den Knopf zum Übergang 
in den Simulationsmodus.
\newline
Durch Anwenden der Transformationsregeln kann der Nutzer hier schrittweise
überprüfen, wie sich seine Anordnung verhält. Sollten keine Transformationen
mehr ausgeführt werden können, überprüft das Spiel, ob das Ziel erreicht
wurde und zeigt einen Dialog zur Information über Erfolg oder Niederlage.
Bei Leveln mit durch Zeit/Anzahl der Transformationen definierten Zielen
überprüft das Spiel schon während der Simulation regelmäßig auf das Erreichen
der Ziele.
\newline
\newline
\paragraph{{[}Spielmenü{]}}
Das Spielmenü beinhaltet zu jeder Zeit die Folgenden Punkte:
\begin{itemize}
	\item Weiterspielen
	\item \textasciitilde\textasciitilde Levelziele anzeigen \textasciitilde\textasciitilde
	\item Level beenden (zurück zur Levelübersicht des aktuellen Pakets)
	\item Zurück zu Hauptbildschirm
	\item \textasciitilde\textasciitilde Erfolge \textasciitilde\textasciitilde
	\item Einstellungen
\end{itemize}
Im Platziermodus bietet es zudem den Eintrag "Spielfeld zurücksetzen",
wodurch alle vom Nutzer platzierten Objekte wieder.
