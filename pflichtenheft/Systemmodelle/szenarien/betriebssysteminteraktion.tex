\subsection{Betriebssysteminteraktion}
Der Benutzer befindet sich in einem Spielmodus (Platzier- bzw. 
Simulationsmodus), als er sich gezwungen fühlt, das Spiel zu verlassen.
Dies kann sowohl ein direktes Verlassen der Applikation über den "Home"
Knopf von Android als Folge, als auch Ereignisse des Betriebssystems wie 
z.B. einen eingehenden Anruf als Grund haben. \newline
Die Applikation verliert dadurch ihre Grafikkontexte und wird nach einiger
Zeit ganz aus dem Speicher entfernt. Daher speichert die App sofort
ihren aktuellen Zustand, d.h. die Belegung des Spielfelds durch den Nutzer 
und ggf. den Fortschritt der Simulation. \newline
Bei einem direkten Wiedereintritt (ohne Verdrängung aus dem Hauptspeicher)
stellt die Applikation nur ihren Grafikkontext wieder her. Dies wird dem
Nutzer durch eine entsprechende Warteanzeige verdeutlicht. \newline
Im anderen Fall startet das Spiel komplett neu, bietet aber, beim
Starten des unterbrochenen Levels, an, den gespeicherten Spielstand
wiederherzustellen. D.h., es gibt intern zu jedem einzelnen Level die Möglichkeit,
einen zuletzt verwendeten Spielstand abzuspeichern, da man ja nicht
unbedingt das zuletzt unterbrochene Level weiterspielen muss.
\newline
\newline
\#\#\# Eine Alternative wäre, nur einen Spielstand speichern zu müssen, der 
beim Eintritt des Nutzers in den Hauptbildschirm direkt angeboten wird,
und beim Ablehnen verworfen wird. Der Vorteil ist, neben dem geringeren
Speicheraufwand, dass man direkt weitermachen kann, wo man aufhörte.
Nachteile sind aber, dass man kein anderes Level beginnen kann, ohne den
Spielstand zu verwerfen. Außerdem muss ein Nutzerwechsel möglich sein, 
was wahrscheinlich etwas inhomogen wirkt. Besser wäre es also, das
obere Prinzip zu verfolgen, und trotzdem die Möglichkeit für einen
"Quick-Continue" zu bieten. \#\#\#
