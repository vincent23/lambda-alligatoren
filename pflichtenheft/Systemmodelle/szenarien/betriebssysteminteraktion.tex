\subsubsection{Betriebssysteminteraktion}
\begin{itemize}
\item Der Benutzer befindet sich in einem der Spielmodi (Platzier- bzw. 
Simulationsmodus)
\item Er wird genötigt, das Spiel direkt zu verlassen, d.h. ohne den Umweg über jegliche Menüs.
\item Dies kann sowohl durch Betätigung des "`Home"'-Buttons von Android als auch durch
Ereignisse des Betriebssystems, wie z.B. einen eingehenden Anruf, passieren.
\item Die Applikation verliert dadurch ihre Grafikkontexte.
\item Nach einiger Zeit wird sie zudem ganz aus der Liste der laufenden Programme entfernt.
\item Daher speichert die App sofort nach der Benachrichtigung über ihr Schließen ihren aktuellen Zustand
\item Dazu gehören die Belegung des Spielfelds durch den Nutzer und ggf. den Fortschritt der Simulation.
\item Bei einem direkten Wiedereintritt (ohne Verdrängung aus dem Hauptspeicher) stellt die Applikation nur ihren Grafikkontext wieder her.
\item Die Dauer dieses Vorgangs wird dem Spieler durch einen Wartebildschirm signalisiert.
\item Im anderen Fall startet das Spiel komplett neu.
\item Es bietet aber beim Starten des unterbrochenen Levels an, den gespeicherten Spielstand wiederherzustellen. 
\item D.h., es gibt intern zu jedem einzelnen Level die Möglichkeit,
einen zuletzt verwendeten Spielstand abzuspeichern, und es ist auch möglich, andere Level als das
unterbrochene zu spielen, ohne dass dieser verloren geht.
\end{itemize}
