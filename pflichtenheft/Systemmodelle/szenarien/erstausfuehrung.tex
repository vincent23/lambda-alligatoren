\subsubsection{Erstausführung}
\paragraph{Erstes Öffnen der Applikation}
\begin{itemize}
\item Ein Benutzer startet zum ersten Mal die Applikation. 
\item Es wird ein informativer Ladebildschirm gezeigt.
\item Es geschieht ein automatischer Übergang in die Sequenz zum
Anlegen eines Nutzers. \newline (Siehe Szenario "`Nutzer anlegen"')
\item Der Nutzer wird auf den Hauptbildschirm des Spiels geleitet. 
\item Das soeben erstellte Nutzerprofil ist automatisch aktiv.
\item Der Spieler drückt den Button "`Spielen"', um ein erstes Spiel zu starten.
\item Alternativ gibt es aber bereits auch Zugriff auf alle anderen Funktionen,
die das Spiel im Hauptbildschirm bietet.
\end{itemize}

\paragraph{Erstes Spiel eines Nutzerprofils}
\begin{itemize}
\item Das Spiel überspringt die Levelpaket- und Levelauswahl.
\item Stattdessen wird direkt das erste Level des ersten Pakets (Tutoriallevel) gestartet.
\item Es wird gegebenenfalls eine Einleitungssequenz abgespielt.
\item Der Platziermodus wird gestartet.
\item Die Ziele des Levels werden in einem Overlay dargestellt (Tutoriallevel: Alligator aus der Objektleiste ziehen und auf dem Feld ablegen).
\item Die ersten wichtigen Bedienflächen werden schrittweise nacheinander hervorgehoben und deren Funktion
einzeln mit Hilfe von Piktogrammen erläutert.
\item Ziel des Levels ist es, Objekte auf dem Spielfeld abzulegen.
\item Hat der Nutzer alle geforderten Buttons einmal erfolgreich benutzt, ist dieses Tutoriallevel beendet.
\item Der "`Level beendet"'-Dialog wird angezeigt. 
\item Dr Spieler kann nun entweder direkt zum nächsten Level, in die Levelübersicht des aktuellen Pakets oder
ins Hauptmenü übergehen.
\end{itemize}
