\subsubsection{Erstausführung}
\paragraph{Erstes Öffnen der Applikation}\mbox{}\newline
\begin{itemize}
\item Ein Benutzer startet zum ersten Mal die Applikation. 
\item Es wird ein Bildschirm mit Namen des Spiels gezeigt, der gleichzeitig darauf
hinweist, dass das Spiel lädt und der Nutzer zu warten hat.
\item Es geschieht ein automatischer Übergang in die Sequenz zum
Anlegen eines Nutzers (Siehe Szenario ``Nutzer anlegen'').
\item Der Nutzer wird auf den Hauptbildschirm des Spiels geleitet. 
\item Das das soeben erstellte Nutzerprofil ist automatisch eingestellt.
\item Der Spielr drückt die Schaltfläche ``Spielen''.
\item Alternativ gibt es aber bereits auch Zugriff auf alle anderen Funktionen,
die das Spiel im Hauptbildschirm bietet.
\end{itemize}

\paragraph{Erstes Spiel eines Nutzerprofils}\mbox{}\newline
\begin{itemize}
\item Das Spiel überspringt die Levelpaket- und Levelauswahl.
\item Stattdessen wird direkt das erste Level des ersten Pakets (Tutoriallevel) gestartet.
\item + Es wird eine Einleitungssequenz abgespielt.
\item Die Ziele des Levels werden dargestellt (Krokodil aus der Objektleiste ziehen und auf dem Feld ablegen).
\item Der Platziermodus wird geöffnet.
\item Die einzelnen Bedienflächen werden erst schrittweise aktiviert und deren Funktion
einzeln mithilfe von Piktogrammen erläutert. 
\item Hat der Nutzer alle Schaltflächen ein mal erfolgreich benutzt ist das Tutoriumslevel beendet.
\item Es wird übergegangen in den "Level beendet" Dialog. 
\item Von hier aus kann er entweder direkt zum nächsten Level, in die Levelübersicht des aktuellen Pakets oder
ins Hauptmenü übergehen.
\item Dem Nutzer stehen nun alle Funktionen frei zu Verfügung, die das Spiel 
bietet.
\end{itemize}
