\subsubsection{Erstausführung}
{[}Erstes Öffnen der Applikation{]}\newline
Ein Benutzer startet zum ersten Mal die Applikation. Ihm wird daraufhin
ein Bildschirm mit Namen des Spiels gezeigt, der gleichzeitig darauf
hinweist, dass das Spiel lädt und der Nutzer zu warten hat.
Anschließend geschieht ein automatischer Übergang in die Sequenz zum
Anlegen eines Nutzers
\newline
\newline
Weiter mit {[}Nutzer anlegen{]}
\newline
\newline
Ist diese durchgeführt, wird der Nutzer auf den Hauptbildschirm des
Spiels geleitet. Da das soeben erstellte Nutzerprofil automatisch eingestellt
ist muss der Spieler nur noch auf die Schaltfläche "Spielen" drücken.
Alternativ gibt es aber bereits auch Zugriff auf alle anderen Funktionen,
die das Spiel im Hauptbildschirm bietet.
\newline
\newline
\paragraph{{[}Erstes Spiel eines Nutzerprofils{]}}
Da es sich um die das erste Spiel mit dem gewählten Profil handelt, wird 
direkt davon ausgegangen, dass der Spieler das erste Level des ersten 
Pakets bzw. das Tutorium spielen möchte. Daher wird die Paket- und Levelauswahl 
übersprungen. Wie in der Rubrik [Durchschnittliches Spiel] bei [Level spielen] 
beschrieben, hat der Nutzer nun den Platziermodus vor sich. Im Tutoriumslevel 
werden die einzelnen Bedienflächen erst schrittweise aktiviert und deren Funktion
einzeln mithilfe von Piktogrammen erläutert. Hat der Nutzer alle Schaltflächen
ein mal erfolgreich benutzt ist das Tutoriumslevel beendet und befindet
sich im "Level beendet" Dialog. Von hier aus kann er entweder direkt zum
nächsten Level übergehen, in die Levelübersicht des aktuellen Pakets oder
ins Hauptmenü.
\newline
\newline
Dem Nutzer stehen nun alle Funktionen frei zu Verfügung, die das Spiel 
bietet.
