\subsubsection{Exemplarisches Spiel}
\paragraph{Voraussetzungen}
\begin{itemize}
\item Der Nutzer besitzt bereits ein eingerichtetes Spielerprofil.
\item Der Nutzer befindet sich zu Beginn auf dem Hauptbildschirm des Spiels.
\end{itemize}

\paragraph{Level wählen}
\begin{itemize}
\item Der Nutzer drückt den Button "`Spielen"'.
\item Er gelangt in die Levelpaketauswahl.
\item Diese stellt die Gruppierungen der einzelnen Level nach Schwierigkeit, thematischem Zusammenhang und 
Hintergrunderzählung nebeneinander dar. 
\item Durch seitliches Wischen kann hier das gewünschte Paket in den Vordergrund gebracht werden (Noch nicht freigeschaltete Pakete sind deaktiviert).
\item Durch einfaches Drücken wird ein Levelpaket ausgewählt bzw. betreten.
\item Einzelne Levelpakete können auch, aufgrund von zu geringem Spielfortschritt des 
eingestellten Nutzerprofils, gesperrt sein.
\item Nach der Auswahl sieht der Nutzer ein Raster mit nummerierten Buttons.
\item Jeder Button korrespondiert mit einem Level aus dem Levelpaket.
\item Innerhalb eines Pakets sind Level erst nach erfolgreichem Abschluss aller vorhergehender Level aus dem Paket spielbar.
\end{itemize}

\paragraph{Level Spielen}
\begin{itemize}
\item Der Nutzer befindet sich in der Levelauswahl.
\item Drückt der Nutzer nun auf einen zu einem
Level gehörenden Button, so wird ihm, je nach Level, zuerst eine kurze
Animation gezeigt, die die Hintergrunderzählung oder neue Spielinhalte vermittelt. 
\item Diese kann allerdings mit einem entsprechenden Button übersprungen werden.
\item Das Level startet im Platziermodus.
\item Vor diesen legt sich zunächst das Overlay, das die für den Abschluss des Levels zu erreichenden Ziele
zeigt.
\item Bei Leveln, in denen es um bestimmte Konstellationen geht, die auf
dem Spielfeld erscheinen müssen, ist dies ein Bild der Formation; für Level, die eine bestimmte Anzahl an durchlaufenen Transformationen
verlangen, ist dies ein mit Piktogrammen angereicherter Text.
\item Der Dialog muss am Menüelement nach oben weggeschoben werden. 
\end{itemize}


\paragraph{Platziermodus}
\begin{itemize}
\item Der Bildschirm des Platziermodus enthält im Hintergrund die Darstellung
des derzeitigen Zustands des Spielfelds.
\item Davor befindet sich eine Leiste mit jeweils einem Bild von Alligator, Ei und altem Alligator auf dem Bildschirm.
\item Von diesen können Kopien durch Drag\&Drop auf dem Spielfeld platziert werden. 
\item  Durch Drücken auf ein Objekt erscheint 
ein Dialog, in dem diesem eine Farbe zugeordnet werden kann.
\item Dort befindet sich eine Palette mit sechs vordefinierten Farben.
\item Durch drücken einer der Farben wird diese Farbe auf das Objekt angewandt und der Dialog geschlossen.
\item Solange sich noch ungefärbte Eier oder Alligatoren auf dem Spielfeld befinden (deren Verhalten
nicht definiert ist) kann nicht in den Simulationsmodus gewechselt werden.
\item Der Simulationsmodus ist durch Drücken eines Buttons vom Platziermodus
aus erreichbar.
\item Soll ein Objekt vom Spielfeld entfernt werden, muss dieses durch Drag\&Drop
in die Objektleiste zurückgelegt werden.
\item Es ist ein Button vorhanden, der das Spielmenü öffnet.
\item Desweiteren können die zu erreichenden Ziele eines Level durch Ziehen an dem zugehörigen Menüelement angezeigt werden.
\item Das Zurücksetzen des Spielfelds kann durch einen Button im Spielmenü ausgelöst werden.
\item Falls eingestellt, gibt es Buttons für das Hinein- und Herauszoomen im Spielfeld.
\item Ansonsten lässt sich die Zoomfunktion auch über die gängigen Pinchgesten mit zwei Fingern bedienen.
\end{itemize}

\paragraph{Simulationsmodus}
\begin{itemize}
\item Der Benutzer befindet sich im Simulationsmodus.
\item Es ist zunächst das Spielfeld zu sehen, so wie der Spieler es im Platziermodus belegt hat.
\item Die Drag\&Drop Leiste aus dem Platziermodus ist verschwunden.
\item Es gibt Buttons für das Rückgängigmachen und Ausführen 
einzelner Simulationsschritte bzw. Transformationen.
\item Ein kombinierter "`Start"' und "`Pause"' Knopf kontrolliert die automatische Simulation.
\item Es gibt einen Button, der zurück in den Platziermodus führt.
\item Ausgeführte Transformationen werden auf das Spielfeld angewendet.
\item Durch das Ausführen von Transformationen kann der Nutzer schrittweise
überprüfen, wie sich seine Anordnung verhält.
\item Standardmäßig zoomt das Spiel beim Starten der automatischen Simulation komplett heraus, sodass alle Objekte auf dem Spielfeld sichtbar sind.
\item Der manuelle Zoom bleibt während der automatischen Simulation verfügbar, sodass hier der Bildausschnitt nachjustiert werden kann.
\item Wenn keine Transformationen mehr ausgeführt werden können, überprüft das Spiel, ob das Ziel erreicht
wurde.
\item Anschließend zeigt es einen Dialog zur Information über Erfolg oder Niederlage.
\item Bei Leveln, mit durch Zeit/Anzahl der Transformationen definierten Zielen,
überprüft das Spiel schon während der Simulation regelmäßig auf das Erreichen
der Ziele.
\end{itemize}

%\paragraph{Spielmenü}\mbox{}\newline
%Das Spielmenü beinhaltet zu jeder Zeit die folgenden Punkte:
%\begin{itemize}
%	\item Weiterspielen
%	\item Level zurücksetzen
%	\item Einstellungen
%	\item Level beenden (zurück zur Levelübersicht des aktuellen Pakets)
%	\item Erfolge
%	\item Zurück zu Hauptbildschirm
%\end{itemize}
