\subsubsection{Exemplarisches Spiel}
\paragraph{Voraussetzungen}
\begin{itemize}
\item Der Nutzer besitzt ein bereits eingerichtetes Spielerprofil.
\item Der Nutzer befindet sich zu Beginn auf dem Hauptbildschirm des Spiels.
\end{itemize}

\paragraph{Level wählen}
\begin{itemize}
\item Der Nutzer drückt den Button "`Spielen"'.
\item Er gelangt in die Levelpaketauswahl.
\item Diese stellt die Gruppierungen der einzelnen Level nach Schwierigkeit, thematischem Zusammenhang und 
Hintergrunderzählung nebeneinander dar. 
\item Durch seitliches Wischen kann hier das gewünschte Paket in den Vordergrund gebracht werden (Noch nicht freigeschaltete Pakete sind deaktiviert).
\item Durch einfaches Drücken wird ein Levelpaket ausgewählt bzw. betreten.
\item Einzelne Levelpakete können erst mit ausreichenden Fortschritt angewählt werden.
\item Nach der Auswahl sieht der Nutzer ein Raster mit nummerierten Buttons.
\item Jeder Button korrespondiert mit einem Level aus dem Levelpaket.
\item Innerhalb eines Pakets sind Level erst nach erfolgreichem Abschluss aller vorherigen Level aus dem Paket spielbar.
\end{itemize}

\paragraph{Level Spielen}
\begin{itemize}
\item Der Nutzer befindet sich in der Levelauswahl.
\item Ein Klick des Buttons startet das jeweilige Level beziehungsweise dessen Einführungssequenz. 
\item Die Animation kann gegebenenfalls übersprungen werden.
\item Das Level startet im Platziermodus.
\item Im Vordergrund werden Levelziele eingeblendet.
\item Bei Leveln des Typs "'Färben und Einfügen"'  ist es ein Bild der Konstellation; für Level des Typs "'Schrittanzahl"' ist es ein mit Piktogrammen angereicherter Text.
\item Der Dialog muss am Menüelement nach oben weggeschoben werden. 
\end{itemize}


\paragraph{Platziermodus}
\begin{itemize}  
\item Der Bildschirm zeigt den derzeitigen Zustand des Spielfelds.
\item Davor befindet sich eine Leiste mit jeweils einem Piktogramm pro Spielelement.
\item Elemente können durch Drag\&Drop auf dem Spielfeld platziert werden. 
\item  Durch Drücken eines Objekt erscheint 
ein Dialog, in dem diesem eine Farbe zugeordnet werden kann.
\item Dort befindet sich eine Palette mit sechs vordefinierten Farben.
\item Durch Drücken einer der Farben erhält das Objekt die gewählte Farbe. Der Dialog wird geschlossen.
\item Solange sich noch ungefärbte Eier oder Alligatoren auf dem Spielfeld befinden (deren Verhalten
nicht definiert ist) kann nicht in den Simulationsmodus gewechselt werden.
\item Der Simulationsmodus ist durch Drücken eines Buttons vom Platziermodus
aus erreichbar.
\item Soll ein Objekt vom Spielfeld entfernt werden, muss dieses durch Drag\&Drop
in die Objektleiste zurückgelegt werden.
\item Es ist ein Button vorhanden, der das Spielmenü öffnet.
\item Die zu erreichenden Ziele eines Level können durch das Betätigen eines Menüelements angezeigt werden.
\item Das Spielfeld kann durch das Betätigen eines Buttons im Spielmenü zurückgesetzt werden.
\item Der Bildschirmausschnitt lässt sich wahlweise durch Gesten oder Schaltflächen anpassen.
\end{itemize}

\paragraph{Simulationsmodus}
\begin{itemize}
\item Der Benutzer befindet sich im Simulationsmodus.
\item Das Spielfeld ist im letzten gespeicherten Zustand.
\item Die Drag\&Drop Leiste aus dem Platziermodus ist verschwunden.
\item Schaltflächen zum Vorwärts- und Rückwärtsbewegen in der Simulation werden angezeigt.
\item Ein kombinierter "`Start"' und "`Pause"' Knopf kontrolliert die automatische Simulation.
\item Es gibt einen Button, der zurück in den Platziermodus führt.
\item Ausgeführte Transformationen werden auf das Spielfeld angewendet.
\item Durch das Ausführen von Transformationen kann der Nutzer schrittweise
überprüfen, wie sich seine Anordnung verhält.
\item Der Bildausschnitt wird automatisch so gewählt, dass die Konstellation komplett sichtbar ist.
\item Der manuelle Zoom bleibt während der automatischen Simulation verfügbar, sodass der Bildausschnitt nachjustiert werden kann.
\item Nach Ablauf aller möglichen Transformationen wird das Erreichen des Levelziels überprüft.
\item Anschließend wird ein Dialog zur Information über Erfolg oder Niederlage gezeigt.
\item Bei Leveln des Typs "Schrittanzahl" überprüft das Spiel schon während der Simulation regelmäßig auf das Erreichen
der Ziele.
\end{itemize}

%\paragraph{Spielmenü}\mbox{}\newline
%Das Spielmenü beinhaltet zu jeder Zeit die folgenden Punkte:
%\begin{itemize}
%	\item Weiterspielen
%	\item Level zurücksetzen
%	\item Einstellungen
%	\item Level beenden (zurück zur Levelübersicht des aktuellen Pakets)
%	\item Erfolge
%	\item Zurück zu Hauptbildschirm
%\end{itemize}
