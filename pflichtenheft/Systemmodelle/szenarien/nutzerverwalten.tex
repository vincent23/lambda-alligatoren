\subsubsection{Nutzer verwalten}
\paragraph{Voraussetzungen}
\begin{itemize}
\item Erstausführung bereits geschehen
\end{itemize}

\paragraph{Ohne eigenes Profil anmelden}
\begin{itemize}
\item Der Benutzer befindet sich auf dem Haupbildschirm des Spiels.
\item Die rechte Bildschirmseite zeigt einen Button, welcher Namen und Avatar des aktiven Spielers enthält.
\item Durch Drücken des Buttons gelangt der Benutzer auf den Bildschirm zur Nutzerverwaltung.
\item Hier sieht er zunächst eine Liste mit allen bereits 
eingerichteten Nutzern.
\item Da der Benutzer allerdings noch kein eigenes Profil
angelegt hat, wählt er den Listeneintrag "`Neues Profil erstellen"' aus.
\item Diese Aktion leitet ihn weiter auf den in Szenario "`Nutzer anlegen"'
beschriebenen Erstellungsprozess.
\item Im Anschluss befindet sich der Benutzer auf dem Hauptbildschirm des Spiels.
\end{itemize}

\paragraph{Nutzer ändern}
\begin{itemize}
\item Der Benutzer möchte Name und Avatar ändern.
\item Er drückt auf dem Hauptbildschirm auf den Button "`Einstellungen"'.
\item In dem sich nun öffnenden Bildschirm findet er einen Button "`Nutzer bearbeiten"'.
\item Diesen betätigt der Nutzer und findet sich auf dem dazugehörigen
Bildschirm wieder. 
\item Der Bildschirm zeigt nun Name und Avatar, sowie Buttons zum Ändern, an.
\item Das Löschen des Nutzers und das Zurücksetzten des Spielstandes ist möglich. 
\item Diese Buttons führen die gewünschte Option nach einer Sicherheitsabfrage aus.
\item Durch einen Button "`Zurück"' gelangt der Benutzer wieder zum Einstellungsbildschirm.
\end{itemize}
