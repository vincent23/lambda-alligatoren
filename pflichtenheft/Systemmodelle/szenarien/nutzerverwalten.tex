\subsubsection{Nutzer verwalten}
\paragraph{Voraussetzungen}
\begin{itemize}
\item Erstausführung bereits geschehen
\end{itemize}

\paragraph{Ohne eigenes Profil anmelden}
\begin{itemize}
\item Der Benutzer befindet sich auf dem Haupbildschirm des Spiels.
\item Hier begegnet ihm auf der rechten Bildschirmseite ein großer Button, 
die Namen und Avatarabbildung des zur Zeit eingestellten Nutzers darstellt.
\item Da diese aber nicht dem gerade aktiven Benutzer entsprechen, möchte er den eingestellten Nutzer ändern.
\item Dazu drückt der Nutzer auf den besagten Button.
\item Daraufhin gelangt er auf den Bildschirm zur Nutzerverwaltung.
\item Hier sieht er zunächst eine Liste mit allen bereits 
eingerichteten Nutzern.
\item Da der Benutzer allerdings noch kein eigenes Profil
angelegt hat, wählt er den Listeneintrag "`Neues Profil erstellen"' aus.
\item Diese Aktion leitet ihn weiter auf den in Szenario "`Nutzer anlegen"'
beschriebenen Erstellungsprozess.
\item Im Anschluss befindet sich der Benutzer auf dem Hauptbildschirm des Spiels wieder.
\end{itemize}

\paragraph{Nutzer ändern}
\begin{itemize}
\item Im Anschluss merkt der Benutzer aber, dass sowohl der Name falsch geschrieben
als auch der falsche Avatar ausgewählt wurde.
\item Zum Ändern muss er auf dem Hauptbildschirm auf den Button "`Einstellungen"' drücken.
\item In dem sich nun öffnenden Bildschirm findet er einen Button "`Nutzer bearbeiten"'.
\item Diesen betätigt der Nutzer durch Drücken und findet sich auf dem dazugehörigen
Bildschirm wieder. 
\item Der "`Nutzer bearbeiten"'-Bildschirm bezieht sich nur auf den aktuell eingestellten Benutzer.
\item Der Bildschirm zeigt zuerst Name und Avatar an.
\item Außerdem gibt es Buttons, die den Nutzer löschen und um seinen Spielfortschritt
zurückzusetzen. 
\item Durch Drücken auf Namen und Avatar erscheinen Änderungsdialoge,
ähnlich zu jenen aus der "`Nutzer anlegen"' Sequenz.
\item Die anderen Buttons stellen zunächst noch durch eine gesonderte Abfrage die Intention des 
Nutzers sicher und führen gegebenenfalls die gewünschte Operation aus.
\item Durch einen Button "`Zurück"' gelangt der Benutzer wieder zum Einstellungsbildschirm.
\end{itemize}
