\subsection{Nutzer verwalten}
Voraussetzung: Erstausführung bereits geschehen \newline
Referenzen: Spielerwechsel
\newline
\newline
Ein Benutzer hat die Applikation bereits gestartet und befindet sich somit
auf dem Haupbildschirm des Spiels. Hier begegnet ihm auf der rechten 
Bildschirmseite eine große Schaltfläche, die Namen und Avatarabbildung des 
zur Zeit eingestellten Nutzers darstellt. Da diese aber nicht dem gerade aktiven
Benutzer entsprechen, möchte er sie ändern. Dazu drückt der Nutzer die
besagte Schaltfläche, woraufhin er auf den Bildschirm zur Nutzerverwaltung 
geleitet wird. Hier sieht er zunächst eine Liste mit allen bereits 
eingerichteten Nutzern. Da der Benutzer allerdings noch kein eigenes Profil
angelegt hat, wählt er den Listeneintrag "neues Profil erstellen" aus.
Diese Aktion leitet ihn weiter auf den in Szenario "Nutzer anlegen" 
beschriebenen Erstellungsprozess.
\newline
\newline
Weiter mit {[}Nutzer anlegen{]}
\newline
\newline
Im Anschluss merkt der Benutzer aber, dass sowohl der Name falsch geschrieben
als auch der falsche Avatar ausgewählt wurde. Da er sich auf dem
Hauptbildschirm befindet und sein Spielerprofil derzeit eingestellt ist,
muss er nur auf die Schaltfläche "Einstellungen" drücken. In dem sich nun
öffnenden Bildschirm findet er zuoberst eine Schaltfläche "Nutzer bearbeiten".
Diese betätigt der Nutzer durch drücken und findet sich auf dem dazugehörigen
Bildschirm wieder. Es werden sowohl Name und Avatar angezeigt, als auch
Schaltflächen um den Nutzer zu löschen und um seinen Spielfortschritt
zurückzusetzen. Durch Drücken auf Namen und Avatar erscheinen Änderungsdialoge,
ähnlich zu jenen aus der "Nutzer anlegen" Sequenz. Die anderen Schaltflächen
stellen zunächst noch durch eine gesonderte Abfrage die Intention des 
Nutzers sicher und führen gegebenenfalls die gewünschte Operation aus.
