\section{Nichtfunktionale Anforderungen}

\subsection{Allgemeine Ziele \textit{(Auftraggebersicht)}}
\begin{requirements}
	\req{NF110} Das Spielprinzip vermittelt unterschwellig die Funktionsweise des (typenlosen) Lambda Kalküls
	\req{NF120} Das Spiel bietet Langzeitmotivation
	\req{NF130} Das Spiel ist für die Nutzung durch mehrere Personen ausgelegt (Geschwister, Betreuer etc.)
	\req{NF140} Die Navigation durch das Spiel ist intuitiv, sowohl für Kinder als auch Erwachsene
	\req{NF150} Das Spiel ist insgesamt kindgerecht entworfen:
		\begin{itemize}
			\item Piktogramme werden überall verwendet, wo möglich
			\item Auf Schrift, insbesondere auch "$\lambda$", wird so weit es geht vermieden
		\end{itemize}
\end{requirements}

\subsection{Benutzbarkeit, Performance und Stabilität \textit{(Nutzersicht)}}
\begin{requirements}
	\req{NF210} Die Bedienung des Spiels verläuft insgesamt frustrationsfrei (Präzisierung folgt):
	\req{NF220} Die Ladezeiten der Levels beträgt durchweg unter 15 Sekunden.
	\req{NF230} Die Simulation und deren Animationen verlaufen auf aktueller Hardware flüssig.
	\req{NF240} Es wird möglichst auf immer wiederkehrende Animationen, die den Spielfluss unterbrechen, verzichtet.
	\req{NF250} Das Spiel läuft stabil; es gilt:
		\begin{itemize}
			\item das Spiel verhält sich zu jeder Zeit vorhersehbar.
			\item alle (Folge-)Zustände und Übergänge sind zu jeder Zeit definiert.
			\item unerwartete Eingaben und Zustände werden abgefangen.
			\item es geschieht kein unerwartetes Beenden der App.
			\item es gibt definierte Grenzen in denen das Programm stabil läuft. Überschreiten dieser wird verhindert. Eine solche Grenze könnte lauten "Es sind höchstens 1000 Krokodile und Eier auf dem Spielfeld".
		\end{itemize}
	\req{NF260} Das Spiel behält sich implizit alle relevanten Interaktionen des Nutzers, also:
		\begin{itemize}
			\item aktuelle Zustände werden beim Verlassen wenn möglich automatisch gesichert.
			\item falls Sichern nicht möglich ist, wird der Nutzer gewarnt.
		\end{itemize}
\end{requirements}

\subsection{Qualität und Rechtliches \textit{(Entwicklersicht)}}
\begin{requirements}
	\req{NF310} Der im Zuge des Projekts erstellte Code ist gut
		\begin{itemize}
			\item zu warten (Einhaltung des Eclipse Codestils).
			\item zu erweitern (Modularer Aufbau).
			\item dokumentiert (Javadoc).
			\item mit Testfällen abgedeckt (jUnit, Emma).
		\end{itemize}
	\req{NF330} Das Projekt baut auf libgdx auf.
	\req{NF340} Eine kommerzielle Veröffentlichung des Produkts ist möglich, u.a. gilt:
		\begin{itemize}
			\item benutzte Assets und Bibliotheken sind kommerziell nutzbar.
			\item es finden sich Hinweise auf die jeweiligen Urheber und Lizenzen im Programm.
		\end{itemize}
	\req{NF350} Das Spiel verlangt vom Nutzer nur die nötigsten Daten und Berechtigungen.
\end{requirements}
