\section{Spielaufbau}

Das Spiel besitzt eine Levelarchitektur, wobei mehrere Level in Levelpaketen zusammengefasst werden.
Innerhalb eines solchen Levelpakets kann der Spieler durch das Lösen eines Levels das jeweils nächste Level freischalten und schrittweise das gesamte Paket abschließen.

\subsection{Spielelemente}
Der Spieler startet zunächst im Platziermodus und bekommt dort eine vom gewählten Level abhängige Ausgangskonstellation von Spielelementen präsentiert. 
Spielelemente sind dabei:

        \begin{description}
                \item[Alligatoren] haben eine Farbe und können eine beliebige Anzahl von Eiern und Familien beschützen.
                Sie repräsentieren die Abstraktionen im \(\lambda\)-Kalkül.
                Die Elemente, die vom Alligator beschützt werden, sind vertikal unter ihm angeordnet.
                Alligatoren fressen, falls sie dazu die Möglichkeit haben, die rechts von ihnen stehende Familie und verschwinden daraufhin (siehe "`Fressregel"').

                \item[Alte Alligatoren] beschützen mehrere Familien, fressen jedoch selbst keine andere Familien.
                Alte Alligatoren dienen zur Darstellung von Klammerungen und greifen sonst nicht selbst in die Auswertung ein.

                \item[Eier] haben, wie Alligatoren, eine Farbe.
                Eier können entweder alleine stehen oder von einem gleichfarbigen Alligator direkt oder indirekt beschützt werden. Sie repräsentieren die Variablen des \(\lambda\)-Kalküls.

                \item[Familien] sind Ansammlungen von Alligatoren, alten Alligatoren und Eiern, in denen normale und alte Alligatoren ihre Eier und Unterfamilien beschützen.
                Eine Familie besteht dabei immer aus einem Alligator, der beliebig viele Unterfamilien und Eier beschützt, wobei Unterfamilien wieder nach dem gleichen Schema aufgebaut sind.

        \end{description}

\subsection{Regeln}
	Die Auswertung der je nach Aufgabenstellung vom Spieler veränderten Ausgangskonstellation bildet dabei den zweiten Teil des Spiels und erfolgt im 			    	Simulationsmouds. 
	Sie folgt folgenden Regeln:

	\begin{description}
		            \item[Die Fressregel] In einer Konstellation aus verschieden zusammengestellten Familien frisst immer der am weitesten links oben stehende 						Alligator die rechts von ihm stehende Familie.
		            Durch das Fressen verschwindet dieser Alligator, und alle von ihm beschützten Eier gleicher Farbe werden durch die gefressene Familie ersetzt.
		            Dieser Vorgang wiederholt sich so lange, bis es keinen Alligator mehr gibt, der eine rechtsstehende Familie fressen kann.
		            Die Fressregel entspricht der \(\beta\)-Reduktion des \(\lambda\)-Kalküls.

		            \item[Die Umfärberegel] Ein Alligator kann keine Familie fressen, die eine Farbe enthält, welche auch in seiner Familie vorkommt.
		            Deshalb wird vor jedem Fressen überprüft, ob es zu einem Farbkonflikt kommt, und gegebenenfalls muss die zu fressende Familie umgefärbt werden.
		            Die Umfärberegel entspricht der \(\alpha\)-Konversion im \(\lambda\)-Kalkül.

		            \item[Alte Alligatoren Regel] Alte Alligatoren beschützen immer mindestens zwei Familien.
		            Bewacht ein alter Alligator nach einer Abfolge von Fressvorgängen nur noch eine Familie, so verschwindet er, da er nicht mehr benötigt wird.

	\end{description}


\subsection{Aufgabentypen}
Je nach Art des Levels gibt es verschiedene Ziele, die der Spieler zum Lösen eines Levels erreichen muss.
Folgende Aufgabentypen sind dabei möglich:
        \begin{description}
                \item[Färben und Einfügen] Der Spieler bekommt zu Spielbeginn eine Ausgangs- eine Endkonstellation, bestehend aus den Spielelementen.
                Das Ziel ist nun die Anfangskonstellation so zu verändern, dass sie nach Ausführung aller Auswertungsregeln (Siehe "`Auswertung"') in die gegebene Endkonstellation überführt wird.
                Das Manipulieren der Anfangskonstellation ist dabei die eigentliche Aufgabe des Spielers.
                In einfacheren Leveln kann der Spieler dazu vorgegebene Elemente der Ausgangskonstellation umfärben, in komplexeren Level kann er dann zusätzlich Spielelemente an dafür vorgesehenen Stellen in die Ausgangskonstellation einfügen. 
                Die zur Manipulation der Ausgangskonstellation zur Verfügung stehenden Spielelemente sind dabei je nach Level in ihrer Anzahl und Farbe beschränkt.

                \item[Schrittanzahl] In dieser Aufgabenstellung wird dem Spieler eine Anfangskonstellation vorgegeben, welche er dann, je nach Schwierigkeitsgrad, auf verschiedene Art und Weise verändern kann.
                Statt einer Endkonstellation wird eine bestimmte Anzahl von Auswertungsschritten vorgegeben, welche bei der Auswertung der von ihm veränderten Anfangskonstellation erreicht werden muss.

                \item[Multiple Choice] Eine weiterer Aufgabentyp besteht in einer gegebenen Ausgangskonstellation, welche aber diesmal nicht manipuliert werden kann. 
                Es wird eine bestimmte Anzahl an Endkonstellationen zur Auswahl gegeben und nach der erreichten Endkonstellation nach der Auswertung gefragt.

        \end{description}

        Falls bei Auswertung der vom Spieler veränderten Anfangskonstellation eine vorher definierte Anzahl an Auswertungsschritten oder auf dem Bildschirm angezeigten Spielelemente überschritten wird, so gilt das Level als nicht bestanden. 
        Gleiches gilt natürlich auch, falls die oben genannten Spielziele am Ende der Auswertung nicht erreicht wurden.
        

