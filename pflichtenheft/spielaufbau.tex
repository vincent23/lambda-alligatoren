\section{Spielaufbau}

\subsection{Spielelemente}

	\begin{description}
		\item[Alligatoren] Alligatoren können eine beliebige Anzahl von Eiern, mit denen sie eine Farbe teilen, beschützen.
		Alligatoren fressen, falls sie dazu die Möglichkeit haben, die rechts von ihnen stehende Familie und verschwinden daraufhin.

		\item[Alte Alligatoren] Diese Art von Alligatoren beschützt mehrere Familien, frisst jedoch selbst keine andere Familien.

		\item[Eier] Ein Ei wird immer von einem Alligator aus derselben Familie beschützt, es gibt also keine Eier, die nicht von einem normalen Alligator beschützt werden.
		Das Ei hat dabei dieselbe Farbe wie der es bewachende Alligator.

		\item[Familien] Familien sind Ansammlungen von Alligatoren, alten Alligatoren und Eiern, in denen normale und alte Alligatoren ihre Eier bzw. Unterfamilien beschützten.
		Die Familien haben dabei eine feste Struktur: An oberster Stelle kann ein alter Alligator stehen welcher von einer beliebigen Anzahl an normalen Alligatoren in vertikaler Anordnung gefolgt wird.
		Unter diesen befinden sich dann horizontal beliebig viele Eier und Unterfamilien, welche sich wieder nach dem gleichen Schema zusammensetzen.

	\end{description}

\subsection{Auswertungsregeln}

	\begin{description}

		\item[Die Fressregel] In einer Konstellation aus verschieden zusammengestellten Familien frisst immer der am weitesten links oben stehende Alligator die rechts von ihm stehende Familie.
		Durch das Fressen verschwindet dieser Alligator, und alle von ihm beschützten Eier werden durch die gefressene Familie ersetzt.
		Dieser Vorgang wiederholt sich so lange, bis es keinen Alligator mehr gibt, der eine rechtsstehende Familie fressen kann.

		\item[Die Umfärberegel] Eine Alligator kann keine Familie fressen, die die gleiche Farbverteilung wie seine eigene Familie besitzt.
		Deshalb wird vor jedem Fressen überprüft, ob es zu einem Farbkonflikt kommt und gegebenenfalls muss die zu fressende Familie umgefärbt werden.

		\item[Alte Alligatoren] Alte Alligatoren beschützten immer mindestens zwei Familien.
		Bewacht ein alter Alligator nach einer Abfolge von Fressvorgängen nur noch eine Familie, so verschwindet er.

	\end{description}

\subsection{Spielziele}

	\begin{itemize}

		\item Der Spieler bekommt zu Spielbeginn eine Ausgangs- und eine Endkonstellation bestehend aus den Spielelementen, also Familien von Alligatoren und Eiern, präsentiert. Das Spielziel ist nun die Anfangskonstellation so zu verändern, dass sie nach Ausführung aller Auswertungsregeln in die gegebene Endkonstellation überführt wird. Das Manipulieren der Anfangskonstellation ist dabei die eigentliche Aufgabe des Spielers. So hat er die Möglichkeit, gekennzeichnete Alligatoren oder Eier zu verfärben oder, bei komplexeren Levels, eine vorgegebenen Anzahl von Spielelementen in die Konstellation einzufügen und sie so zu konfigurieren.

		\item Eine andere Aufgabenstellung ist es dem Spieler wie oben beschrieben eine Anfangskonstellation vorzugeben, welcher er wieder manipulieren kann. Anstatt einer Endkonstellation wird aber eine bestimmte Anzahl von Auswertungsschritten vorgegeben, welche bei der Auswertung der von ihm verändert Anfangskonstellation erreicht werden muss.

		\item Ein weitere Spielidee ist dem Spieler eine Ausgangskonstellation zu präsentieren, die er aber selbst nicht manipulieren kann.
		Vielmehr wird ihm eine bestimmte Anzahl an Endkonstellationen zur Auswahl gegeben und er muss entscheiden, welche dieser Konstellation nach der Auswertung der Spielregeln erreicht wird.

		\item Falls bei Auswertung der vom Spieler veränderten Anfangskonstellation eine vorher definierte Anzahl an Auswertungsschritten oder auf dem Bildschirm angezeigten Spielelemente überschritten wird, so gilt das Level als nicht bestanden.
		Gleiches gilt natürlich auch falls die oben genannten Spielziele am Ende der Auswertung nicht erreicht wurden.
	\end{itemize}


