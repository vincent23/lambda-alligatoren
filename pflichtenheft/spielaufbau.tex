\section{Spielaufbau}

\subsection{Spielelemente}

	\begin{description}
		\item[Alligatoren] Alligatoren haben eine Farbe und bilden Familienoberhäupter. Das heißt, sie haben immer mindestens einen direkten Nachkommen.
		Sie repräsentieren die Abstraktionen im \(\lambda\)-Kalkül.
		Die Elemente, die zu der Familie eines Alligators gehören, sind dabei unter ihm vertikal angeordnet.
		Alligatoren fressen, falls sie dazu die Möglichkeit haben, die rechts von ihnen stehende Familie und verschwinden daraufhin (siehe "`Fressregel"').

		\item[Alte Alligatoren] Diese Art von Alligatoren haben immer mehrere direkte Nachkommen, frisst jedoch selbst keine andere Familien.
		Alte Alligatoren dienen nur zur Darstellung von Klammern und greifen sonst nicht selbst in die Auswertung ein.

		\item[Eier] Eier haben wie Alligatoren eine Farbe.
		Sie stellen die Variablen des \(\lambda\)-Kalküls dar.
		Aus ihnen schlüpft immer sofort die Familie, die ein Alligator mit der gleichen Farbe über ihm gerade gegessen hat.

		\item[Familien] Familien sind Ansammlungen von Alligatoren, alten Alligatoren und Eiern, die den gleichen normalen oder alten Alligator als Familienoberhaupt haben, eigeschlossen diesem.
		Eine Familie besteht dabei immer aus einem Alligator, der beliebig viele Unterfamilien und Eier als Nachkommen hat.
		Unterfamilien sind wieder genauso aufgebaut.
		Familien werden auch als Unterfamilien bezeichnet, wenn das Oberhaupt der Familie selbst wieder zu einer Familie gehört.

	\end{description}

\subsection{Auswertungsregeln}

	\begin{description}

		\item[Die Fressregel] In einer Konstellation aus verschieden zusammengestellten Familien frisst immer der am weitesten links oben stehende Alligator die rechts von ihm stehende Familie.
		Durch das Fressen verschwindet dieser Alligator, und alle von ihm beschützten Eier werden durch die gefressene Familie ersetzt.
		Dieser Vorgang wiederholt sich so lange, bis es keinen Alligator mehr gibt, der eine rechtsstehende Familie fressen kann.
		Die Fressregel entspricht der \(\beta\)-Reduktion des \(\lambda\)-Kalküls.

		\item[Die Umfärberegel] Eine Alligator kann keine Familie fressen, die die gleiche Farbverteilung wie seine eigene Familie besitzt.
		Deshalb wird vor jedem Fressen überprüft, ob es zu einem Farbkonflikt kommt und gegebenenfalls muss die zu fressende Familie umgefärbt werden.
		Die Umfärberegel entspricht der \(\alpha\)-Konversion im \(\lambda\)-Kalkül.

		\item[Alte Alligatoren] Alte Alligatoren beschützten immer mindestens zwei Familien.
		Bewacht ein alter Alligator nach einer Abfolge von Fressvorgängen nur noch eine Familie, so verschwindet er.

	\end{description}

\subsection{Spielziele}

	\begin{itemize}

		\item Der Spieler bekommt zu Spielbeginn eine Ausgangs- und eine Endkonstellation bestehend aus den Spielelementen, also Familien von Alligatoren und Eiern, präsentiert. Das Spielziel ist nun die Anfangskonstellation so zu verändern, dass sie nach Ausführung aller Auswertungsregeln in die gegebene Endkonstellation überführt wird. Das Manipulieren der Anfangskonstellation ist dabei die eigentliche Aufgabe des Spielers. So hat er die Möglichkeit, gekennzeichnete Alligatoren oder Eier zu verfärben oder, bei komplexeren Levels, eine vorgegebenen Anzahl von Spielelementen in die Konstellation einzufügen und sie so zu konfigurieren.

		\item Eine andere Aufgabenstellung ist es dem Spieler wie oben beschrieben eine Anfangskonstellation vorzugeben, welcher er wieder manipulieren kann. Anstatt einer Endkonstellation wird aber eine bestimmte Anzahl von Auswertungsschritten vorgegeben, welche bei der Auswertung der von ihm verändert Anfangskonstellation erreicht werden muss.

		\item Ein weitere Spielidee ist dem Spieler eine Ausgangskonstellation zu präsentieren, die er aber selbst nicht manipulieren kann.
		Vielmehr wird ihm eine bestimmte Anzahl an Endkonstellationen zur Auswahl gegeben und er muss entscheiden, welche dieser Konstellation nach der Auswertung der Spielregeln erreicht wird.

		\item Falls bei Auswertung der vom Spieler veränderten Anfangskonstellation eine vorher definierte Anzahl an Auswertungsschritten oder auf dem Bildschirm angezeigten Spielelemente überschritten wird, so gilt das Level als nicht bestanden.
		Gleiches gilt natürlich auch falls die oben genannten Spielziele am Ende der Auswertung nicht erreicht wurden.
	\end{itemize}


