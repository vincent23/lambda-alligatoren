\section{Glossar}
\begin{description}
	\item[Achievement] deutsch: "`Errungenschaft"'. Zusätzliches Spielziel, das nicht für den eigentlichen Spielverlauf relevant ist, aber zur Motiviation des Spielers dient.
		Der Spieler kann die erlangten Achievements betrachten und bekommt beim Erreichen des Ziels den Erfolg mitgeteilt.
	\item[API] Ist die Abkürzung für application programming interface (deutsch: Schnittstelle zur Anwendungsprogrammierung). Sie dient dazu anderen Programmen eine Anbindung an das zur Schnittstelle gehörende Softwaresystem zu bieten. Das dazugehörige Level zeigt im Falle von Android an welche Versionen von Android die API unterstützt. \url{http://developer.android.com/about/versions/android-4.0.3.html}
	\item[Assets] Medien, also u.a. Soundeffekte, Melodien und Grafiken
	\item[Automatische{[}r{]} Simulation{[}smodus{]}] Funktion, die Schritte innerhalb des Simulationsmodus automatisiert ausführt. Diese kann jederzeit gestartet sowie pausiert werden.
	\item [Avatar bzw. Spieleravatar] Ein Symbol oder eine Figur, die eine bestimmte Person in einem Spiel oder im Internet representiert.
	\item [Drag and Drop] Deutsch: "`Ziehen und Ablegen"'. Bedienungart von  einer Graphischen Benutzeroberfläche durch ein Zeigerelement, dass Elemente ergreift und verschiebt.
	\item [Konstellation] Die Representation eines Lambda-Terms durch Alligatoren und Eier auf dem Spielfeld.
	\item[Lambda-Kalkül (\(\lambda\)-Kalkül)] Eine formale Sprache. \url{http://de.wikipedia.org/wiki/Lambda-Kalkül}
	\item[Lambda-Term (\(\lambda\)-Term)] Ein formaler Ausdruck im \(\lambda\)-Kalkül.
		Er kann durch Anwendung der Regeln des Kalküls ausgewertet werden.
	\item[Level] Logischer Spielabschnitt, hier meist eine Rätselaufgabe.
	\item [Piktogramm] Ein einzelnes Symbol, dass seine Information durch vereinfachte graphische Darstellung vermittelt.
	\item [Pinchgeste]auch "`pinch-to-zoom"' Geste genannt: Das Auseinanderspreizen bzw. Zusammenziehen von Daumen und Zeigefinger. Üblicherweise wird sie zum Zoomen genutzt.
	\item [Sandbox] Deutsch: "`Sandkasten"'. ein Bereich, der dem Spieler ermöglicht seine ganze Kreativität auszuleben. Es gibt minimale Restriktionen seitens des Spiels.
	\item[Simulationsmodus] Modus, in dem der aktuelle Zustand des Spielfelds schrittweise nach den Regeln des \(\lambda\)-Kalküls ausgewertet wird.
	\item[Smartphone] ein Mobiltelefon, dass in der Lage ist viele der Funktionen eines Computers auszuführen. Üblicherweise verfügt es über ein großes Display und Touchbedienung.
	\item[Tablet bzw. Tablet Computer] ein dem Smartphone ähnliches mobiles Endgerät, dass jedoch über eine größere Bildschirmdiagonale verfügt.
	\item[Tutoriallevel] Level, dass eine neue Problemstellung im Spiel einführt und diese erklärt. Beispiele können das Verstehen beschleunigen.
	\item[Zoom] Das flüssige Vergrößern bzw. Verkleinern des gezeigten Spielausschnitts.
\end{description}
