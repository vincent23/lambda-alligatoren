\section{Glossar}
\begin{description}
	\item[Achievement] Zusätzliches Spielziel, das nicht für den eigentlichen Spielverlauf relevant ist, aber zur Motiviation des Spielers dient.
		Der Spieler kann die erlangten Achievements betrachten und bekommt beim Erreichen des Ziels den Erfolg mitgeteilt.
	\item[API] Ist die Abkürzung für application programming interface (deutsch: Schnittstelle zur Anwendungsprogrammierung). Sie dient dazu anderen Programmen eine Anbindung an das zur Schnittstelle gehörende Softwaresystem zu bieten. Das dazugehörige Level zeigt im Falle von Android an welche Versionen von Android die API unterstützt. \url{http://developer.android.com/about/versions/android-4.0.3.html}
	\item[Assets] Medien, also u.a. Soundeffekte, Melodien und Grafiken
	\item[Lambda-Kalkül (\(\lambda\)-Kalkül)] Eine formale Sprache. \url{http://de.wikipedia.org/wiki/Lambda-Kalkül}
	\item[Lambda-Term (\(\lambda\)-Term)] Ein formaler Ausdruck im \(\lambda\)-Kalkül.
		Er kann durch Anwendung der Regeln des Kalküls ausgewertet werden.
	\item[Level] Logischer Spielabschnitt, hier meist eine Rätselaufgabe.
	\item[Libgdx] Grafikbibliothek für Android. \url{http://libgdx.badlogicgames.com/index.html}
	\item[Simulationsmodus] Modus, in dem der aktuelle Zustand des Spielfelds schrittweise nach den Regeln des \(\lambda\)-Kalküls ausgewertet wird.
\end{description}
