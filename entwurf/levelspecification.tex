
\chapter{Level JSON Format Beschreibung}
\section{Präambel}
Die Spezifizierung von Levels des Spiels wird im verbreiteten JSON (JavaScript Object Notation) geschehen.
Die Gründe dafür sind
\begin{enumerate}[a)]
	\item Einfache Schreib- und Editierbarkeit
	\item Gute Les- und Nachvollziehbarkeit
	\item Geringerer Overhead gegenüber XML
	\item Unterstützung von Haus aus durch LibGDX
	\item Gute Austausch- und Erweiterbarkeit gegenüber datenbank- oder codebasierten Herangehensweisen
\end{enumerate}
Dafür ist es zunächst jedoch erforderlich, dass die Struktur der letztendlich für den Aufbau eines Levels benötigten Daten, detailliert spezifiziert ist.
Im folgenden werden die einzelnen Attribute eines Levels und ihre Bedeutungen, sowie ihre Repräsentation in JSON beschrieben.

\section{Grundlegendes}
Zunächst sei erwähnt, dass sich alle Elemente, die von der Anwendung benutzt werden, im Namespace "`de.croggle"' befinden.
Das heißt, dass sich in dem Wurzelobjekt jeder JSON Datei ein Objekt mit diesem Namen befindet.
\begin{lstlisting}[language=json,caption={Standardinhalt jeder JSON Datei der Anwendung}]
{
	"de.croggle" : {
		...
	}
}
\end{lstlisting}

Da die Beschreibungen von Levels, ähnlich wie Quellcode, komplexe Sachverhalte im Lambda Kalkül abbilden können, deren Wirkungsweise nicht direkt ersichtlich sein kann,
sieht die Spezifikation der JSON Repräsentation Kommentarattribute vor.
Diese können und sollen verwendet werden, um besonders schwer nachzuvollziehende Stellen zu dokumentieren, aber auch, um einfach auf die Ideen und Zielsetzungen innerhalb eines bestimmten Levels hinzuweisen.
Solche Kommentarattribute werden durch das Präfix "`\_comment"' markiert und eingeleitet.
Attributsnamen, die mit diesem beginnen sind dementsprechend reserviert.
Da die Kommentare jeweils durch eine Zeichenkette im Namen eines Attributes eingeleitet werden und JSON keine Namen von Attributen in Listen (eingeleitet durch "`["') vorsieht, sind Kommentare nur in Objekten (eingeleitet durch "`\{"') zulässig.
Dies ist insofern keine Einschränkung, da Listen im Allgemeinen nur anonyme Instanzen von Elementen beinhalten, deren es für gewöhnlich keinerlei weitere Erklärungen bedarf.
Der Wert eines solchen Kommentarattributes kann einerseits aus einem einfachen Stringliteral (Zeichenkette) bestehen, wenn es sich um kurze, einzeilige Kommentare handelt.
Andererseits kann bei längeren Kommentaren auch eine Liste von Stringliteralen angegeben werden, wobei jedes Element auf eine neue Zeile geschrieben wird.
Damit wird der Einschränkung im JSON Format Rechnung getragen, dass es keine Möglichkeit gibt, sogenannte Multiline Strings anzugeben.
Zuletzt ermöglicht die Spezifikation auch mehrere Kommentarattribute innerhalb des gleichen Objektes.
Dafür wird den jeweiligen "`\_comment"'-Attributnamen noch jeweils eine Ziffernfolge angehängt, welche von null beginnend aufsteigt.
Dies ist nötig, da der JSON Standard keine gleichen Attributnamen innerhalb einunddesselben Objekts vorsieht.
Listing 4.2 verdeutlicht die Anwendung von Kommenatren in JSON Files.
\begin{lstlisting}[language=json,caption={Kommentare in einer JSON Datei}]
{
	"_comment" : "Dies ist ein kurzer Kommentar",
	"de.croggle" : {
		"_comment": [
			"Längere Kommentare, die ",
			"über mehrere Zeilen gehen, ",
			"werden mittels einer Liste repräsentiert"
		]
	},
	"_comment0" : "Mehrere Kommentare in ein und demselben Objekt...",
	"_comment1" : "... werden mittels aufsteigender Nummern als Postfix unterschieden"
}
\end{lstlisting}

\section{Levelpakete}
Levels werden durch sogenannte Pakete gruppiert.
Diese Pakete besitzen folgende Eigenschaften, die durch die Spezifikation repräsentiert werden müssen:
\begin{enumerate}[a)]
	\item Einen Namen zur Bezeichnung des Pakets
	\item Eine Grafik, die zum Thema des Pakets passt
	\item Eine Position des Pakets, um die Reihenfolge der Pakete festzulegen
	\item Abhängigkeiten, die zum Freischalten der Box erfüllt sein müssen
	\item Eine Verbindung zwischen dem Paket und den dazugehörigen Leveln (kann allerdings auch nur im Level angegeben sein)
\end{enumerate}
Hierbei muss jedoch nicht allein auf JSON zurückgegriffen werden.
Auch Dateisysteminformationen können berücksichtigt werden.
Allerdings muss berücksichtigt werden, dass durch Festlegung von Konventionen keine zu starke Einschränkung passiert, sodass die Erweiterbarkeit nicht eingeschränkt wird.

\section{Levels}
Einzelne Levels 
