\section{Model View Controller}
Wir benutzen das Architekturmuster "`Model View Controller"' um einen flexiblen Programmentwurf zu bieten.
Dieser soll spätere Änderungen und Erweiterungen erleichtern und eine Wiederverwendung der einzelnen 
Programmkomponenten ermöglichen.
Die Realisierung des Musters erfolg bei uns über eine Unterteilung der verschiedenen Bereiche in verschiedene Packages. Das Package "`ui"' den Bereich des "`Views"' repräsentiert, die Unterpackges
 "`model"' in "`game"' und "`data"' das "`Model"' und das Package ??? den "`Controller"'.

\section{libgdx}
Als Framework benutzen wir libgdx. Libgdx ist speziell auf die Anforderungen der Spielentwicklung angepasst
und bietet so zur Entwicklung von "`Croggle"' an. Desweiteren bietet libgdx einen einfachen Aufbau des UIs sowie die Möglichkeit vereinfachtes Rendering zu nutzen. Die umfangreiche Dokumentation erleichtert die
Einarbeitung in das Framework. Auch durch die Tatsache, dass libgdx ein open source Projekt und freeware ist, bietet sich libgdx als Framework an. Nach kurzer Abwägung haben wir uns dahr gegen die verwendung des Android Frameworks und für die Verwendung von libgdx entschieden.

\section{Aufbau des "`Model"'}
Im "`Model"' haben wir uns dafür entschieden, die Repräsentation der Alligatoren Konstellationen in einer 
veränderbaren (mutable) Baumstruktur zu speichern. Im Gegensatz zur Variante die Konstellation in einer nicht veränderbaren (immutable) Form zu speichern, bietet diese Form den Vorteil, das Referenzen auf Objekte eindeutig sind und daher in der Baumstruktur auch auf den Vorgänger (parent) eines Baumelements zuzugreifen (getParent()). Muss die Konstellation im Laufe des Spiels geändert werden, können einzelne Teilbäume der Konstellation geändert werden. Allerdings ergibt sich hier das Problem, das beim einfügen von Teilbäumen an mehreren Stellen immer vollständige Kopien(deep copy) dieses Teilbaumes eingefügt werden müssen, damit es keine Probleme mit Referenzen auftreten.
Einen weiteren Grund für die Entscheidung mit veränderbaren Objekten zu arbeiten ist, dass diese ein vertrautes Konzept darstellen, wohingegen für die Arbeit mit nicht veränderbaren Objekten eine längere einarbeitungs Zeit in den Umgang und den Konzepte der funktionalen Programmierung nötig wäre. 

\section{Visitor Pattern}
Durch die Darstellung der Alligatoren Konstellation in einer Baumstruktur liegt die Verwendung des Visitor
 Patterns nahe. Das Muster wird hier vorragig benutzt um einfach neue Operationen, die auf der Konstelation ausgeführt werden sollen, einfach hinzu fügen zu können. Desweiteren sollen die verwandten Operationen, 
die auf den verschiedenen Objekten des Baumes ausgeführt werden, zentral im Besucher gespeichert werden um
sie so einfach und zentral verwalten und modifizieren zu können. 

