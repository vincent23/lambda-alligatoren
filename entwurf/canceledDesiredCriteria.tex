\chapter{Nicht umgesetzte Wunschkriterien}

\section{Sandboxlevel speichern/laden}
	Der Leveleditor wird wegen der sehr aufwändigen Umsetzung und daraus resultierender Zeitprobleme 
nicht in den Entwurf aufgenommen und daher aus dem Projekt gestrichen.
Dies betrifft Punkt /F260+/ des Pflichtenheftes.  
 
\section{Avatarerstellung}
Die Erstellung eigener Avatare durch den Nutzer wird  nicht umgesetzt, da es für das eigentliche Kernspiel und das damit verbundene Lernen nicht von Relevanz ist.
Es würde jedoch einen nicht zu vernachlässigenden Teil der verfügbaren Entwicklungs- und Implementatierungszeit in Anspruch nehmen, weshalb diese Entscheidung nötig wurde.
Die Avatarwahl funktioniert stattdessen über die vom System bereitgestellten Avatare, die vom Spieler ausgewählt werden können.
Dies betrifft den Punkt /F130+/ des Pflichtenheftes.  

\section{Automatische Kameraführung}
Die automatische Kameraführung während der Auswertung wird nicht umgesetzt.
Nach ausführlichem Abwägen wurde dieses Wunschkriterium wegen schlechtem Verhältnis von Nutzen zu Aufwand gestrichen.
Dies betrifft Punkt /F236+/ des Pflichtenheftes. 

\section{Achievements}
Die Achievements für einen nichtterminierenden Ausdruck und für eine bestimmte Anzahl an Zeilen und Spalten werden gestrichen.
Der Grund dafür liegt darin, dass sich diese Achievements in der Umsetzung stark von den anderen Arten von Achievements unterscheiden.
Wegen schwieriger Definition einer "`Spielsitzung"' auf mobilen Geräten wird außerdem das Achievement für eine bestimmte Anzahl an gelösten Leveln in einer Spielsitzung gestrichen.
Dies betrifft Teile des Punktes /F250+/ des Pflichtenheftes.  
