\chapter{Einführung}

In der Entwurfsphase wird der Grundstein für eine gute Implementierungsphase gelegt.
Daher ist es von enormer Wichtigkeit, dass in dieser Phase sehr sorgfältig vorgegangen wird.

Ausgehend von den Anforderungen, die während der Planungsphase ersonnen und im Pflichtenheft festgehalten wurden, muss jetzt eine softwaretechnische Lösung im Sinne einer Softwarearchitektur entwickelt werden, die ebenjene Anforderungen zu erfüllen vermag.
Um einen hohen Standard bei der Entwicklung der Software zu gewährleisten, werden anerkannte Prinzipien der Softwareentwicklung berücksichtigt.

Zu diesen gehören das Geheimnisprinzip, das wichtige Entwurfsentscheidungen in einzelnen Klassen kapselt, und somit wenige Details der Implementierung nach außen dringen lässt.
Um die Verständlichkeit des Entwurfs und später der Implementierung zu garantieren, werden zyklische Abhängigkeiten zwischen Klassen vermieden, und das Lokalitätsprinzip angewendet.
Spätere Änderungen des Codes sollen durch die Trennung des Verhaltens von der Implementierung, genauso wie durch lose Kopplung zwischen den einzelnen Klassen und Paketen, gewährleistet werden.

Durch die Anwendung dieser Prinzipen soll das Projekt "`Croggle"' höchsten Anforderungen genügen.
