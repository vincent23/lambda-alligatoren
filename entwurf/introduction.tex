\chapter{Einführung}

In diesem Entwurfsdokument soll sowohl die dahinterstehende Architektur als auch Designkonzepte der Applikation "´Croggle"' vermitteln.
Die Applikation will Grundschülern schon in ihrem jungen Alter grundlegende Prinzipien und Funktionsweisen der funktionalen Programmierung, speziell dem untypisiertem Lambda Kalkül, zu vermitteln. Die Applikation folgt dem Model-View-Controller Muster um die logische Aufteilung von Repräsentation, Darstellung und Verwaltung zu gewährleisten. Dies ist essenziel um ein flexibles Programmdesign zu erhalten, bei dem Modifikationen, Erweiterungen und Wartung ohne signifikanten Aufwand möglich sind. Den wichtigesten Teil dieses Dokumentes bildet das Klassendiagramm, in dem alle wichtigen Klassen, Methoden und Attribute modelliert sind. Ein weiterer essentieller Punkt des Entwurfsdokumentes sind Sequenzdiagramme, die die Interaktionen zwischen den einzelnen Klassen darstellen. Zusätzlich werden andere 
Punkte die für den Aufbau der Applikation wichtig sind genau beschrieben, dazu gehören der Aufbau der JSON Dateien für Level und Wunschkriterien, die im Pflichtenheft beschrieben aber im Entwurf nicht umgesetzt wurden.
