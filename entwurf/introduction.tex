\chapter{Einführung}

In diesem Entwurfsdokument soll sowohl die dahinterstehende Architektur als auch Designkonzepte der Applikation "`Croggle"' vermitteln.
Die Applikation will Grundschülern schon früh grundlegende Prinzipien und Funktionsweisen der funktionalen Programmierung, speziell dem untypisiertem Lambda-Kalkül, vermitteln.
Die Applikation folgt dem Model-View-Controller Muster, um die logische Aufteilung von Repräsentation, Darstellung und Verwaltung zu gewährleisten.
Den wichtigesten Teil dieses Dokumentes bildet das Klassendiagramm in dem alle wichtigen Klassen, Methoden und Attribute modelliert sind.
Ein weiterer essenzieller Punkt des Entwurfsdokumentes sind Sequenzdiagramme, die die Interaktionen zwischen den einzelnen Klassen darstellen.
Zusätzlich werden andere Punkte, die für den Aufbau der Applikation wichtig sind, genau beschrieben. Dazu gehören der Aufbau der JSON Dateien für Level und Wunschkriterien, die im Pflichtenheft beschrieben, aber im Entwurf nicht umgesetzt wurden.
