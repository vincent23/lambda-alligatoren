\section{Einleitung}

In der Entwurfsphase wird der Grundstein für eine gute Implementierungsphase gelegt. Das ist der Grund warum es von enormer Wichtigkeit ist, dass in dieser Phase sehr sorgfältig vorgangen wird.

Ausgehend von den Anforderungen, die wir während der Planungsphase ersonnen und in unserem Pflichtenheft festgehalten haben, müssen wir jetzt eine softwaretechnische Lösung im Sinne einer Softwarearchitektur entwickeln, die ebenjene Anforderungen zu erfüllen vermag.
Um einen hohen Standard bei der Entwicklung der Software zu gewährleisten werden anerkannte Prinzipien der
Softwareentwicklung berücksichtigt. Zu diesen gehören das Geheimnisprinzip, das wichtige 
Entwurfsentscheidungen in einzelne Klassen kapselt und so wenige Details der Implementierung nach außen dringen lässt. Um die Verständlichkeit des Entwurfs und später der Implementierung zu garantieren 
werden zyklische Abhängigkeiten zwischen Klassen vermieden und das Lokalitätsprinzip angewendet. Spätere
 Änderungen des Codes sollen durch die Trennung des Verhaltens und der Implementierung genauso wie durch eine Lose Kopplung zwischen den einzelnen Klassen und Paketen gewährleistet werden. Durch die Anwendung
dieser Prinzipen soll das Projekt "`Croggle"' höchsten Anforderungen genügen. 
