\documentclass[t]{beamer}
\usetheme[deutsch]{KIT}
\setbeamercovered{transparent}
\setbeamertemplate{navigation symbols}{}
\graphicspath{ {Systemmodelle/images_blank/} }

\KITfoot{ Croggle - Praxis der Softwareentwicklung WS 13/14}
\usepackage[utf8]{inputenc}
\usepackage{ngerman}
\usepackage[scaled]{uarial}
\renewcommand*\familydefault{\sfdefault}
\usepackage[T1]{fontenc}
\usenavigationsymbols


\usepackage{listings}
\usepackage{xcolor}

% from http://tex.stackexchange.com/questions/83085/how-to-improve-listings-display-of-json-files
% be careful not to extend lines beyond 88 characters of width.
\colorlet{punct}{red!60!black}
\definecolor{background}{HTML}{EEEEEE}
\definecolor{delim}{RGB}{20,105,176}
\colorlet{numb}{magenta!60!black}
\lstdefinelanguage{json}{
	% TODO remove hardcoded, smaller than normal font size if switching font
	basicstyle=\normalfont\ttfamily\footnotesize,
	numbers=left,
	numberstyle=\scriptsize,
	stepnumber=1,
	numbersep=8pt,
	showstringspaces=false,
	breaklines=true,
	frame=lines,
	backgroundcolor=\color{background},
	tabsize=2,
	literate=
		*{0}{{{\color{numb}0}}}{1}
		 {1}{{{\color{numb}1}}}{1}
		 {2}{{{\color{numb}2}}}{1}
		 {3}{{{\color{numb}3}}}{1}
		 {4}{{{\color{numb}4}}}{1}
		 {5}{{{\color{numb}5}}}{1}
		 {6}{{{\color{numb}6}}}{1}
		 {7}{{{\color{numb}7}}}{1}
		 {8}{{{\color{numb}8}}}{1}
		 {9}{{{\color{numb}9}}}{1}
		 {:}{{{\color{punct}{:}}}}{1}
		 {,}{{{\color{punct}{,}}}}{1}
		 {\{}{{{\color{delim}{\{}}}}{1}
		 {\}}{{{\color{delim}{\}}}}}{1}
		 {[}{{{\color{delim}{[}}}}{1}
		 {]}{{{\color{delim}{]}}}}{1}
		 {ö}{{\"o}}{1}
         {ä}{{\"a}}{1}
         {ü}{{\"u}}{1},
}

% \strong from http://tex.stackexchange.com/questions/14667/does-latex-define-a-semantic-equivalent-of-textbf
\makeatletter
\newcommand{\strong}[1]{\@strong{#1}}
\newcommand{\@@strong}[1]{\textbf{\let\@strong\@@@strong#1}}
\newcommand{\@@@strong}[1]{\textnormal{\let\@strong\@@strong#1}}
\let\@strong\@@strong
\makeatother


\title{Croggle}
\subtitle{PSE - Entwurfsphase \\[0.3cm]
Lukas Böhm $\cdot$ Tobias Hornberger $\cdot$ Jonas Mehlhaus \\ Iris Mehrbrodt  $\cdot$ Vincent Schüßler $\cdot$ Lena Winter}
\institute[IPD]{Institut für Programmstruktutren und Datenorganisation}

\TitleImage[height=\titleimageht]{banner.png}


\begin{document}

\begin{frame}
        \maketitle
\end{frame}

\begin{frame}
	\frametitle{Designentscheidungen}
	Wegweisende Entscheidungen für den weiteren Aufbau der App:\\
	\begin{itemize}
		\item Model View Controller
		\item libgdx
		\item Aufbau des Model
		\item Visitor Pattern
		\item Levelrepräsentation mit JSON
		\item SQLite Datenbank
		\item Observer/Listener Pattern
	\end{itemize}
	%%\includegraphics<1>[width=0.85\textwidth]{mvc.jpg}

	%%\includegraphics<2>[width=0.8\textwidth]{libgdx.png}
\end{frame}

\begin{frame}
	\frametitle{Nicht umgesetzte Wunschkriterien}
	Nicht modellierte Wunschkriterien, die nicht weiter verfolgt werden: \\
	\begin{itemize}
		\item Avatarerstellung (/F130+/)
		\item Automatische Kameraführung (/F236+/)
		\item Bestimmte Achievements (/F250+/)
		\begin{itemize}
			\item Nichtterminierender Ausdruck
			\item Anzahl Spalten/Zeilen
			\item Erreichtes innerhalb einer Spielsitzung
		\end{itemize}

		\item Sandboxlevel speichern/laden (/F260+/)
	\end{itemize}
\end{frame}

\begin{frame}
	\frametitle{Repräsentation der \(\lambda\)-Terme}
	\begin{center}
		\includegraphics[width=0.8\textwidth]{umlAwesome.png}
	\end{center}
\end{frame}

\begin{frame}
	\frametitle{Sequenzdiagramm \(\beta\)-Reduktion: (\(\lambda\)x.x) y}
	\begin{center}
	\includegraphics<1>[width= 0.4\textwidth]{Alligator1.png}
	\includegraphics<2>[width=0.5\textwidth]{Beta-Reduktion.png}
	\end{center}
\end{frame}

\begin{frame}
	\frametitle{Call-by-Name Anwendung}
	\begin{center}
	\includegraphics[width=0.5\textwidth]{Beta-Reduktion-withGreen.png}
	\end{center}
\end{frame}

\begin{frame}
	\frametitle{Call-by-Name Anwendung: Detailansicht}
	\includegraphics[width=\textwidth]{FindLocation.png}
\end{frame}



\begin{frame}
	\frametitle{Änderungen im Baum}
	\begin{center}
		\includegraphics<1>[width=0.5\textwidth]{Beta-Reduktion-withBlue.png}
	\end{center}
\end{frame}

\begin{frame}
	\frametitle{Änderungen im Baum: Detailansicht}
	\includegraphics[width=0.95\textwidth]{ReplaceEggs.png}
\end{frame}

\begin{frame}
	\frametitle{Spielbrett aufräumen}
	\begin{center}
		\includegraphics[width=0.5\textwidth]{Beta-Reduction-withYellow.png}
	\end{center}
\end{frame}

\begin{frame}
	\frametitle{Ergebnis}
	\begin{center}
		\includegraphics[width=0.4\textwidth]{Alligator2.png}
	\end{center}
\end{frame}

\begin{frame}
	\frametitle{}
	\begin{center}
	\begin{Huge}
		Vielen Dank\\ für Ihre Aufmerksamkeit
	\end{Huge}
	\end{center}
\end{frame}

%%\begin{frame}
%%	\frametitle{\(\beta\)-Reduktion: Detailansicht}
%%	\includegraphics[width=\textwidth]{unnoetig.png}
%%\end{frame}

\end{document}
