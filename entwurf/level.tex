\chapter{Level}
In den Anfangskonstellationen werden ungefärbte Elemente durch die Variable o dargestellt. 
Besondere Level wie Tutoriallevel haben hier zusätzlich eine Beschreibung um ihr Ziel zu verdeutlichen.
Farben werden hier als Variablen dargestellt.

\section{Levelpacket 1}
\begin{itemize}
	\item{Level 1} 
		\begin{itemize}
			\item{Typ:} Färbelevel 
			\item{Anfangskonstellation:} \( o\)
			\item{Endkonstellation:} \( x\)
			\item{gesperrte Farben:} -  
			\item{Beschreibung:} Erstes Tutoriallevel in dem das Einfärben von Elementen erklärt wird.
		\end{itemize}

	\item{Level 2} 
		\begin{itemize}
			\item{Typ:} Färbelevel 
			\item{Anfangskonstellation:} \((\lambda o . o ) y\)   
			\item{Endkonstellation:} \(y\) 
			\item{gesperrte Farben:} y 
			\item{Beschreibung:} Zweites Tutoriallevel in dem die \(\beta\)-Reduktion gezeigt wird.
								Benötigte Kenntnisse des Spieler hierfür ist das Einfärben von Elementen. 
		\end{itemize}

	\item{Level 3} 
		\begin{itemize}
			\item{Typ:} Einfügelevel 
			\item{Anfangskonstellation:} \(\lambda x . x \)   
			\item{Endkonstellation:} \(y\)
			\item{gesperrte Farben:} -  
			\item{Beschreibung:} Drittes Tutoriallevel in dem das Einfügen von Elementen auf das Spielfeld erklärt wird.
								Benötigte Kenntnisse des Spieler hierfür ist das Einfärben von Elementen. 
		\end{itemize}

	\item{Level 4} 
		\begin{itemize}
			\item{Typ:} Einfügelevel
			\item{Anfangskonstellation:} \(\lambda x . x \)    
			\item{Endkonstellation:} \(y y\)
			\item{gesperrte Farben:} - 
		\end{itemize}

	\item{Level 5} 
		\begin{itemize}
			\item{Typ:} Einfügelevel
			\item{Anfangskonstellation:} \(\lambda x . x \)    
			\item{Endkonstellation:}  \(\lambda x . x  \lambda x . x  \)
			\item{gesperrte Farben:} x
		\end{itemize}

	\item{Level 6} 
		\begin{itemize}
			\item{Typ:} Einfärbelevel mit Schrittanzahl
			\item{Schrittanzahl:} 5
			\item{Anfangskonstellation:} \(\lambda o . o o  \lambda o . o o \lambda o . o o  \lambda o . o o \lambda o . o o  \lambda o . o o\) 
			\item{Endkonstellation:}  -
			\item{gesperrte Farben:} -
			\item{Beschreibung:}Viertes Tutoriallevel, das das System der Schrittanahllevel erklärt.
		\end{itemize}

	\item{Level 7} 
		\begin{itemize}
			\item{Typ:} Multiple-Choice 
			\item{Anfangskonstellation:} \((\lambda x . x ) y\)    
			\item{Wahlmöglichkeiten:}  
				\begin{itemize}
					\item[1.] \(y\)
					\item[2.] \(\lambda x . x \) 
					\item[3.] \(y y\)
				\end{itemize}
			\item{Endkonstellation:} \(y\)
			\item{Beschreibung:}Fünftes Tutoriallevel in dem das Sytem des Multiple-Choice Levels erklärt wird.
		\end{itemize}

	\item{Level 8} 
		\begin{itemize}
			\item{Typ:} Multiple-Choice 
			\item{Anfangskonstellation:} \((\lambda x . \lambda y . x ) z\)    
			\item{Wahlmöglichkeiten:}  
				\begin{itemize}
					\item[1.] \( \lambda y . z\) 
					\item[2.] \( \lambda x . x \) 
					\item[3.] \( \lambda x . z\)
				\end{itemize}
			\item{Endkonstellation:}\( \lambda y . z\)
		\end{itemize}

	\item{Level 9} 
		\begin{itemize}
			\item{Typ:} Einfärbelevel mit Schrittanzahl
			\item{Schrittanzahl:} 10
			\item{Anfangskonstellation:} \(\lambda o . o o \)  \(\lambda o . o o \) 
			\item{Endkonstellation:}  -
			\item{gesperrte Farben:} -
		\end{itemize}

	\item{Level 10} 
		\begin{itemize}
			\item{Typ:} Multiple-Choice 
			\item{Anfangskonstellation:} \((\lambda x . \lambda y . x x ) u v\)    
			\item{Wahlmöglichkeiten:}  
				\begin{itemize}
					\item[1.] \( u u\) 
					\item[2.] \( v v \) 
					\item[3.] \( \lambda y . u\)
				\end{itemize}
			\item{Endkonstellation:}\( \lambda u u\)
		\end{itemize}



\end{itemize}
