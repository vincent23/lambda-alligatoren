\documentclass[parskip=full]{scrreprt}
\usepackage[T1]{fontenc}
\usepackage[utf8]{inputenc}
\usepackage[ngerman]{babel}
\usepackage{lmodern}
\usepackage{color}
\usepackage{mathtools}
\usepackage{amssymb}
\usepackage{graphicx}
\usepackage{hyperref}
\usepackage{dirtree}% directory trees
\usepackage{listings}% for source code listings
\usepackage{xcolor}% for the json formatting to work without modification
\usepackage{needspace}

\begin{document}

\chapter{Inhalte einer Level JSON Datei}
\section{Multiple-Choice-Level}
\dirtree{%
.1 \emph{"'de{.}croggle"'} : Object - Namensraum.
	.2 \emph{"'levels"'} : List - Liste mit Levelobjekten{.} Gewöhnlich nur ein Paket.
		.3 : Object - Darstellung einzelner Levels.
			.4 \emph{"'type"'} : "'multiple choice"'.
			.4 \emph{"'description"'} : String - Beschreibung zu einem Level.
			.4 \emph{"'design"'} : String - Pfad zu einem Leveldesign.
			.4 \emph{"'animation"'} : String - Pfad zu einer Animation.
			.4 \emph{"'abort simulation after"'} : Integer - Nach wie vielen Schritten ein Level gewonnen (positiv) oder verloren (negativ) ist.
			.4 \emph{"'hints"'} : List - Strings mit Pfaden zu Hilfegrafiken.
			.4 \emph{"'data"'} (multiple choice) : Object - Spezielle Daten für die einzelnen Leveltypen.
				.5 \emph{"'initial"'} : Object - Board Objekt der An\-fangs\-kon\-stel\-la\-tion.
				.5 \emph{"'answers"'} : List - Liste mit Board Objekten der möglichen Antworten.
				.5 \emph{"'correct answer"'} : unsigned Integer - Der Index der richtigen Antwort zur Fragestellung.
}

\section{Färbe-Level}
\dirtree{%
.1 \emph{"'de{.}croggle"'} : Object - Namensraum.
	.2 \emph{"'levels"'} : List - Liste mit Levelobjekten{.} Gewöhnlich nur ein Paket.
		.3 : Object - Darstellung einzelner Levels.
			.4 \emph{"'type"'} : "'color edit"'.
			.4 \emph{"'description"'} : String - Beschreibung zu einem Level.
			.4 \emph{"'design"'} : String - Pfad zu einem Leveldesign.
			.4 \emph{"'animation"'} : String - Pfad zu einer Animation.
			.4 \emph{"'abort simulation after"'} : Integer - Nach wie vielen Schritten ein Level gewonnen (positiv) oder verloren (negativ) ist.
			.4 \emph{"'hints"'} : List - Strings mit Pfaden zu Hilfegrafiken.
			.4 \emph{"'data"'} (multiple choice) : Object - Spezielle Daten für die einzelnen Leveltypen.
				.5 \emph{"'initial constellation"'} : Object - Board Objekt der An\-fangs\-kon\-stel\-la\-tion.
				.5 \emph{"'objective"'} : Object - Board Objekt der zu erreichenden Konstellation.
				.5 \emph{"'user colors"'} : List - Liste mit 6 verschiedenen Integer Werten, die die Farben beschreiben, die der Benutzer zum Färben von Elementen benutzen soll. 
				.5 \emph{"'blocked colors"'} : List - Liste mit Integer Werten, die blockierte Farben beschreiben.
}

\section{Einfüge-Level}
\dirtree{%
.1 \emph{"'de{.}croggle"'} : Object - Namensraum.
	.2 \emph{"'levels"'} : List - Liste mit Levelobjekten{.} Gewöhnlich nur ein Paket.
		.3 : Object - Darstellung einzelner Levels.
			.4 \emph{"'type"'} : "'term edit"'.
			.4 \emph{"'description"'} : String - Beschreibung zu einem Level.
			.4 \emph{"'design"'} : String - Pfad zu einem Leveldesign.
			.4 \emph{"'animation"'} : String - Pfad zu einer Animation.
			.4 \emph{"'abort simulation after"'} : Integer - Nach wie vielen Schritten ein Level gewonnen (positiv) oder verloren (negativ) ist.
			.4 \emph{"'hints"'} : List - Strings mit Pfaden zu Hilfegrafiken.
			.4 \emph{"'data"'} (multiple choice) : Object - Spezielle Daten für die einzelnen Leveltypen.
				.5 \emph{"'initial constellation"'} : Object - Board Objekt der An\-fangs\-kon\-stel\-la\-tion.
				.5 \emph{"'objective"'} : Object - Board Objekt der zu erreichenden Konstellation.
				.5 \emph{"'blocked colors"'} : List - Liste mit Integer Werten, die blockierte Farben beschreiben.
				.5 \emph{"'user colors"'} : List - Liste mit 6 verschiedenen Integer Werten, die die Farben beschreiben, die der Benutzer zum Färben von Elementen benutzen soll.
				.5 \emph{"'blocked types"'} : List - String Liste mit Namen von Elementtypen, die im Level nicht platzierbar sind (egg, colored alligator, aged alligator).
}
\end{document}
