\chapter{Umsetzung der Muss- und Wunschkriterien}


\section{Musskriterien}

% TODO
Alle Musskriterien konnten vollständig implementiert werden. Zur besseren Übersicht ist hier die Art und Weise der Umsetzung kurz zusammengefasst.

\subsection{Erstellung und Auswertung von Alligatorenkonstellationen}

% TODO mal schauen wie weit das mit dem drag & drop noch wird
Es können vorgegebene Alligatorenkonstellationen, je nach Leveltyp, entweder nur umgefärbt, oder durch weitere Objekte vom Spieler ergänzt werden.
Die so erstellten Konstellationen können im Simulationsmodus schrittweise ausgewertet werden.

\subsection{Kontrolle des Lernfortschritts durch Eltern oder Lehrer}

Über die erfassten Statistiken können Eltern und Lehrer den Lernfortschritt der Spieler einsehen.

\subsection{Interaktive Einführung und Erklärung der Regeln}

% TODO lol, das haben wir irgendwie vergessen, oder?
Ähh jaaa\dots

\subsection{Bedienung über ein Tablet mit Toucheingabe}

Die Applikation ist vollständig über Toucheingabe bedienbar, wobei sowohl Multi-Touch-Gesten implementiert wurden, aber ebenso eine Single-Touch-Bedienung möglich ist.
Zudem ist die Applikation auf aktuellen Android Tablets gemäß der Spezifikation im Pflichtenheft lauffähig.

\subsection{Langzeitmotivation des Spielers wird aufrechterhalten}

Über das Achievementsystem und die einfache Erweiterung um neue Level wird eine Langzeitmotivation gewährleistet.


\section{Umgesetzte Wunschkriterien}

Folgende Wunschkriterien konnten implementiert werden.

\subsection{Speicherung mehrerer Spielstände für verschiedene Benutzer}

Die Applikation unterstützt das Anlegen von sechs verschiedenen Benutzern und den Wechsel zwischen diesen.
Statistiken, Achievements und Levelfortschritt werden für diese Benutzer jeweils separat gespeichert und erlauben so eine Trennung der Spielerprofile.

\subsection{Verschiedene Leveltypen zur Verdeutlichung unterschiedlicher Aspekte des Lambda-Kalküls}

% TODO haben wir das überhaupt richtig umgesetzt?
% war mit mehreren leveltypen auch multiple-choice etc. gemeint oder so dinge wie church numerals etc.?

\subsection{Unterstützung mehrerer Sprachen}

Bei der Implementierung wurde auf eine einfache Lokalisierbarkeit geachtet.
Bisher ist eine englische und eine deutsche Version vorhanden, Erweiterung um weitere Sprachen sind durch die Erstellung einer weiteren Sprachdatei einfach möglich.

\subsection{Hilfestellungen für Farbenblinde}

Die Anwendung erlaubt die Aktivierung eines "'Farbenblindmodus"`, wodurch die Farben der Alligatoren und Eier durch zweifarbige, kontrastreiche Muster ersetzt werden.
Hierbei fehlt noch eine vollständige Abdeckung aller 30 Farben, jedoch können weitere Muster einfach hinzugefügt werden.

\section{Nicht umgesetzte Wunschkriterien}

Zusätzlich zu den bereits im Entwurf gestrichenen Wunschkriterien wurden in der Implementierung folgende nicht umgesetzt.

\subsection{Für Smartphones angepasste Version}

Bereits die unangepasste, eigentlich für Tablets ausgelegte Version der Applikation stellte sich auch auf Smartphones als überaschend gut nutzbar heraus.
Daher wurde keine zusätzliche Arbeit in die Erstellung einer gesonderten Version für kleinere Geräte investiert, auch um den Testaufwand einzuschränken.
Zudem ist durch die Nutzung des Model-View-Controller-Musters eine solche Anpassung auch noch im Nachhinein gut möglich und könnte in einer zukünftigen Version umgesetzt werden.

\subsection{Vermittlung einer Geschichte durch Animationen}

Da die Erstellung von weiterem Artwork, insbesondere von Animationen, sehr viel mehr Zeit gekostet hätte, wurde dieses Kriterium nicht implementiert.
Das Format zur Beschreibung der Level sieht dies jedoch weiterhin vor, und eine Anzeige der Animationen wäre relativ einfach zu implementieren.
Zur Umsetzung fehlen daher hauptsächlich Animationen, die durch das flexible Levelladesystem einfach nachgeliefert werden können.
