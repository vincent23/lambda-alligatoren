\chapter{Umsetzung der Muss- und Wunschkriterien}


\section{Musskriterien}

\subsection{Erstellung und Auswertung von Alligatorenkonstellationen}

\subsection{Kontrolle des Lernfortschritts durch Eltern oder Lehrer}

\subsection{Interaktive Einführung und Erklärung der Regeln}

\subsection{Bedienung über ein Tablet mit Toucheingabe}

\subsection{Langzeitmotivation des Spielers wird aufrechterhalten}


\section{Umgesetzte Wunschkriterien}

\subsection{Speicherung mehrerer Spielstände für verschiedene Benutzer}

\subsection{Verschiedene Leveltypen zur Verdeutlichung unterschiedlicher Aspekte des \(\lambda\)-Kalküls}

\subsection{Unterstützung mehrerer Sprachen}

\subsection{Hilfestellungen für Farbenblinde}


\section{Nicht umgesetzte Wunschkriterien}

Zusätzlich zu den bereits im Entwurf gestrichenen Wunschkriterien wurden in der Implementierung folgende nicht umgesetzt.

\subsection{Für Smartphones angepasste Version}

Bereits die unangepasste, eigentlich für Tablets ausgelegte Version der Applikation stellte sich auch auf Smartphones als überaschend gut nutzbar heraus.
Daher wurde keine zusätzliche Arbeit in die Erstellung einer gesonderten Version für kleinere Geräte investiert, auch um den Testaufwand einzuschränken.
Zudem ist durch die Nutzung des Model-View-Controller-Musters eine solche Anpassung auch noch im Nachhinein gut möglich und könnte in einer zukünftigen Version umgesetzt werden.

\subsection{Vermittlung einer Geschichte durch Animationen}

Da die Erstellung von weiterem Artwork, insbesondere von Animationen, sehr viel mehr Zeit gekostet hätte, wurde dieses Kriterium nicht implementiert.
Das Format zur Beschreibung der Level sieht dies jedoch weiterhin vor, und eine Anzeige der Animationen wäre relativ einfach zu implementieren.
Zur Umsetzung fehlen daher hauptsächlich Animationen, die durch das flexible Levelladesystem einfach nachgeliefert werden können.
