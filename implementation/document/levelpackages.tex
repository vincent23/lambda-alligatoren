\chapter{Levelpakete}
In den Anfangskonstellationen werden ungefärbte Elemente durch die Variable \strong{"'o"'} dargestellt. 
Besondere Level wie Tutoriallevel haben hier zusätzlich eine Beschreibung um ihr Ziel zu verdeutlichen.
Farben werden hier als Variablen dargestellt.

\section{Levelpaket 0}
\begin{itemize}
	\item{Level 0} 
		\begin{description}
			\item[Typ:] Färbelevel 
			\item[Anfangskonstellation:] \( o\)
			\item[Endkonstellation:] \( x\)
			\item[gesperrte Farben:] -  
			\item[Beschreibung:] Erstes Tutoriallevel, in dem das Einfärben von Elementen erklärt wird.
		\end{description}

	\item{Level 1} 
		\begin{description}
			\item[Typ:] Färbelevel 
			\item[Anfangskonstellation:] \((\lambda o . o ) y\)   
			\item[Endkonstellation:] \(y\) 
			\item[gesperrte Farben:] y 
			\item[Beschreibung:] Zweites Tutoriallevel, in dem die \(\beta\)-Reduktion gezeigt wird.
								Benötigte Kenntnis des Spielers hierfür ist das Einfärben von Elementen. 
		\end{description}

	\item{Level 2} 
		\begin{description}
			\item[Typ:] Einfügelevel 
			\item[Anfangskonstellation:] \(\lambda x . x \)   
			\item[Endkonstellation:] \(y\)
			\item[gesperrte Farben:] -  
			\item[Beschreibung:] Drittes Tutoriallevel, in dem das Einfügen von Elementen auf das Spielfeld erklärt wird.
								Benötigte Kenntnis des Spielers hierfür ist das Einfärben von Elementen. 
		\end{description}

	\item{Level 3} 
		\begin{description}
			\item[Typ:] Einfügelevel
			\item[Anfangskonstellation:] \(\lambda x . x \)    
			\item[Endkonstellation:] \(y y\)
			\item[gesperrte Farben:] - 
		\end{description}

	\item{Level 4} 
		\begin{description}
			\item[Typ:] Einfügelevel
			\item[Anfangskonstellation:] \(x y z\)    
			\item[Endkonstellation:]  \(x z\)
			\item[gesperrte Farben:] x, y, z
		\end{description}

	\item{Level 5} 
		\begin{description}
			\item[Typ:] Einfärbelevel mit Schrittanzahl
			\item[Schrittanzahl:] 5
			\item[Anfangskonstellation:] \(\lambda o . o   \lambda o . o  \lambda o . o   \lambda o . o  \lambda o . o   \lambda o . o \) 
			\item[Endkonstellation:]  -
			\item[gesperrte Farben:] -
			\item[Beschreibung:] Viertes Tutoriallevel, das das System der Schrittanzahllevel erklärt.
		\end{description}

	\item{Level 6} 
		\begin{description}
			\item[Typ:] Multiple-Choice 
			\item[Anfangskonstellation:] \((\lambda x . x ) y\)    
			\item[Wahlmöglichkeiten:] \hfill
				\begin{itemize}
					\item[1.] \(y\)
					\item[2.] \(\lambda x . x \) 
					\item[3.] \(y y\)
				\end{itemize}
			\item[Endkonstellation:] \(y\)
			\item[Beschreibung:] Fünftes Tutoriallevel, in dem das System der Multiple-Choice Levels erklärt wird.
		\end{description}

	\item{Level 7} 
		\begin{description}
			\item[Typ:] Multiple-Choice 
			\item[Anfangskonstellation:] \((\lambda x . \lambda y . x ) z\)    
			\item[Wahlmöglichkeiten:] \hfill
				\begin{itemize}
					\item[1.] \( \lambda y . z\) 
					\item[2.] \( \lambda x . x \) 
					\item[3.] \( \lambda x . z\)
				\end{itemize}
			\item[Endkonstellation:]\( \lambda y . z\)
		\end{description}

	\item{Level 8} 
		\begin{description}
			\item[Typ:] Einfärbelevel mit Schrittanzahl
			\item[Schrittanzahl:] 10
			\item[Anfangskonstellation:] \(\lambda o . o o \)  \(\lambda o . o o \) 
			\item[Endkonstellation:]  -
			\item[gesperrte Farben:] -
		\end{description}

	\item{Level 9} 
		\begin{description}
			\item[Typ:] Multiple-Choice 
			\item[Anfangskonstellation:] \((\lambda x . \lambda y . x x ) u v\)    
			\item[Wahlmöglichkeiten:] \hfill
				\begin{itemize}
					\item[1.] \( u u\) 
					\item[2.] \( v v \) 
					\item[3.] \( \lambda y . u\)
				\end{itemize}
			\item[Endkonstellation:]\(u u\)
		\end{description}
	
	\item{Level 10} 
		\begin{description}
			\item[Typ:] Einfügelevel
			\item[Anfangskonstellation:] \(y x \)    
			\item[Endkonstellation:] \(y y x\)
			\item[gesperrte Farben:] y, x 
		\end{description}

	\item{Level 11} 
		\begin{description}
			\item[Typ:] Multiple-Choice 
			\item[Anfangskonstellation:] \((\lambda x . x ((\lambda y . y) (\lambda z . z ) ) \)    
			\item[Wahlmöglichkeiten:] \hfill
				\begin{itemize}
					\item[1.] \( \lambda y . y\) 
					\item[2.] \( z \) 
					\item[3.] \(\lambda z . z \)
				\end{itemize}
			\item[Endkonstellation:]\( \lambda z . z\)
		\end{description}


\end{itemize}
\section{Levelpaket 1}
\begin{itemize}

	\item{Level 0}
		\begin{description}
			\item[Typ:] Multiple-Choice 
			\item[Anfangskonstellation:] \((\lambda p .\lambda q . p q p ) (\lambda x . \lambda y . x ) (\lambda x . \lambda y . y )  \)    
			\item[Wahlmöglichkeiten:] \hfill
				\begin{itemize}
					\item[1.] \(\lambda x . \lambda y . x \) 
					\item[2.] \(\lambda x . \lambda y . y \)
					\item[3.] \(\lambda y . x \)
				\end{itemize}
			\item[Endkonstellation:]\(\lambda x . \lambda y . y \)
		\end{description}

	\item{Level 1}
		\begin{description} 
		\item[Typ:] Färbelevel 
		\item[Anfangskonstellation:] \(\lambda x . ( x \lambda y .  ( \lambda z . (z z) y )  )  o o o\)   
		\item[Endkonstellation:] \(a c c b\) 
		\item[gesperrte Farben:] -
		\item[Beschreibung:] komplexeres Färbelevel mit 3 benötigten Umfärbungen. 

		\end{description}
	
	\item{Level 2}
		\begin{description}
			\item[Typ:] Färbelevel
			\item[Anfangskonstellation:] \( \lambda x . ( o o ) \lambda y . ( o o ) o \)    
			\item[Endkonstellation:] \(z z z z\)
			\item[gesperrte Farben:] -
		\end{description}
			
\end{itemize}
