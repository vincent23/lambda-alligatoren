\section{Package: game.level}

\subsection{LevelControllerTest}
	\begin{description}
		\item[testSize] Testet ob der \emph{LevelController} die richtige Anzahl an Leveln lädt.
		\item[testLevel] Testet ob das erste Level der Liste auch das erste Level ist.
	\end{description}

\subsection{LeveLoadHelperTest}
	\begin{description}
		\item[testCase0] Überprüft das Json von Level 00x00.
		\item[testCase1] Überprüft das Json von Level 00x01.
		\item[testCase2] Überprüft das Json von Level 00x02.
		\item[testCase3] Überprüft das Json von Level 00x03.
		\item[testCase4] Überprüft das Json von Level 00x04.
		\item[testCase5] Überprüft das Json von Level 00x05.
		\item[testCase6] Überprüft das Json von Level 00x06.
		\item[testCase7] Überprüft das Json von Level 00x07.
		\item[testCase8] Überprüft das Json von Level 00x08.
		\item[testCase9] Überprüft das Json von Level 00x09.
		\item[testCase10] Überprüft das Json von Level 00x10.
		\item[testCase11] Überprüft das Json von Level 00x11.
		\item[testCase20] Überprüft das Json von Level 01x00.
		\item[testCase21] Überprüft das Json von Level 01x01.
		\item[testCase22] Überprüft das Json von Level 01x02.



	\end{description}

\subsection{LoadColorEditLevelFromJsonTest}
	\begin{description}
		\item[testCase0] Überprüft ob das Json von Level 00x00 richtig ausgelesen wird.
		\item[testCase1] Überprüft ob das Json von Level 00x01 richtig ausgelesen wird.
		\item[testCase5] Überprüft ob das Json von Level 00x05 richtig ausgelesen wird.
		\item[testCase8] Überprüft ob das Json von Level 00x08 richtig ausgelesen wird.
	\end{description}

\subsection{LoadPackageTest}
	\begin{description}
		\item[testLoading] Testet ob der \emph{LevelPackagesController} die richtige Anzahl an Levelpaketen lädt.
		\item[testLoadedValues] Testet ob die Werte korrekt aus den Json Dateien ausgelesen werden.
	\end{description}
