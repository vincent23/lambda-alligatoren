\section{Package: game}

\subsection{ColorTest}

\begin{description}
\item{testMaxColorsEqualsColorStringLength}
Testet, ob genug Farben vorhanden sind, um die maximale Anzahl an möglichen Farben abzudecken.
\end{description}

\subsection{SimulatorTest}

\begin{description}
\item[testOmega]
Testet die korrekte Auswertung eines Schrittes des \(\Omega\)-Terms.

\item[testLevel2]
Testet die korrekte Auswertung des zweiten Levels aus dem Entwurf.

\item[testLevel8]
Testet die korrekte Auswertung des achten Levels aus dem Entwurf.

\item[testLevel10]
Testet die korrekte Auswertung des zehnten Levels aus dem Entwurf.

\item[testLevel12]
Testet die korrekte Auswertung des zwölften Levels aus dem Entwurf.

\item[testTwoAlligators]
Testet die korrekte Auswertung eines Terms mit zwei Alligatoren.

\item[testTakeFirst]
Testet die korrekte Auswertung eines Terms, bei dem das erste von zwei Eiern übrig bleibt.

\item[testTakeSecond]
Testet die korrekte Auswertung eines Terms, bei dem das zweite von zwei Eiern übrig bleibt.

\item[testOldAlligator]
Testet die korrekte Auswertung eines Terms, der einen alten Alligator enthält.

\item[testColorRule]
Testet die korrekte Auswertung eines Terms, bei dem eine Umfärbung notwendig ist.

\item[testYCombinatorOneStep]
Testet die korrekte Auswertung eines Schrittes des Y-Combinators.

\item[testIncrementZero]
Testet die korrekte Auswertung einer Inkrementierung von 0 mit Church-Numeralen.

\item[testOnePlusOne]
Testet die korrekte Auswertung von \(1+ 1\) mit Church-Numeralen.

\item[testThreePlusFour]
Testet die korrekte Auswertung von \(3 + 4\) mit Church-Numeralen.

\end{description}
