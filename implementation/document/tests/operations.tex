\section{Package: game.board.operations}
\subsection{CollectBoundColorsTest}
	\begin{description}
		\item[testSimpleBoundColor] Überprüft an einem einfachen Beispiel, ob die Anzahl der Farben im \emph{Board} richtig
			berechnet wird und die richtigen Farben erkannt werden.
		\item[testRootBoundColor] Überprüft ob erkannt wird, dass das \emph{Board} selbst keine Farbe hat. 
		\item[testMultipleBoundColor] Überprüft, ob auch bei einem \emph{Board} mit mehreren Farben die Anzahl und Art der 
			Farben richtig erkannt wird.
		\item[testBoundColorsMultipleOccurences]Überprüft, ob Anzahl und Farbe der Farben richtig erkannt wird, wenn mehrere 
			Farben mehrmals im \emph{Board} vorkommen. 
	\end{description}

\subsection{CollectFreeColorsTest}
	\begin{description}
		\item[testSimpleFreeColor] Überprüft die Erkennung eines frei stehenden gefärbten Eis und dessen Farbe.
		\item[testSimpleNonFreeColor] Überprüft ob nicht alleine stehende Eier als solche erkannt werden.
		\item[testRootFreeColor]Überprüft, dass das \emph{Board} nicht als alleine stehendes Ei erkannt wird.
		\item[testFreeColorWithParent] Überprüft die Erkennung eines anders als sein Parent gefärbten Eis und dessen Farbe.
		\item[testMultipleFreeColors] Überprüft die Erkennung von mehreren anders als die jeweiligen Parents gefärbten Eiern 
			und deren Farbe.
		\item[testFreeColorMultipleOccurences]Überprüft die Erkennung von mehreren anders als die jeweiligen Parents gefärbten 
			Eiern und deren Farbe, auch wenn diese mehrmals auftritt.
	\end{description}

\subsection{CreateDepthMapTest}
	\begin{description}
		\item[testSimple] Überprüft an einem einfachen Beispiel ob für jedes \emph{BoardObject} erkannt wird in welcher Ebene 
			des Baumes es liegt.
	\end{description}

\subsection{CreateHeightMapTest}
	\begin{description}
		\item[testSimple] Überprüft an einem einfachen Beispiel ob für jedes \emph{BoardObject} erkannt wird wie hoch es ist.
		\item[testCase0] Überprüft ob die Höhe eines \emph{Boards} richtig berechnet wird.
	\end{description}

\subsection{CreateWidthMapTest}
	\begin{description}
		\item[testSimple] Überprüft an einem einfachen Beispiel ob für jedes \emph{BoardObject} erkannt wird wie breit es ist.
		\item[testCase0] Überprüft ob die Breite eines \emph{Boards} richtig berechnet wird.
	\end{description}

\subsection{ExchangeColorTest}
	\begin{description}
		\item[testSimple] Überprüft an einem einfachen Beispiel ob das umfärben \emph{BoardObjects}  richtig funktioniert.
	\end{description}

\subsection{FindEatingTest}
	\begin{description}
		\item[testSimple] Überprüft an einem einfachen Beispiel ob das Finden des \emph{Alligators}, der als nächstes fressen 				kann, funktioniert.
		\item[testPrecedence]Überprüft ob der richtige \emph{Alligator}, als derjenige der als nächstes fressen kann, 
			zurückgegeben wird an einem komplizierterem Beispiel.
	\end{description}
\subsection{FlattenTreeTest}
	\begin{description}
		\item[testSimple] Überprüft die Kompirimierung des \emph{Boards} als Liste.
	\end{description}
\subsection{GetParentHierachyTest}
	\begin{description}
		\item[testSimple] Überprüft an einem einfachen Beispiel ob für ein Kind die richtige Parent Hierachie zurückgegeben 
			wird.
	\end{description}
\subsection{ReplaceEggsTest}
	\begin{description}
		\item[testSimple] Überprüft an einem einfachen Beispiel ob das Ersetzten der Eier eines Alligators korrekt 
			funktioniert.
		\item[testSimpleRecolorFree]Überprüft an einem einfachen Beispiel ob das Ersetzten der Eier eines Alligators korrekt 
			funktioniert, wenn keine Umfärbungen nötig sind.
		\item[testSimpleRecolorBound]Überprüft an einem einfachen Beispiel ob das Ersetzten der Eier eines Alligators korrekt 
			funktioniert, wenn Umfärbungen nötig sind.
	\end{description}
