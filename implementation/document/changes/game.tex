\section{Package: game}

\subsection{Color}

\subsubsection{Hinzugefügte Methoden}
\begin{description}
\item[uncolored]
Methode, die die Farbe zurückgibt, die ungefärbten Objekten zugewiesen wird.

\item[getRepresentations]
Methode, die alle Farbrepräsentationen liefert.

\item[getRepresentation]
Methode zum Umwandeln einer Farbe im Spiel in eine tatsächlich darstellbare Farbe.

\item[equals]
Farben sind bei gleicher \emph{id} äquivalent, weshalb \emph{equals} überschrieben wurde.

\item[hashCode]
Siehe \emph{equals}.

\item[compareTo]
Siehe \emph{equals}.

\end{description}


\subsection{ColorController}

\subsubsection{Hinzugefügte Methoden}
\begin{description}
\item[getUncolored]
Liefert die Farbe, die ungefärbten Objekten zugewiesen wird.
\end{description}

\subsubsection{Modifizierte Methoden}
\begin{description}
\item[getRepresentation]
Umbenannt von \emph{getRepresantation}.

\end{description}


\subsection{EditLevelGameController}
Neu hinzugefügte Klasse, die einen spezielleren \emph{GameController} repräsentiert, der für \emph{EditLevel} genutzt wird.

\subsubsection{Hinzugefügte Methoden}
\begin{description}
\item[onAfterLoadProgress]
Callbackmethode aus \emph{GameController}.
Wird hier genutzt um das letzte \emph{Board} des Benutzers zu laden.

\item[onBeforeSaveProgress]
Callbackmethode aus \emph{GameController}.
Wird hier genutzt um das aktuelle \emph{Board} des Benutzers zu speichern.

\item[createColorController]
Callbackmethode aus \emph{GameController}.
Wird hier genutzt um dem \emph{ColorController} nutzbare sowie gesperret Farben mitzuteilen.

\item[onFinishedSimulation]
Callbackmethode aus \emph{GameController}.
Wird hier genutzt um die Lösung des Benutzers zu validieren.

\end{description}


\subsection{GameController}

\subsubsection{Hinzugefügte Methoden}

\subsubsection{Modifizierte Methoden}

\subsubsection{Entfernte Methoden}



\subsection{MultipleChoiceGameController}
Neu hinzugefügte Klasse, die einen spezielleren \emph{GameController} repräsentiert, der für \emph{MultipleChoiceLevel} genutzt wird.

\subsubsection{Hinzugefügte Methoden}
\begin{description}
\item[setSelection]
Methode, um die Auswahl des Benutzers zu setzen.
\item[createPlacementScreen]
Überschriebene Methode von \emph{GameController}, um für \emph{MultipleChoiceLevel} eine andere Darstellung zu erzeugen.
\end{description}


\subsection{Simulator}

\subsubsection{Hinzugefügte Methoden}

\subsubsection{Modifizierte Methoden}

\subsubsection{Entfernte Methoden}

