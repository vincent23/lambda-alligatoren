\section{Package: game}

\subsection{Color}

\subsubsection{Hinzugefügte Methoden}
\begin{description}
\item[uncolored]
Methode, die die Farbe zurückgibt, die ungefärbten Objekten zugewiesen wird.

\item[getRepresentations]
Methode, die alle Farbrepräsentationen liefert.

\item[getRepresentation]
Methode zum Umwandeln einer Farbe im Spiel in eine tatsächlich darstellbare Farbe.

\item[equals]
Farben sind bei gleicher \emph{id} äquivalent, weshalb \emph{equals} überschrieben wurde.

\item[hashCode]
Siehe \emph{equals}.

\item[compareTo]
Siehe \emph{equals}.

\end{description}


\subsection{ColorController}

\subsubsection{Hinzugefügte Methoden}
\begin{description}
\item[getUncolored]
Liefert die Farbe, die ungefärbten Objekten zugewiesen wird.
\end{description}

\subsubsection{Modifizierte Methoden}
\begin{description}
\item[getRepresentation]
Umbenannt von \emph{getRepresantation}.

\end{description}


\subsection{EditLevelGameController}
Neu hinzugefügte Klasse, die einen spezielleren \emph{GameController} repräsentiert, der für \emph{EditLevel} genutzt wird.

\subsubsection{Hinzugefügte Methoden}
\begin{description}
\item[onAfterLoadProgress]
Callbackmethode aus \emph{GameController}.
Wird hier genutzt um das letzte \emph{Board} des Benutzers zu laden.

\item[onBeforeSaveProgress]
Callbackmethode aus \emph{GameController}.
Wird hier genutzt um das aktuelle \emph{Board} des Benutzers zu speichern.

\item[createColorController]
Callbackmethode aus \emph{GameController}.
Wird hier genutzt um dem \emph{ColorController} nutzbare sowie gesperret Farben mitzuteilen.

\item[onFinishedSimulation]
Callbackmethode aus \emph{GameController}.
Wird hier genutzt um die Lösung des Benutzers zu validieren.

\end{description}


\subsection{GameController}

\subsubsection{Hinzugefügte Methoden}
\begin{description}
\item[setupColorController]
Wird zur Re-Initialisierung des \emph{ColorController} aufgerufen.

\item[createColorController]
Methode, um die Erstellung des \emph{ColorController} durch Unterklassen anpassen zu können.

\item[getColorController]
Methode, um den \emph{ColorController} zu erhalten.

\item[enterPlacement]
Methode, die für einen Wechsel in den Platziermodus sorgt.
Sichtbarkeit von private auf public geändert.

\item[enterSimulation]
Methode, die für einen Wechsel in den Simulationsmodus sorgt.
Sichtbarkeit von private auf public geändert.

\item[onFinishedSimulation]
Callbackmethode zum Überschreiben Unterklassen, die nach Abschluss der Simulation aufgerufen wird.

\item[registerSimulationBoardEventListener]
Siehe \emph{registerBoardEventListener} unter "`Entfernte Methoden"'.

\item[unregisterSimulationBoardEventListener]
Siehe \emph{unregisterBoardEventListener} unter "`Entfernte Methoden"'.

\item[registerPlacementBoardEventListener]
Siehe \emph{registerBoardEventListener} unter "`Entfernte Methoden"'.

\item[unregisterPlacementBoardEventListener]
Siehe \emph{unregisterBoardEventListener} unter "`Entfernte Methoden"'.

\item[getPlacementBoardEventListener]
Gibt den \emph{BoardEventMessenger} für den Platziermodus zurück.

\item[onObjectPlaced]
Neues Event, siehe \emph{BoardEventListener}.

\item[onObjectRemoved]
Neues Event, siehe \emph{BoardEventListener}.

\item[onObjectMoved]
Neues Event, siehe \emph{BoardEventListener}.

\item[onAge]
Neues Event, siehe \emph{BoardEventListener}.

\item[evaluateStep]
Führt einen Schritt der Simulation aus und prüft, ob diese abgeschlossen ist.

\item[isInSimulationMode]
Gibt zurück, ob sich der Spieler im Simulationsmodus befindet.

\item[canUndo]
Gibt zurück, ob ein Schritt in der Simulation rückgängig gemacht werden kann.

\item[undo]
Macht einen Schritt in der Simulation rückgängig.

\item[reset]
Setzt das durch den Benutzer bearbeitete \emph{Board} auf den ursprünglichen Zustand zurück.

\item[getShownBoard]
Gibt das aktuell angezeigte \emph{Board} zurück.

\item[getElapsedTime]
Gibt die bisher im Level verbrachte Zeit zurück.

\item[getTimeStamp]
Gibt den Zeitpunkt des Levelstarts zurück.

\item[setElapsedTime]
Setzt die bisher im Level verbrachte Zeit.

\item[updateTime]
Aktualisiert die verstrichene Zeit seit dem Levelstart.

\item[setTimeStamp]
Setzt den Zeitpunkt des Levelstarts.

\item[getLevel]
Gibt das aktuelle Level zurück.

\item[createPlacementScreen]
Methode zum Überschreiben durch Unterklassen, um eine andere Darstellung des Platziermodus zu erreichen.

\item[isSolved]
Gibt zurück, ob das aktuell geladene Level gelöst ist.

\item[getProgress]
Gibt den \emph{LevelProgress} des aktuellen Levels und Benutzers zurück.

\item[setUserBoard]
Setzt das durch den Benutzer bearbeitete \emph{Board}.

\item[onBeforeSaveProgress]
Methode zum Überschreiben durch Unterklassen, die vor dem Speichern des \emph{LevelProgress} aufgerufen wird.

\item[onAfterLoadProgress]
Methode zum Überschreiben durch Unterklassen, die nach dem Laden des \emph{LevelProgress} aufgerufen wird.

\item[getUserBoard]
Gibt das durch den Benutzer bearbeitete \emph{Board} zurück.

\item[onUsedHint]
Muss aufgerufen werden, um die Benutzung eines Hinweises zu registrieren.

\item[getSimulator]
Gibt den aktuell genutzten \emph{Simulator} zurück.
\end{description}

\subsubsection{Modifizierte Methoden}
\begin{description}
\item[Konstruktor]
Referenz auf \emph{AlligatorApp} als zusätzliches Argument eingefügt.

\item[onObjectRecolored]
Geändertes Interface, siehe \emph{ObjectRecoloredListener}.

\item[onEat]
Geändertes Interface, siehe \emph{EatEventListener}.

\item[onAgedAlligatorVanishes]
Geändertes Interface, siehe \emph{AgedAlligatorVanishesListener}.

\item[onHatched]
Umbenannt von \emph{onReplaceEgg}, siehe \emph{Listener}.
\end{description}

\subsubsection{Entfernte Methoden}
\begin{description}
\item[onCompletedLevel]
Sichtbarkeit von public auf private geändert.

\item[registerBoardEventListener]
Aufteilung in \emph{registerPlacementBoardEventListener} und \emph{registerSimulationBoardEventListener}.

\item[unregisterBoardEventListener]
Aufteilung in \emph{unregisterPlacementBoardEventListener} und \emph{unregisterSimulationBoardEventListener}.

\end{description}


\subsection{MultipleChoiceGameController}
Neu hinzugefügte Klasse, die einen spezielleren \emph{GameController} repräsentiert, der für \emph{MultipleChoiceLevel} genutzt wird.

\subsubsection{Hinzugefügte Methoden}
\begin{description}
\item[setSelection]
Methode, um die Auswahl des Benutzers zu setzen.
\item[createPlacementScreen]
Überschriebene Methode von \emph{GameController}, um für \emph{MultipleChoiceLevel} eine andere Darstellung zu erzeugen.
\end{description}


\subsection{Simulator}

\subsubsection{Hinzugefügte Methoden}
\begin{description}
\item[canUndo]
Methode, die zurückgibt, ob aktuell ein Schritt rückgängig gemacht werden kann.

\item[getSteps]
Methode, die die aktuelle Anzahl an ausgewerteten Schritten zurückgibt.

\item[getCurrentBoard]
Methode, die das aktuell verwendete \emph{Board} zurückgibt.

\end{description}

\subsubsection{Modifizierte Methoden}
\begin{description}
\item[evaluate]
Return-Wert von \emph{Board} auf \emph{boolean} geändert.
Kann außerdem \emph{ColorOverflowException} und \emph{AlligatorOverflowException} auslösen.

\end{description}
