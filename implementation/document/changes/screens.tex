\section{Package: ui.screens}

\subsection{AbstractScreen}

\subsubsection{Hinzugefügte Methoden}
\begin{description}
\item[getViewportWidth und getViewportHeight]
Getter für die Breite und Höhe des Viewports (des tatsächlich angezeigten Bereiches).

\item[initializeWidgets]
In dieser Methode sollten alle Widgets, die im TableLayout der Ansicht verarbeitet werden initialisiert werden. Dass dies nicht einfach im Konstruktor geschieht, hat den Grund, dass das Laden aller Grafiken am Ende der AlligatorApp.create() Methode gebündelt wird, und erst danach Widgets erstellt werden können.
\item[areWidgetsInitialized]
Um zu verhindern, dass initializeWidgets() bei jedem Anzeigen einer Ansicht neu gestartet wird, was auf die Performanz schlagen würde, signalisiert diese Methode, ob die Initialisierung schon erfolgt ist.
\item[showLogicalPredecessor]
Zeigt den Screen, der auf einer logischen Ebene direkt über dem aktuell Angezeigten liegt. Per Default ist dies derselbe, der beim Drücken des "'Back"'-Buttons gezeigt würde. Diese Methode muss überschrieben werden, wenn es einen genauer definierten Vorgänger gibt.

\end{description}


\subsection{AchievementScreen}

\subsubsection{Hinzugefügte Methoden}
\begin{description}
\item[initializeWidgets]
Initialisierung von Widgets.

\end{description}

\subsubsection{Modifizierte Methoden}
\begin{description}
\item[Konstruktor]
Nimmt nicht mehr den einen \emph{GameController} als Argument entgegen.

\end{description}



\subsection{CreditsScreen}
Neuer Screen, der die Namen der App-Entwickler zeigt sowie einen Hinweis auf Bret Victor's Idee und seine Webseite.

\subsubsection{Hinzugefügte Methoden}
\begin{description}
\item[initializeWidgets]
Initialisierung von Widgets.

\end{description}



\subsection{LevelPackagesScreen}

\subsubsection{Hinzugefügte Methoden}
\begin{description}
\item[initilizeWidgets]
Initialisierung von Widgets.
\item[showLogicalPredecessor]
Zeigt den Screen, der auf einer logischen Ebene direkt über dem aktuell Angezeigten liegt: in diesem Fall das Hauptmenü.

\end{description}

\subsubsection{Modifizierte Methoden}
\begin{description}
\item[Konstruktor]
Nimmt nicht mehr den einen \emph{LevelPackageController} als Argument entgegen.

\end{description}


\subsection{LevelsOverviewScreen}
\subsubsection{Hinzugefügte Methoden}
\begin{description}
\item[showLogicalPredecessor]
Zeigt den Screen, der auf einer logischen Ebene direkt über dem aktuell Angezeigten liegt: in diesem Fall die Levelpaketübersicht.

\end{description}


\subsection{LoadingScreen}
Neuer Screen, der während Ladephasen des Spiels angezeigt wird. Er ersetzt außerdem den Splashscreen.

\subsubsection{Hinzugefügte Methoden}
\begin{description}
\item[render]
Zeigt den aktuellen Ladefortschritt an.


\end{description}



\subsection{MainMenuScreen}
Klasse implementiert jetzt noch zusätzlich \emph{ProfileChangeListener}.

\subsubsection{Hinzugefügte Methoden}
\begin{description}
\item[initilizeWidgets]
Initialisierung von Widgets.
\item[onProfileChange]
Methode wird aufgerufen, wenn das aktuelle Profil geändert wird.

\end{description}

\subsubsection{Modifizierte Methoden}
\begin{description}
\item[Konstruktor]
Nimmt nur noch eine \emph{AlligatorApp} als Argument entgegen.

\end{description}


\subsection{MultipleChoiceScreen}
Klasse implementiert jetzt \emph{SettingsChangeListener}.

\subsubsection{Hinzugefügte Methoden}
\begin{description}

\item[onShow]
Methode wird beim Zeigen des Screens aufgerufen.

\item[hide]
Methode wird aufgerufen, wenn der Screen ausgeblendet wird.

\item[render]
Methode wird zum Rendern des Screens aufgerufen.

\item[onSettingsChange]
Methode wird aufgerufen wenn die Einstellungen geändert werden.

\end{description}

\subsubsection{Modifizierte Methoden}
\begin{description}
\item[Konstruktor]
Nimmt statt einem \emph{GameController} einen \emph{MultipleChoiceGameController} als Argument entgegen.

\end{description}

\subsection{PlacementModeScreen}
Klasse implementiert jetzt \emph{SettingsChangeListener}.

\subsubsection{Hinzugefügte Methoden}
\begin{description}

\item[onShow]
Methode wird beim Zeigen des Screens aufgerufen.

\item[hide]
Methode wird aufgerufen, wenn der Screen ausgeblendet wird.

\item[render]
Methode wird zum Rendern des Screens aufgerufen.

\item[onSettingsChange]
Methode wird aufgerufen wenn die Einstellungen geändert werden.
\end{description}

\subsubsection{Modifizierte Methoden}
\begin{description}
\item[Konstruktor]
Nimmt statt einem \emph{GameController} einen \emph{MultipleChoiceGameController} als Argument entgegen.

\end{description}



\subsection{ProfileSetAvatarScreen}

\subsubsection{Hinzugefügte Methoden}
\begin{description}
\item[initilizeWidgets]
Initialisierung von Widgets.
\item[onShow]
Methode wird beim Zeigen des Screens aufgerufen.
\item[setProfileName]
Setter für den Profilnamen.
\item[setIsInEditMode]
Teilt dem Screen mit, dass er zur Profilbearbeitung aufgerufen wurde, was z.B. bewirkt, dass er bei Bestätigung der Namenseingabe nicht zur Avatarauswahl schaltet, sondern das Profil entsprechend aktualisiert.
\end{description}

\subsubsection{Modifizierte Methoden}
\begin{description}
\item[Konstruktor]
Nimmt nur noch eine \emph{AlligatorApp} als Argument entgegen.

\end{description}



\subsection{ProfileSetNameScreen}

\subsubsection{Hinzugefügte Methoden}
\begin{description}
\item[initilizeWidgets]
Initialisierung von Widgets.
\item[onShow]
Methode wird beim Zeigen des Screens aufgerufen.
\item[setIsInEditMode]
Teilt dem Screen mit, dass er zur Profilbearbeitung aufgerufen wurde, was z.B. bewirkt, dass er bei Bestätigung kein neues Profil erstellt, sondern das aktuelle bearbeitet.
\item[showBackButton]
Legt fest ob der Button, der zurück zum vorigen Screen navigiert, angezeigt wird oder nicht.
\end{description}

\subsubsection{Modifizierte Methoden}
\begin{description}
\item[Konstruktor]
Nimmt nur noch eine \emph{AlligatorApp} als Argument entgegen.

\end{description}




\subsection{QuitGameOverlay}
Neues Overlay, dass beim Verlassen des Spiels den Spieler noch einmal fragt ob er wirklich das Spiel verlassen will.

\subsubsection{Hinzugefügte Methoden}
\begin{description}
\item[initilizeWidgets]
Initialisierung von Widgets.
\item[onShow]
Methode wird beim Zeigen des Screens aufgerufen.
\item[render]
Methode wird zum Rendern des Overlays aufgerufen.
\end{description}


\subsection{SelectProfileScreen}
Klasse implementiert jetzt ProfileChangeListener.
\subsubsection{Hinzugefügte Methoden}
\begin{description}
\item[initilizeWidgets]
Initialisierung von Widgets.
\item[onProfileChange]
Methode wird aufgerufen wenn das aktuelle Profil geändert wird.

\end{description}

\subsubsection{Modifizierte Methoden}
\begin{description}
\item[Konstruktor]
Nimmt nur noch eine \emph{AlligatorApp} als Argument entgegen.

\end{description}


\subsection{SettingsScreen}
Klasse implementiert jetzt ProfileChangeListener.
\subsubsection{Hinzugefügte Methoden}
\begin{description}
\item[initilizeWidgets]
Initialisierung von Widgets.
\item[onProfileChange]
Methode wird aufgerufen wenn das aktuelle Profil geändert wird.


\end{description}

\subsubsection{Modifizierte Methoden}
\begin{description}
\item[Konstruktor]
Nimmt nur noch eine \emph{AlligatorApp} als Argument entgegen.

\end{description}


\subsection{SimulationModeScreen}
Klasse implementiert jetzt SettingsChangeListener.
\subsubsection{Hinzugefügte Methoden}
\begin{description}
\item[initilizeWidgets]
Initialisierung von Widgets.
\item[onProfileChange]
Methode wird aufgerufen wenn das aktuelle Profil geändert wird.

\item[render]
Zeichnet alle erforderlichen Änderungen
\item[onShow]
Methode wird beim Zeigen des Screens aufgerufen.
\item[hide]
Methode wird aufgerufen wenn der Screen ausgeblendet wird.
\item[onSettingsChange]
Methode wird aufgerufen wenn die Einstellungen geändert werden.


\end{description}

\subsubsection{Modifizierte Methoden}
\begin{description}
\item[Konstruktor]
Nimmt nur noch eine \emph{AlligatorApp} als Argument entgegen.

\end{description}

\subsection{SplashScreen}
Funktionalität wurde mit den LoadingScreen zusammengeführt. Die Klasse selbst wurde entfernt.



\subsection{StatisticScreen}
Klasse implementiert jetzt ProfileChangeListener.
\subsubsection{Hinzugefügte Methoden}
\begin{description}
\item[initilizeWidgets]
Initialisierung von Widgets.
\item[onProfileChange]
Methode wird aufgerufen wenn das aktuelle Profil geändert wird.

\item[onShow]
Methode wird beim Zeigen des Screens aufgerufen.



\end{description}

\subsubsection{Modifizierte Methoden}
\begin{description}
\item[Konstruktor]
Nimmt nur noch eine \emph{AlligatorApp} als Argument entgegen.

\end{description}

