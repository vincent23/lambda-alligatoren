\section{Package: ui.screens}

subsection{AbstractScreen}

\subsubsection{Hinzugefügte Methoden}
\begin{description}
\item[getViewportWidth() und getViewportHeight()]
Getter für die Breite und Höhe des Viewports.

\item[initilizeWidgets() und areWidgetsInizilized()]
Initialisierung und Abfrage der selben von Widgets.

\end{description}


\subsection{AchievementScreen}

\subsubsection{Hinzugefügte Methoden}
\begin{description}
\item[initilizeWidgets()]
Initialisierung von Widgets.

\end{description}

\subsubsection{Modifizierte Methoden}
\begin{description}
\item[Konstruktor]
Nimmt nicht mehr den einen \emph{GameController} als Argument entgegen.

\end{description}



\subsection{CreditsScreen}
Neuer Screen, der die Namen der App-Entwickler zeigt sowie einen Hinweis auf Brett Victor's Idee und seine Webseite.

\subsubsection{Hinzugefügte Methoden}
\begin{description}
\item[initilizeWidgets()]
Initialisierung und Abfrage der selben von Widgets.

\end{description}



\subsection{LevelPackagesScreen}

\subsubsection{Hinzugefügte Methoden}
\begin{description}
\item[initilizeWidgets()]
Initialisierung von Widgets.
\item[showLogicalPredecessor()]
Zeigt den Screen, der auf einer logischen Ebene direkt über dem aktuell Angezeigten liegt.

\end{description}

\subsubsection{Modifizierte Methoden}
\begin{description}
\item[Konstruktor]
Nimmt nicht mehr den einen \emph{LevelPackageController} als Argument entgegen.

\end{description}


\subsection{LevelsOverviewScreen}
\subsubsection{Hinzugefügte Methoden}
\begin{description}
\item[showLogicalPredecessor()]
Zeigt den Screen, der auf einer logischen Ebene direkt über dem aktuell Angezeigten liegt.

\end{description}


\subsection{LoadingScreen}
Neuer Screen, der während Ladephasen des Spiels angezeigt wird.

\subsubsection{Hinzugefügte Methoden}
\begin{description}
\item[render(float delta)]
Zeigt den aktuellen Ladefortschritt an.


\end{description}



\subsection{MainMenuScreen}
Klasse implementiert jetzt noch zusätzlich \emph{ProfileChangeListener}.

\subsubsection{Hinzugefügte Methoden}
\begin{description}
\item[initilizeWidgets()]
Initialisierung von Widgets.
\item[onProfileChange]
Methode wird beim Ändern des aktiven Profils aufgerufen.
\end{description}

\subsubsection{Modifizierte Methoden}
\begin{description}
\item[Konstruktor]
Nimmt nur noch eine \emph{AlligatorApp} als Argument entgegen.

\end{description}


\subsection{MultipleChoiceScreen}
Klasse implementiert jetzt \emph{SettingsChangeListener}.

\subsubsection{Hinzugefügte Methoden}
\begin{description}

\item[onShow()]
Methode wird beim Zeigen des Screens aufgerufen.

\item[hide()]
Methode wird aufgerufen um den Screen verschwinden zu lassen.

\item[render(float delta)]
Methode wird zum Rendern des Screens aufgerufen.

\item[onSettingsChange(Profile profile)]
Methode wird beim Ändern der aktuellen Einstellungen aufgerufen.
\end{description}

\subsubsection{Modifizierte Methoden}
\begin{description}
\item[Konstruktor]
Nimmt statt einem \emph{GameController} einen \emph{MultipleChoiceGameController} als Argument entgegen.

\end{description}

\subsection{PlacementModeScreen}
Klasse implementiert jetzt \emph{SettingsChangeListener}.

\subsubsection{Hinzugefügte Methoden}
\begin{description}

\item[onShow()]
Methode wird beim Zeigen des Screens aufgerufen.

\item[hide()]
Methode wird aufgerufen um den Screen verschwinden zu lassen.

\item[render(float delta)]
Methode wird zum Rendern des Screens aufgerufen.

\item[onSettingsChange(Profile profile)]
Methode wird beim Ändern der aktuellen Einstellungen aufgerufen.
\end{description}

\subsubsection{Modifizierte Methoden}
\begin{description}
\item[Konstruktor]
Nimmt statt einem \emph{GameController} einen \emph{MultipleChoiceGameController} als Argument entgegen.

\end{description}



\subsection{ProfileSetAvatarScreen}

\subsubsection{Hinzugefügte Methoden}
\begin{description}
\item[initilizeWidgets()]
Initialisierung von Widgets.
\item[onShow()]
Methode wird beim Zeigen des Screens aufgerufen.
\item[setProfileName(String profileName)]
Setter für den Profilnamen.
\item[setIsInEditMode(boolean isInEditMode)]
Setter für die Variable ob es im Bearbeitungsmodus ist.
\end{description}

\subsubsection{Modifizierte Methoden}
\begin{description}
\item[Konstruktor]
Nimmt nur noch eine \emph{AlligatorApp} als Argument entgegen.

\end{description}



\subsection{ProfileSetNameScreen}

\subsubsection{Hinzugefügte Methoden}
\begin{description}
\item[initilizeWidgets()]
Initialisierung von Widgets.
\item[onShow()]
Methode wird beim Zeigen des Screens aufgerufen.
\item[setIsInEditMode(boolean isInEditMode)]
Setter für die Variable ob es im Bearbeitungsmodus ist.
\item[showBackButton(boolean showBackButton)]
Legt fest ob der Zurückknopf angezeigt wird oder nicht.
\end{description}

\subsubsection{Modifizierte Methoden}
\begin{description}
\item[Konstruktor]
Nimmt nur noch eine \emph{AlligatorApp} als Argument entgegen.

\end{description}




\subsection{QuitGameOverlay}
Neues Overlay, dass beim Verlassen des Spiels den Spieler noch einmal fragt ob er wirklich das Spiel verlassen will.

\subsubsection{Hinzugefügte Methoden}
\begin{description}
\item[initilizeWidgets()]
Initialisierung von Widgets.
\item[onShow()]
Methode wird beim Zeigen des Screens aufgerufen.
\item[render(float delta)]
Methode wird zum Rendern des Overlays aufgerufen.
\item[resize(int width, int height)]
%TODO: herrausfinden was das eigentlich tut.
\item[dispose()]
%TODO: herrausfinden was das eigentlich tut.
\end{description}


\subsection{SelectProfileScreen}
Klasse implementiert jetzt ProfileChangeListener.
\subsubsection{Hinzugefügte Methoden}
\begin{description}
\item[initilizeWidgets()]
Initialisierung von Widgets.
\item[onProfileChange]
Methode wird beim Ändern des aktiven Profils aufgerufen.

\end{description}

\subsubsection{Modifizierte Methoden}
\begin{description}
\item[Konstruktor]
Nimmt nur noch eine \emph{AlligatorApp} als Argument entgegen.

\end{description}


\subsection{SettingsScreen}
Klasse implementiert jetzt ProfileChangeListener.
\subsubsection{Hinzugefügte Methoden}
\begin{description}
\item[initilizeWidgets()]
Initialisierung von Widgets.
\item[onProfileChange]
Methode wird beim Ändern des aktiven Profils aufgerufen.

\end{description}

\subsubsection{Modifizierte Methoden}
\begin{description}
\item[Konstruktor]
Nimmt nur noch eine \emph{AlligatorApp} als Argument entgegen.

\end{description}


\subsection{SettingsScreen}
Klasse implementiert jetzt SettingsChangeListener.
\subsubsection{Hinzugefügte Methoden}
\begin{description}
\item[initilizeWidgets()]
Initialisierung von Widgets.
\item[onProfileChange]
Methode wird beim Ändern des aktiven Profils aufgerufen.
\item[render]
Zeigt den aktuellen Ladefortschritt an.
\item[onShow()]
Methode wird beim Zeigen des Screens aufgerufen.
\item[hide()]
Methode wird aufgerufen um den Screen verschwinden zu lassen.
\item[onSettingsChange(Profile profile)]
Methode wird beim Ändern der aktuellen Einstellungen aufgerufen.

\end{description}

\subsubsection{Modifizierte Methoden}
\begin{description}
\item[Konstruktor]
Nimmt nur noch eine \emph{AlligatorApp} als Argument entgegen.

\end{description}

\subsection{SplashScreen}
Funktionalität wurde mit den LoadingScreen zusammengeführt. Die Klasse selbst wurde entfernt.



\subsection{StatisticScreen}
Klasse implementiert jetzt ProfileChangeListener.
\subsubsection{Hinzugefügte Methoden}
\begin{description}
\item[initilizeWidgets()]
Initialisierung von Widgets.
\item[onProfileChange]
Methode wird beim Ändern des aktiven Profils aufgerufen.
\item[onShow]
Methode wird beim Zeigen des Screens aufgerufen.



\end{description}

\subsubsection{Modifizierte Methoden}
\begin{description}
\item[Konstruktor]
Nimmt nur noch eine \emph{AlligatorApp} als Argument entgegen.

\end{description}

