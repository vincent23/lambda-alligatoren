\section{Package: game.board.operations.validation (ehemals game.visitor)}
\subsection{AbstractBoardError}
	Oberklasse aller \emph{BoardError}. Implementiert das \emph{BoardError}-Interface.
	\subsubsection{Hinzugefügte Methoden}
	\begin{description}
		\item[Konstruktor] Es wurde ein Konstruktor hinzugefügt, der ein \emph{BoardObject} als Grund für den auftretenden 
			Fehler übergeben bekommt.
		\item[getCause]
			Gibt den Grund, also das \emph{BoardObject} zurück, das den Error verursacht hat.
	\end{description}

\subsection{AbstractBoardValidator}

\subsection{AgedAlligatorChildlessError}
	\emph{BoardError}, der auftritt, wenn ein \emph{AgedAlligator} keine Kinder hat.
	\subsubsection{Hinzugefügte Methoden}
	\begin{description}
		\item[Konstruktor] Es wurde ein Konstruktor hinzugefügt, der einen \emph{AgedAlligator} als Grund für den auftretenden 
			Fehler übergeben bekommt.
		\item[haveDispatched]
			Behandelt den Error auf entsprechende Weise mit Hilfe eines \emph{BoardErrorDispatcher}.
	\end{description}

\subsection{BoardError}
	Interface, dass alle \emph{BoardError} implementieren. \emph{BoardError} werden benutzt um bei einem invaliden
	 \emph{Board} eine entsprechende, die Ursache des Fehlers enthaltene Fehlermeldung zu erhalten.
	\subsubsection{Hinzugefügte Methoden}
	\begin{description}
		\item[getCause]
			Gibt den Grund, also das \emph{BoardObject} zurück, das den Error verursacht hat.
		\item[haveDispatched]
			Behandelt den Error auf entsprechende Weise mit Hilfe eines \emph{BoardErrorDispatcher}.
	\end{description}

\subsection{BoardErrorDispatcher}
	Interface in dem Methoden definiert werden um die unterschiedlichen Arten von \emph{BoardErrors} zu behandeln.
	\subsubsection{Hinzugefügte Methoden}
	\begin{description}
		\item[dispatch]
			Überladene Methode zur Behandlung unterschiedlicher Arten von \emph{BoardError}.
	\end{description}

\subsection{BoardErrorType}
	Enum in dem die verschiedenen \emph{BoardError}-Typen definiert werden.
	\subsubsection{Hinzugefügte Methoden}
	\begin{description}
		\item[all]
			Gibt ein Array mit allen in diesem Enum definierten \emph{BoardError}-Typen zurück.
	\end{description}

\subsection{ColoredAlliagtorChildlessError}
	\emph{BoardError}, der auftritt, wenn ein \emph{ColoredAlligator} keine Kinder hat.
	\subsubsection{Hinzugefügte Methoden}
	\begin{description}
		\item[Konstruktor] Es wurde ein Konstruktor hinzugefügt, der einen \emph{ColoredAlligator} als Grund für den 
			auftretenden Fehler übergeben bekommt.
		\item[haveDispatched]
			Behandelt den Error auf entsprechende Weise mit Hilfe eines \emph{BoardErrorDispatcher}.
	\end{description}

\subsection{EmptyBoardError}
	\emph{BoardError}, der auftritt, wenn ein \emph{Board} leer ist.
	\subsubsection{Hinzugefügte Methoden}
	\begin{description}
		\item[Konstruktor] Es wurde ein Konstruktor hinzugefügt, der ein \emph{Board} als Grund für den auftretenden 
			Fehler übergeben bekommt.
		\item[haveDispatched]
			Behandelt den Error auf entsprechende Weise mit Hilfe eines \emph{BoardErrorDispatcher}.
	\end{description}

\subsection{FindBoardErrors}

\subsection{ObjectUncoloredError}
	\emph{BoardError}, der auftritt, wenn ein \emph{ColoredAlligator} oder ein \emph{Egg} nicht eingefärbt ist.
	\subsubsection{Hinzugefügte Methoden}
	\begin{description}
		\item[Konstruktor] Es wurde ein Konstruktor hinzugefügt, der ein \emph{InternalBoardObject} als Grund für den 
			auftretenden Fehler übergeben bekommt.
		\item[haveDispatched]
			Behandelt den Error auf entsprechende Weise mit Hilfe eines \emph{BoardErrorDispatcher}.
	\end{description}

\subsection{ValidateConstellation}
