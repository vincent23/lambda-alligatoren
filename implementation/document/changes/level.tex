\section{Package: game.level}

\subsection{ColorEditLevel}
	Diese Klasse erbt nun von \emph{EditLevel}.

	\subsubsection{Modifizierte Methoden}
		\begin{description}
			\item[Konstruktor] Dem Konstruktor wurden alle Argumente hinzugefügt, die ein Level beschreiben.
		\end{description}


\subsection{EditLevel}
	Da die Klassen \emph{ColorEditLevel} und \emph{TermEditLevel} viele gemeinsame Methoden und in vielen Fällen 
	eine gleiche Behandlung benötigen, hat es sich angeboten eine gemeinsame Oberklasse einzufügen. 

	\subsubsection{Hinzugefügte Methoden}
		\begin{description}
			\item[Konstruktor] Dem Konstruktor wurden alle Argumente hinzugefügt, die ein Level beschreiben.
			\item[getUserColor] Gibt ein Array von Farben zurück, die der Benutzer für dieses Level verwenden kann.
			\item[getBlockedColor] Gibt ein Array von Farben zurück, die der Benutzer in diesem Level nicnt benutzen 
				darf
		\end{description}


\subsection{InvalidJsonException}
	Für die einfache Behandlung von Fehlern, die in den Json Dateien der Level auftreten können, hat es sich 
	angeboten, eine eigene Exception für diese Fehlerursache einzufügen.   

\subsection{Level}

	\subsubsection{Hinzugefügte Methoden}
		\begin{description}
			\item[getAbortSimulationAfter]
			\item[getUnlocked]
			\item[setUnlocked]
			\item[isLevelSolved] Gibt zurück ob das übergebene \emph{Board} und Schrittanzahl 
				das Level löst oder nicht.
			\item[isSolveable] Gibt zurück ob die festgelegte maximale Simulationsschrittzahl bereits 
				überschritten wurde.
			\item[isSolved]
			\item[setSolvedTrue]
			\item[createGameController] Methode, die einen GameController für dieses Level erstellt und zurückgibt.
			\item[getLevelId] Methode, die eine Konkatenation von Levelpackage Index und Level Index zurückgibt.
			\item[getShowObjectBar] Methode, die zurückgibt ob im Plaziermodus für dieses Level die \emph{ObjectBar} 
				angezeigt werden soll oder nicht.
		\end{description}
	\subsubsection{Modifizierte Methoden}
		\begin{description}
			\item[Konstruktor] Dem Konstruktor wurden alle Argumente hinzugefügt, die ein Level beschreiben.
		\end{description}

\subsection{LevelController}

	\subsubsection{Modifizierte Methoden}
		\begin{description}
			\item[Konstruktor] Der Konstruktor wurde um einige Argumente ergänzt.
		\end{description}
	\subsubsection{Entfernte Methoden}
		\begin{description}
			\item[Konstruktor] Überflüssige Konstruktoren wurden entfernt.
		\end{description}


\subsection{LevelPackage}

	\subsubsection{Hinzugefügte Methoden}
		\begin{description}
			\item[getAnimation]
			\item[getDesign]
		\end{description}
	\subsubsection{Modifizierte Methoden}
		\begin{description}
			\item[Konstruktor] Der Konstruktor wurde um einige Argumente ergänzt.
		\end{description}
	\subsubsection{Entfernte Methoden}
		\begin{description}
			\item[Konstruktor] Überflüssige Konstruktoren wurden entfernt.
		\end{description}



\subsection{LevelPackagesController}

	\subsubsection{Hinzugefügte Methoden}
		\begin{description}
			\item[getLevelController] Methode, die zu einem gegebenen \emph{LevelPackage} Index einen
			 \emph{LevelController} zurückgibt. 
			\item[getLevelPackages] Methode, die eine Liste aller vorhandenen Levelpakete zurückgibt.
		\end{description}


\subsection{MultipleChoiceLevel}

	\subsubsection{Modifizierte Methoden}
		\begin{description}
			\item[Konstruktor] Der Konstruktor wurde um einige Argumente ergänzt.
		\end{description}



\subsection{TermEditLevel}
	Diese Klasse erbt nun von \emph{EditLevel}.

	\subsubsection{Modifizierte Methoden}
		\begin{description}
			\item[Konstruktor] Der Konstruktor wurde um einige Argumente ergänzt.
		\end{description}

