\section{Package: game.event}
Die folgenden Listener wurden zusätzlich zum Entwurf in das "`event"' Paket aufgenommen:
\begin{description}
	\item[AlligatorAgesListener]
		Wenn ein farbiger Alligator frisst, muss er (statt wie in der Entwurfsphase angenommen zu sterben) vorerst altern.
		Um diesen Vorgang zu kommunizieren, z.B. um eine Animation dazu ablaufen zu lassen, exisitiert das hierzugehörige Event.
	\item[BoardEditedListener]
		In der Entwurfsphase wurde vernachlässigt, wie der BoardActor mit Änderungen durch den Nutzer umzugehen hat.
		Da er auch sonst mit Events arbeitet, war es naheliegend, Änderungen zuerst intern durchzuführen und anschließend das Rendering zu benachrichtigen.
		Der Listener bietet dazu Methoden zum Empfangen solcher Benachrichtungen.
	\item[ReplaceEventListener $\rightarrow$ EggHatchListener]
		Mit der Einführung der BoardEdited Events wurden allgemein gehaltene Eventnamen für diese reserviert.
		Durch die Umbenennung entspricht der Name nun auch besser dem genauen Anwendungsszenario.
\end{description}
