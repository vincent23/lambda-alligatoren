\section{Package: data}
\subsection{Lokalisierung}
Das Paket "`data"' wurde lediglich um eine Abstraktionsschicht des Lokalisierungsmechanismus' erweitert.
Die dazugehörigen Klassen heißen:
\begin{description}
	\item[LocalizationBackend] Interface, dass grundlegende Methoden zur Lokalisierung bereitstellt.
	\item[AndroidLocalizationBackend] Implementierung des LocalizationBackend für die Android Plattform
\end{description}

Die Funktionsweise des LocalizationHelpers wurde dahingehend geändert, dass er seine Funktionalität statisch anbietet.
Dies ermöglicht die äußerst dezente Anwendung bei der Lokalisierung der Software, da statische Importe verwendet und unnötig lange Bezeichner vermieden werden können.
Die oben bereits genannten Backends werden einmalig zum Programmstart initialisiert.


\subsection{AssetManager}
Der AssetManager stellt eine Erweiterung des von libgdx gelieferten AssetManagers dar.
Er kümmert sich zusätzlich um die Verwaltung von zur Laufzeit generierten Texturen, wie sie etwa für die Hintergründe farbiger Spielfeldelemente (BoardObjectActors) benötigt werden.
