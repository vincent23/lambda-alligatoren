\section{Package: game.board.operations (ehemals game.visitor)}
Alle Klassen aus dem ehemaligen Paket \emph{game.visitor} wurden entweder in dieses, oder in das untergeordnete Paket \emph{game.board.operations.validation} verschoben.
Dabei wurde das Postfix "`Visitor"' vom Klassennamen gestrichen.
Bei neu hinzugefügten \emph{BoardObjectVisitor} werden im Folgenden die Methoden des Interface nicht extra aufgeführt.

\subsection{CountBoardObjects}

\subsubsection{Modifizierte Methoden}
\begin{description}
\item[count]
Neue Version hinzugefügt, die eine Konfiguration der gezählten \emph{BoardObject}-Objekte erlaubt.
\end{description}


\subsection{CreateDepthMap}
Neu für den Renderer hinzugefügter \emph{BoardObjectVisitor} um die Höhe der Hierarchie über Objekten zu berechnen.

\subsubsection{Hinzugefügte Methoden}
\begin{description}
\item[create]
Gibt eine Abbildung von allen Objekten in der übergebenen Hierarchie und ihren berechneten Höhen zurück.
\end{description}


\subsection{CreateHeightMap}
Neu für den Renderer hinzugefügter \emph{BoardObjectVisitor} um die Höhe der Hierarchie unter Objekten zu berechnen.

\subsubsection{Hinzugefügte Methoden}
\begin{description}
\item[create]
Gibt eine Abbildung von allen Objekten in der übergebenen Hierarchie und ihren berechneten Höhen zurück.
\end{description}


\subsection{CreateWidthMap}
Neu für den Renderer hinzugefügter \emph{BoardObjectVisitor} um die Breite von Objekten, inklusive der von ihren Kindelementen, zu berechnen.

\subsubsection{Hinzugefügte Methoden}
\begin{description}
\item[create]
Gibt eine Abbildung von allen Objekten in der übergebenen Hierarchie und ihren berechneten Breiten zurück.
\end{description}


\subsection{ExchangeColor}
Neu hinzugefügter \emph{BoardObjectVisitor}, um eine Farbe in einer Familie durch eine andere zu ersetzen.

\subsubsection{Hinzugefügte Methoden}
\begin{description}
\item[recolor]
Färbt alle Alligatoren und Eier der übergebenen Farbe in der übergebenen Familie durch die zweite übergebene Farbe um.

\end{description}


\subsection{FlattenTree}

\subsubsection{Hinzugefügte Methoden}
\begin{description}
\item[toList]
Gleiches Verhalten wie \emph{toArray}, nur wird eine Liste anstatt eines Arrays zurückgegeben.
\end{description}

\subsubsection{Modifizierte Methoden}
\begin{description}
\item[toArray]
Umbenannt von \emph{flatten}. Hinzufügen einer Variante, die nur \emph{InternalBoardObject} arbeitet.
\end{description}


\subsection{GetParentHierarchy}
Neu hinzugefügter \emph{BoardObjectVisitor}, um eine Liste der Elternelemente eines Objektes in der Hierarchie zu ermitteln.

\subsubsection{Hinzugefügte Methoden}
\begin{description}
\item[get]
Gibt die Liste der Elternelemente des übergebenen Objekts zurück.
\end{description}


\subsection{RemoveAgedAlligators}

\subsubsection{Modifizierte Methoden}
\begin{description}
\item[remove]
Hinzufügen einer Variante, bei der kein \emph{BoardEventMessenger} übergeben werden muss.
\end{description}


\subsection{RemoveUselessAgedAlligators}
Neu hinzugefügter \emph{BoardObjectVisitor}, um auch solche \emph{AgedAlligator} zu entfernen, die durch die Assoziativität unnötig sind.

\subsubsection{Hinzugefügte Methoden}
\begin{description}
\item[remove]
Entfernt alle \emph{AgedAlligator} aus der übergebenen Konstellation, die durch die Assoziativität unnötig sind.
\end{description}



\subsection{ReplaceEggs}

\subsubsection{Modifizierte Methoden}
\begin{description}
\item[replace]
Hinzufügen verschiedene Varianten, bei denen kein \emph{ColorController} oder kein \emph{BoardEventMessenger} übergeben werden muss.
\end{description}
