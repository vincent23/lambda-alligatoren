\section{Package: game.board}

\subsection{AgedAlligator}

\subsubsection{Modifizierte Methoden}
\begin{description}
\item[Konstruktor]
Hinzufügen der Flags \emph{movable} und \emph{removable}.
\end{description}

\subsubsection{Entfernte Methoden}
\begin{description}
\item[getParent]
Bereits in der übergeordneten Klasse \emph{Alligator} implementiert.
\item[setParent]
Bereits in der übergeordneten Klasse \emph{Alligator} implementiert.
\item[isMovable]
Bereits in der übergeordneten Klasse \emph{Alligator} implementiert.
\item[isRemovable]
Bereits in der übergeordneten Klasse \emph{Alligator} implementiert.
\end{description}


\subsection{Alligator}

\subsubsection{Hinzugefügte Methoden}
\begin{description}
\item[isMovable]
Aus den erbenden Klassen \emph{AgedAlligator} und \emph{ColoredAlligator} verschoben.
\item[isRemovable]
Aus den erbenden Klassen \emph{AgedAlligator} und \emph{ColoredAlligator} verschoben.
\end{description}

\subsubsection{Modifizierte Methoden}
\begin{description}
\item[Konstruktor]
Hinzufügen der Flags \emph{movable} und \emph{removable}.
\end{description}


\subsection{BoardObject}

\subsubsection{Hinzugefügte Methoden}
\begin{description}
\item[match]
Methode, um verschiedene \emph{BoardObject}-Objekte auf Äquivalenz zu testen.
\item[matchWithRecoloring]
Methode, um verschiedene \emph{BoardObject}-Objekte auf Äquivalenz zu testen, unter Berücksichtigung von möglicher \(\alpha\)-Konversion.
\end{description}


\subsection{ColoredAlligator}

\subsubsection{Hinzugefügte Methoden}
\begin{description}
\item[match]
Zur Implementierung der Änderungen am Interface \emph{BoardObject}.
\item[matchWithRecoloring]
Zur Implementierung der Änderungen am Interface \emph{BoardObject}.
\end{description}

\subsubsection{Modifizierte Methoden}
\begin{description}
\item[Konstruktor]
Hinzufügen der Flags \emph{movable} und \emph{removable}.
\end{description}


\subsection{ColoredBoardObject}
Erweitert nun auch das Interface \emph{InternalBoardObject}.


\subsection{Egg}

\subsubsection{Hinzugefügte Methoden}
\begin{description}
\item[match]
Zur Implementierung der Änderungen am Interface \emph{BoardObject}.
\item[matchWithRecoloring]
Zur Implementierung der Änderungen am Interface \emph{BoardObject}.
\end{description}

\subsubsection{Modifizierte Methoden}
\begin{description}
\item[Konstruktor]
Hinzufügen der Flags \emph{movable} und \emph{removable}
\end{description}


\subsection{InternalBoardObject}

\subsubsection{Hinzugefügte Methoden}
\begin{description}
\item[copy]
Zur Konkretisierung des Rückgabewertes der Methode aus BoardObject (Kovarianz).
\end{description}


\subsection{NoSuchChildException}
Neu hinzugefügte Exception, die dann ausgelöst wird, wenn versucht wird auf ein nicht vorhandes Kindelement eines \emph{Parent}-Objekts zuzugreifen.


\subsection{Parent}
Implementiert nun das Interface \emph{BoardObject} sowie \emph{Iterable<InternalBoardObject>}.

\subsubsection{Hinzugefügte Methoden}
\begin{description}
\item[insertChild]
Methode, um ein Kindobjekt an einer festen Stelle einzufügen.
\item[getChildPosition]
Methode, um die Position eines Kindobjektes zu bestimmen.
\item[getChildAtPosition]
Methode, um ein Kindobjekt an einer festen Position zu erhalten.
\item[getFirstChild]
Methode, um das erste Kindobjekt zu erhalten.
\item[getChildCount]
Methode, um die Anzahl der Kindobjekte zu bestimmen.
\item[clearChildren]
Methode, um alle Kindobjekte zu entfernen.
\item[acceptOnChildren]
Methode, die einen \emph{BoardObjectVisitor} entgegennimmt und ihn allen Kindelementen übergibt.
\item[match]
Zur Implementierung der Änderungen am Interface \emph{BoardObject}.
\item[matchWithRecoloring]
Zur Implementierung der Änderungen am Interface \emph{BoardObject}.
\end{description}

\subsubsection{Modifizierte Methoden}
\begin{description}
\item[Konstruktor]
Erweiterung um einen Copy-Konstruktor.
\item[replaceChild]
Umbenannt von \emph{replaceChildWith}.
\item[iterator]
Umbenannt von \emph{getIterator}, außerdem Ergänzung um eine Version mit Startindex.
\item[getChildAfter]
Umbenannt von \emph{getNextChild}.
\end{description}
